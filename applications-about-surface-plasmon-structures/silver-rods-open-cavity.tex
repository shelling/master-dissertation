\section{Silver Rods Open Cavity}
Using nano-scale structure to design devices for confining or guiding light may be the major challenges in optical
researches. When the surface plasmon resonance provides dramatic local-field enhancement, a series of silver rods can
has the ability to guide light in finite length through near-field coupling. In additions, Confinement of light can be
also obseved by arrange silver rods to enclose space [Ng, 2006]. Local-field enhancement plays other important role like
gain media in laser that mean the enclosed space can form an open cavity. Here we study the phasor characteristics of a
silver rods open cavity proposed recently.

The open cavity forms of three silver nanocylinder pairs. Fig 4.5 shows the geometry configuration of three pairs of
silver nanocylinders. The interparticle distance is $d$. The interpair distance is $d_p$. The radius of nanocylinders is
$r$. The three pair structures are illuminated with a plane wave of $\mathrm{TE_z}$ polarized mode. Drude Model is used
again in this simulation but parameters is set using the experimental data from Palik. Computational region is set to be
500 nm $\times$ 500 nm and 20 nm UPML is surrounded outside whole region.

Due to the fact the optical response of nanoparticles would act as dipoles when the size of nanoparticles are far
smaller than wavelength of impinged plane wave. The interaction between nanoparticles is similar to the interaction
between dipoles. The higher multipole behavior becomes apparent with increasing size of nanoparticles. For the reason,
the radius of nanocylinders should change the near-field optical response of the open cavity.

Previous study shows the maximum intensity in chamber occrurs when the ratio of wavelength and nanocylinder radius is
around 11 and 12. Two remarkable cases are also shown in phasor distribution. The first case impinges plane wave in
wavelength 460 nm and the six nanocylinders are set as $r = 36$ nm, $d = d_p = 20$ nm, which the ratio is $\lambda /r
= 12.7$. The second case impinges plane wave in wavelength 650 nm and nanocylinders are set as $r = 58$ nm, $d = d_p =
20$ nm, which the ratio is $\lambda /r = 11.2$ . For futher testing of performance of the yaFDTD framework we directly
perform this two cases again because the numerical simulation error usually arises as the field intensity is enhanced
severely.

Fig 4.6 and 4.11 shows the steady state total E field phasor distribution for the first and second case mentioned above
respectively. Comparing with the result shown in previous study (Fig 4.7 and Fig 4.12), the phasor distribution of
yaFDTD can fit it good. 

Fig 4.8 and 4.9 also shown the the partial E field phasor distribution for $\mathrm{E_x}$ and $\mathrm{E_y}$ in the
first case. Fig 4.13 and Fig 4.14 are for the second case. It can be observed the $\mathrm{E_x}$ component is dominant
in the gaps between pairs and the $\mathrm{E_y}$ component is dominant inside the gaps of each pairs. This shows that we
can partially manipulate the $\mathrm{E_x}$ and $\mathrm{E_y}$ by change the distance in pairs $d$ and distance between
pairs $d_p$ to reduce the interaction of dipoles.  For example, when we change the $d_p$ to be 40 nm in the first case
(Fig 4.15), the $\mathrm{E_x}$ would almost disappear (Fig 4.18 and Fig 4.19). The |$\mathrm{E}$| of yaFDTD and previous
study are also shown in Fig 4.16 and Fig 4.17.

\documentclass{book}

\usepackage{fontspec}
\usepackage{wallpaper}
\usepackage{color}
\usepackage{xunicode}
\usepackage{xltxtra}
\usepackage{graphicx}
\usepackage[margin=3cm]{geometry}
\usepackage{type1cm}


\newfontfamily\msjh{Microsoft JhengHei}
\newfontfamily\kai{BiauKai}
\setmonofont[SizeFeatures={Size={9}}]{Monaco}
% \newfontfamily\arial{Arial}
\XeTeXlinebreaklocale "zh"
\setlength{\parindent}{0pt}



\begin{document}
\pagenumbering{roman}
\fontsize{12}{2em}\selectfont

\title{
  \kai 國立台灣大學電機資訊學院光電工程學研究所\\碩士論文\\
  \rm Graduate Institute of Photonics and Optoelectronics\\College of Electrical Engineering and Computer Science\\National Taiwan University\\Master Dissertation\\[1cm]
  \kai 以有限時域差分法分析表面電漿子\\
  \rm Finite-Difference Time-Domain Analysis of Plasmanic Structures
}
\author{
  \kai 許家瑋\\ \rm Jia-Wei Hsu\\\\
  \kai 指導教授:張宏鈞 博士\\ \rm Advisor: Hung-Chun Chang, Ph.D.
}
\date{
  \kai 中華民國 100 年 7 月\\ 
  \rm July 2011
}

\maketitle
\CenterWallPaper{0.4}{"ntu.jpg"}



\chapter*{Acknownledgement}
Thanks to ...


\chapter*{\kai 致謝}
\kai
兩年
\rm

\chapter*{Abstract}
The thesis is splited into two part. The first part describes the formulas of common FDTD components and introduces a real
world implementation. The second part discusses some surface plasmonic structures via the result of its simulations.

\tableofcontents

%% \setlength{\parskip}{0em}

\chapter{Introduction}
\pagenumbering{arabic}
\setcounter{page}{1}
\section{Motivations}
More Lorem ipsum dolor sit amet, consectetur adipiscing elit. Phasellus nec ligula a tortor mattis
consectetur. Phasellus eget dictum quam. Pellentesque cursus, lacus ut rutrum suscipit, dolor sapien varius nisl, sed
aliquam mi augue non magna. Maecenas dignissim aliquet porta. Sed elit purus, vestibulum a posuere ut, volutpat eget
nibh. Phasellus eu dolor ante, a tincidunt massa. Pellentesque porttitor pharetra risus. Sed a sem neque. Etiam varius
rutrum consequat. Donec sagittis nulla sed lectus tristique iaculis. Aliquam placerat sagittis enim nec
aliquam. Phasellus in erat metus. Cum sociis natoque penatibus et magnis dis parturient montes, nascetur ridiculus mus.

Nullam nisl erat, pulvinar a fermentum ut, tempus eget enim. Vivamus vel odio id urna ultricies sollicitudin. Curabitur
lobortis augue rhoncus purus sollicitudin ornare. Nullam sit amet quam quis neque mollis elementum. Cras dapibus felis
eu mi vulputate ut rutrum dui molestie. Ut at venenatis purus. Nunc eget lacus blandit enim pharetra congue posuere in
arcu. Duis pharetra, mi vitae venenatis pellentesque, quam nulla vulputate dui, at facilisis tortor tortor non orci. Sed
lectus erat, suscipit quis aliquam et, lobortis at lorem. Quisque feugiat neque eros. Nunc rutrum adipiscing dolor eu
pulvinar. Nunc elit diam, tincidunt at vulputate ornare, eleifend id ante. Aliquam a augue augue, a hendrerit eros.

Praesent lectus enim, tincidunt id volutpat sit amet, varius sit amet lectus. Nam bibendum consequat tellus accumsan
pellentesque. Nullam enim dolor, eleifend eu faucibus non, convallis congue nibh. Pellentesque vel urna ac lorem congue
vehicula non ut massa. Quisque a metus et est faucibus sollicitudin viverra ut purus. Nullam a leo sit amet ipsum
sagittis volutpat ut at tortor. Nullam quis adipiscing arcu. Nulla facilisi. Integer at dui turpis. Suspendisse luctus
rutrum dui ac mattis. In hac habitasse platea dictumst. Duis pharetra fringilla nulla, at molestie arcu fermentum
eu. Integer id metus vel enim auctor rutrum.

In hac habitasse platea dictumst. Pellentesque elementum dolor vel felis tincidunt in mollis lorem ornare. Fusce non
ligula massa, varius tempus eros. Quisque nunc magna, facilisis at luctus ut, venenatis vel dolor. Mauris nec accumsan
nulla. Fusce in lorem velit, non egestas mauris. Cras laoreet erat eu leo commodo aliquam. In eget mauris lacus, vitae
scelerisque lacus. Proin id massa augue. Ut ac euismod risus. Integer vulputate turpis eu urna pulvinar sit amet ornare
elit sagittis. Nunc mattis enim vel lectus mollis vitae viverra mauris sollicitudin. Proin id dui nunc. Ut ut quam
iaculis est tincidunt venenatis. Nunc at felis in augue suscipit scelerisque. Pellentesque eget tellus at odio malesuada
mollis aliquet ut magna. Suspendisse velit augue, rhoncus ut tempor non, tincidunt vel ligula. Proin elementum arcu sit
amet eros hendrerit lacinia eget et lectus. Donec non justo vitae ipsum consequat scelerisque. Nunc tincidunt, lacus non
fermentum molestie, tellus nibh aliquet diam, ac congue velit lacus eu diam.

Intro to \TeX\footnote{footnote}\marginpar{margin par}\\
\textit{The TeX Book}\footnote{The TeX Book}\\
\msjh 中文\rmfamily \\
\kai 楷書\rmfamily \\
\textcolor{blue}{blue}\\
\fbox{float box}\\ \\
baseline \raisebox{1ex}{upward}\raisebox{-1ex}{downward}\\\\
math expression $ (y^m)^n $ \\\\

tabbing
\begin{tabbing}
  column1 \= column2 \= column3 \\
  item1   \> item2   \> item3   \\
\end{tabbing}

tabular\\
\begin{tabular}{|l|c|r|}
  \hline
  \multicolumn{3}{|c|}{Sample Tabular}\\
  \hline
  first & second & third \\
  \cline{2-3}
  left  & centered & right \\
  \hline
\end{tabular}
\\\\
one -\\
two --\\
three ---\\
minus $ - $\\
control\ space\\
final\\
\TeX\ ignore space behind it.\\
\XeTeX\ does.\\
'\TeX'\\
Here is $\pi$ and $\Pi$\\
the angle of whole circle is 2$\pi$.\\
P\'olya\\





\section{Chapter Outline}

In chapter 2, we go through all common components should be implemented. Starting from the basic update equations in
free space, we adjust the formulas to suit the additional component such as Perfectly Matched Layer, Dispersive
Material, etc.



\chapter{The Finite-Different Time-Domain Method}
\section{The Algorithm}

In 1966, The algorithm of the FDTD method was first introduced by Yee.

\subsection{Finite Difference}
The first thing being concerned is how to discrete space and time in FDTD, in other word, how to turn differential
equations to algerba equations.
\subsubsection{Explicit Leapfrog Scheme}
Laylor's series expansion 
\begin{equation}
  u(x_i+\Delta x) = u|_{x_i} + 
  \Delta x\cdot\left.\frac{\partial u}{\partial x}\right|_{x_i} + 
  \frac{(\Delta x)^2}{2}\cdot\left.\frac{\partial ^2 u}{\partial x^2}\right|_{x_i} + 
  \frac{(\Delta x)^3}{6}\cdot\left.\frac{\partial ^3 u}{\partial x^3}\right|_{x_i} + ...
\end{equation}
\begin{equation}
  u(x_i-\Delta x) = u|_{x_i} -
  \Delta x\cdot\left.\frac{\partial u}{\partial x}\right|_{x_i} + 
  \frac{(\Delta x)^2}{2}\cdot\left.\frac{\partial ^2 u}{\partial x^2}\right|_{x_i} -
  \frac{(\Delta x)^3}{6}\cdot\left.\frac{\partial ^3 u}{\partial x^3}\right|_{x_i} + ...
\end{equation}
difference 
\begin{equation}
  u(x_i+\Delta x) - u(x_i-\Delta x) = 2\Delta x\cdot\left.\frac{\partial u}{\partial x}\right|_{x_i}+...
\end{equation}
\begin{equation}
  \left.\frac{\partial u}{\partial x}\right|_{x_i} = \frac{u(x_i+\Delta x) - u(x_i-\Delta x)}{2\Delta x} = \frac{u^{i+1} - u^{i-1}}{2\Delta x} + O[(\Delta x)^3]
\end{equation}
sum 
\begin{equation}
  u(x_i+\Delta x) + u(x_i-\Delta x) = \left.2u\right|_{x_i} + (\Delta x)^2\cdot\left.\frac{\partial ^2 u}{\partial x^2}\right|_{x_i} + ...
\end{equation}
\begin{equation}
  \left.\frac{\partial^2 u}{\partial x^2}\right|_{x_i} = \frac{u(x_i+\Delta x) - 2u(x_i) + u(x_i-\Delta x)}{(\Delta x)^2} = \frac{u^{i+1} - 2u^i + u^{i-1}}{(\Delta x)^2} % + O[(\Delta x)^2]
\end{equation}
\subsubsection{Semi-Implicit Scheme}
\begin{equation}
  \left.\frac{\partial u}{\partial x}\right|_{x_i} = \frac{u(x_i+\Delta x) - u(x_i)}{\Delta x}
\end{equation}

\subsection{The Update Equations}

The update equations is the core of the FDTD method. In every interation on the timeline, the update equations refresh
the value of field of every point in the simulation region.

In this section, we are going to derive update equation from Maxwell's Equations.Here is the most well-known form of
Maxwell's Equations\index{Maxwell's Equations}:
\begin{gather}
  \label{eq:maxwell}
  \begin{array}{@{}rclr@{}}
    \nabla \cdot D & = & \rho_{\nu} & \mathrm{(Gaussian's\ Law)}\\
    \nabla \times E & = & {\displaystyle -\frac{\partial B}{\partial t}} & \mathrm{(Faraday's\ Law)}\\
    \nabla \cdot B & = & 0 & \\
    \nabla \times H & = & {\displaystyle J_s + \frac{\partial D}{\partial t}} & \mathrm{(Amp\`ere's\ Law)}
  \end{array}
\end{gather}
the equations shown above is for simple conductive media, the simplest lossy media having constant conductivity.

for getting sysmetrical form, novel magnetic current $M$ was added into Faraday's Law
\begin{gather}
  \frac{\partial D}{\partial t} = \nabla \times H - J\\
  \frac{\partial B}{\partial t} =  - \nabla \times E - M
\end{gather}
in terms of E and H
\begin{gather}
  \epsilon\frac{\partial E}{\partial t} = \nabla \times H - \sigma_eE\\
  \mu\frac{\partial H}{\partial t} = - \nabla \times E - \sigma_hH
\end{gather}
gather coefficient 
\begin{gather}
  \frac{\partial E}{\partial t} = \frac{1}{\epsilon_r\epsilon_0}\nabla\times H - \frac{\sigma_e}{\epsilon_r\epsilon_0}E\\
  \frac{\partial H}{\partial t} = - \frac{1}{\mu_r\mu_0}\nabla\times E - \frac{\sigma_h}{\mu_r\mu_0}H
\end{gather}
turn to Gaussian Unit 
\begin{equation}
  \label{eq:gaussian_unit}
  \begin{array}{@{}l@{}}
    {\displaystyle\widetilde{E} = \sqrt{\frac{\epsilon_0}{\mu_0}}E}\\
    {\displaystyle\widetilde{D} = \frac{1}{\sqrt{\epsilon_0\mu_0}}D}\\
    {\displaystyle\widetilde{B} = \frac{1}{\mu_0}B}
  \end{array}
\end{equation}
\begin{gather}
  \frac{\partial \widetilde{E}}{\partial t} = \frac{1}{\epsilon_r\sqrt{\mu_0\epsilon_0}}\nabla\times H - \frac{\sigma_e}{\epsilon_r\epsilon_0}\widetilde{E}\\
  \frac{\partial H}{\partial t} = - \frac{1}{\mu_r\sqrt{\mu_0\epsilon_0}}\nabla\times\widetilde{E} - \frac{\sigma_h}{\mu_r\mu_0}H
\end{gather}
\begin{gather}
  \left(\epsilon_r\frac{\partial}{\partial t} + \frac{\sigma_e}{\epsilon_0}\right)\widetilde{E} = \frac{1}{\sqrt{\mu_0\epsilon_0}}\nabla\times H\\
  \left(\mu_r\frac{\partial}{\partial t} + \frac{\sigma_h}{\mu_0}\right)H = - \frac{1}{\sqrt{\mu_0\epsilon_0}}\nabla\times\widetilde{E}
\end{gather}
$\frac{\partial}{\partial t} \rightarrow j\omega$
\begin{gather}
  j\omega\left(\epsilon_r + \frac{\sigma_e}{j\omega\epsilon_0}\right)\widetilde{E} = c\ \nabla\times H\\
  j\omega\left(\mu_r + \frac{\sigma_h}{j\omega\mu_0}\right)H = - c\ \nabla\times\widetilde{E}
\end{gather}
update equations and constitute relations in simple conductive media under Gaussian Units.
\begin{gather}
  \frac{\partial}{\partial t}\widetilde{D} = \frac{1}{\sqrt{\mu_0\epsilon_0}}\nabla\times H\label{eq:up_d}\\
  \widetilde{D}(\omega) = \left(\epsilon_r + \frac{\sigma_e}{j\omega\epsilon_0}\right)\widetilde{E} = \epsilon_r^*(\omega)\widetilde{E}(\omega)\label{eq:cr_d}\\
  \frac{\partial}{\partial t}\widetilde{B} = -\frac{1}{\sqrt{\mu_0\epsilon_0}}\nabla\times\widetilde{E}\label{eq:up_b}\\
  \widetilde{B}(\omega) = \left(\mu_r + \frac{\sigma_h}{j\omega\mu_0}\right)H = \mu_r^*(\omega)H(\omega)\label{eq:cr_b}
\end{gather}
It's obvious separating constitute relations from updating of electric flux $D$ and magnetic flux $B$. The
material-related coefficients were collected into constitute relations to handle different objects, so that no matter
what object was changed in region of simulation Eq.\ref{eq:up_d} and Eq.\ref{eq:up_b} keep in this form.

For example, The formulas can be simpilified to describe loseless dielectric media by setting $\sigma_e$, $\sigma_h$ as
zero in Eq.\ref{eq:cr_d} and Eq.\ref{eq:cr_b} to be
\begin{gather*}
  \widetilde{D}(\omega) = \epsilon_r\cdot\widetilde{E}(\omega)\\
  \widetilde{B}(\omega) = \mu_r\cdot H(\omega)
\end{gather*}
Performing inverse Fourier Transformation
\begin{gather*}
  \widetilde{D}(t) = \epsilon_r\cdot\widetilde{E}(t)\\
  \widetilde{B}(t) = \mu_r\cdot H(t)
\end{gather*}
but Eq.\ref{eq:up_d} and Eq.\ref{eq:up_b} need not any modification.

Or to describe freespace by setting $\sigma_e$, $\sigma_h$ as zero and $\epsilon_r$, $\mu_r$ as one 
\begin{gather*}
  \widetilde{D}(\omega) = \widetilde{E}(\omega)\\
  \widetilde{B}(\omega) = H(\omega)
\end{gather*}
Performing inverse Fourier Transformation again
\begin{gather*}
  \widetilde{D}(t) = \widetilde{E}(t)\\
  \widetilde{B}(t) = H(t)
\end{gather*}
In general, every material has its own $\epsilon_r^*(\omega)$ varying through whole frequency spectrum duo to its own
characters. By applying some mathematical trick Eq.\ref{eq:cr_d} and Eq.\ref{eq:cr_b} can be specialized for different
material to retrieve $E$ from $D$ in every time step but Eq.\ref{eq:up_d} and Eq.\ref{eq:up_b} can be applied directly
on every kinds of material. That's the best advanteage separating constitute relations out of the two update equations
would be introduce in \ref{sec:dispersive}.

This way also shows some advantagewhen handling perfecly matched layer which would be discussed in \ref{subsec:pml}.

Extend to Cartesian coordinate system.
\begin{gather}
  \frac{\partial}{\partial t}\widetilde{D}_x = \frac{1}{\sqrt{\mu_0\epsilon_0}}\left(\frac{\partial H_z}{\partial y} - \frac{\partial H_y}{\partial z}\right)\label{eq:up_d_x}\\
  \frac{\partial}{\partial t}\widetilde{D}_y = \frac{1}{\sqrt{\mu_0\epsilon_0}}\left(\frac{\partial H_x}{\partial z} - \frac{\partial H_z}{\partial x}\right)\label{eq:up_d_y}\\
  \frac{\partial}{\partial t}\widetilde{D}_z = \frac{1}{\sqrt{\mu_0\epsilon_0}}\left(\frac{\partial H_y}{\partial x} - \frac{\partial H_x}{\partial y}\right)\label{eq:up_d_z}\\
  \frac{\partial}{\partial t}\widetilde{B}_x =-\frac{1}{\sqrt{\mu_0\epsilon_0}}\left(\frac{\partial \widetilde{E}_z}{\partial y} - \frac{\partial \widetilde{E}_y}{\partial z}\right)\label{eq:up_b_x}\\
  \frac{\partial}{\partial t}\widetilde{B}_y =-\frac{1}{\sqrt{\mu_0\epsilon_0}}\left(\frac{\partial \widetilde{E}_x}{\partial z} - \frac{\partial \widetilde{E}_z}{\partial x}\right)\label{eq:up_b_y}\\
  \frac{\partial}{\partial t}\widetilde{B}_z =-\frac{1}{\sqrt{\mu_0\epsilon_0}}\left(\frac{\partial \widetilde{E}_y}{\partial x} - \frac{\partial \widetilde{E}_x}{\partial y}\right)\label{eq:up_b_z}
\end{gather}
\clearpage
Discrete use semi-implicit scheme
\begin{gather}
  \begin{array}{@{}l@{}}
    \displaystyle \frac{\widetilde{D}_x|_{i+\frac{1}{2},j,k}^{n+\frac{1}{2}} - \widetilde{D}_x|_{i+\frac{1}{2},j,k}^{n-\frac{1}{2}}}{\Delta t} = \\
    \displaystyle c_0\left(\frac{H_z|_{i+\frac{1}{2},j+\frac{1}{2},k}^{n} - H_z|_{i+\frac{1}{2},j-\frac{1}{2},k}^{n}}{\Delta y} - \frac{H_y|_{i+\frac{1}{2},j,k+\frac{1}{2}}^{n} - H_y|_{i+\frac{1}{2},j,k-\frac{1}{2}}^{n}}{\Delta z}\right)\\[3em]
    \displaystyle \frac{\widetilde{D}_y|_{i,j+\frac{1}{2},k}^{n+\frac{1}{2}} - \widetilde{D}_x|_{i,j+\frac{1}{2},k}^{n-\frac{1}{2}}}{\Delta t} = \\
    \displaystyle c_0\left(\frac{H_x|_{i,j+\frac{1}{2},k+\frac{1}{2}}^{n} - H_x|_{i,j+\frac{1}{2},k-\frac{1}{2}}^{n}}{\Delta z} - \frac{H_z|_{i+\frac{1}{2},j+\frac{1}{2},k}^{n} - H_z|_{i-\frac{1}{2},j+\frac{1}{2},k}^{n}}{\Delta x}\right)\\[3em]
    \displaystyle \frac{\widetilde{D}_z|_{i,j,k+\frac{1}{2}}^{n+\frac{1}{2}} - \widetilde{D}_z|_{i,j,k+\frac{1}{2}}^{n-\frac{1}{2}}}{\Delta t} = \\
    \displaystyle c_0\left(\frac{H_y|_{i+\frac{1}{2},j,k+\frac{1}{2}}^{n} - H_y|_{i-\frac{1}{2},j,k+\frac{1}{2}}^{n}}{\Delta x} - \frac{H_x|_{i,j+\frac{1}{2},k+\frac{1}{2}}^{n} - H_x|_{i,j-\frac{1}{2},k+\frac{1}{2}}^{n}}{\Delta y}\right)\\[3em]
    \displaystyle \frac{\widetilde{B}_x|_{i,j+\frac{1}{2},k+\frac{1}{2}}^{n+1} - \widetilde{B}_x|_{i,j+\frac{1}{2},k+\frac{1}{2}}^{n}}{\Delta t} = \\
    \displaystyle - c_0\left(\frac{\widetilde{E}_z|_{i,j+1,k+\frac{1}{2}}^{n+\frac{1}{2}} - \widetilde{E}_z|_{i,j,k+\frac{1}{2}}^{n+\frac{1}{2}}}{\Delta y} - \frac{\widetilde{E}_y|_{i,j+\frac{1}{2},k+1}^{n+\frac{1}{2}} - \widetilde{E}_y|_{i,j+\frac{1}{2},k}^{n+\frac{1}{2}}}{\Delta z}\right)\\[3em]
    \displaystyle \frac{\widetilde{B}_y|_{i+\frac{1}{2},j,k+\frac{1}{2}}^{n+1} - \widetilde{B}_y|_{i+\frac{1}{2},j,k+\frac{1}{2}}^{n}}{\Delta t} = \\
    \displaystyle - c_0\left(\frac{\widetilde{E}_x|_{i+\frac{1}{2},j,k+1}^{n+\frac{1}{2}} - \widetilde{E}_x|_{i+\frac{1}{2},j,k}^{n+\frac{1}{2}}}{\Delta z} - \frac{\widetilde{E}_z|_{i+1,j,k+\frac{1}{2}}^{n+\frac{1}{2}} - \widetilde{E}_z|_{i,j,k+\frac{1}{2}}^{n+\frac{1}{2}}}{\Delta x}\right)\\[3em]
    \displaystyle \frac{\widetilde{B}_z|_{i+\frac{1}{2},j+\frac{1}{2},k}^{n+1} - \widetilde{B}_z|_{i+\frac{1}{2},j+\frac{1}{2},k}^{n}}{\Delta t} = \\
    \displaystyle - c_0\left(\frac{\widetilde{E}_y|_{i+1,j+\frac{1}{2},k}^{n+\frac{1}{2}} - \widetilde{E}_y|_{i,j+\frac{1}{2},k}^{n+\frac{1}{2}}}{\Delta x} - \frac{\widetilde{E}_x|_{i+\frac{1}{2},j+1,k}^{n+\frac{1}{2}} - \widetilde{E}_x|_{i+\frac{1}{2},j,k}^{n+\frac{1}{2}}}{\Delta y}\right)
  \end{array}
\end{gather}
Throughout this thesis, we use regular grid, that is, $\Delta x = \Delta y = \Delta z$. That means we could rewrite update equations as
\begin{equation}
  \begin{split}
    \widetilde{D}_x|_{i+\frac{1}{2},j,k}^{n+\frac{1}{2}} & = \widetilde{D}_x|_{i+\frac{1}{2},j,k}^{n-\frac{1}{2}}\\
    & + \frac{c_0\Delta t}{\Delta x}\left(H_z|_{i+\frac{1}{2},j+\frac{1}{2},k}^{n} - H_z|_{i+\frac{1}{2},j-\frac{1}{2},k}^{n} - H_y|_{i+\frac{1}{2},j,k+\frac{1}{2}}^{n} + H_y|_{i+\frac{1}{2},j,k-\frac{1}{2}}^{n}\right)
  \end{split}
\end{equation}
\begin{equation}
  \begin{split}
    \widetilde{D}_y|_{i,j+\frac{1}{2},k}^{n+\frac{1}{2}} & = \widetilde{D}_x|_{i,j+\frac{1}{2},k}^{n-\frac{1}{2}}\\
    & + \frac{c_0\Delta t}{\Delta x}\left(H_x|_{i,j+\frac{1}{2},k+\frac{1}{2}}^{n} - H_x|_{i,j+\frac{1}{2},k-\frac{1}{2}}^{n} - H_z|_{i+\frac{1}{2},j+\frac{1}{2},k}^{n} + H_z|_{i-\frac{1}{2},j+\frac{1}{2},k}^{n}\right)
  \end{split}
\end{equation}
\begin{equation}
  \begin{split}
    \widetilde{D}_z|_{i,j,k+\frac{1}{2}}^{n+\frac{1}{2}} & = \widetilde{D}_z|_{i,j,k+\frac{1}{2}}^{n-\frac{1}{2}}\\
    & + \frac{c_0\Delta t}{\Delta x}\left(H_y|_{i+\frac{1}{2},j,k+\frac{1}{2}}^{n} - H_y|_{i-\frac{1}{2},j,k+\frac{1}{2}}^{n} - H_x|_{i,j+\frac{1}{2},k+\frac{1}{2}}^{n} + H_x|_{i,j-\frac{1}{2},k+\frac{1}{2}}^{n}\right)
  \end{split}
\end{equation}
\begin{equation}
  \begin{split}
    \widetilde{B}_x|_{i,j+\frac{1}{2},k+\frac{1}{2}}^{n+1} & = \widetilde{B}_x|_{i,j+\frac{1}{2},k+\frac{1}{2}}^{n}\\
    & - \frac{c_0\Delta t}{\Delta x}\left(\widetilde{E}_z|_{i,j+1,k+\frac{1}{2}}^{n+\frac{1}{2}} - \widetilde{E}_z|_{i,j,k+\frac{1}{2}}^{n+\frac{1}{2}} - \widetilde{E}_y|_{i,j+\frac{1}{2},k+1}^{n+\frac{1}{2}} - \widetilde{E}_y|_{i,j+\frac{1}{2},k}^{n+\frac{1}{2}}\right)
  \end{split}
\end{equation}
\begin{equation}
  \begin{split}
    \widetilde{B}_y|_{i+\frac{1}{2},j,k+\frac{1}{2}}^{n+1} & = \widetilde{B}_y|_{i+\frac{1}{2},j,k+\frac{1}{2}}^{n}\\
    & - \frac{c_0\Delta t}{\Delta x}\left(\widetilde{E}_x|_{i+\frac{1}{2},j,k+1}^{n+\frac{1}{2}} - \widetilde{E}_x|_{i+\frac{1}{2},j,k}^{n+\frac{1}{2}} - \widetilde{E}_z|_{i+1,j,k+\frac{1}{2}}^{n+\frac{1}{2}} - \widetilde{E}_z|_{i,j,k+\frac{1}{2}}^{n+\frac{1}{2}}\right)
  \end{split}
\end{equation}
\begin{equation}
  \begin{split}
    \widetilde{B}_z|_{i+\frac{1}{2},j+\frac{1}{2},k}^{n+1} & = \widetilde{B}_z|_{i+\frac{1}{2},j+\frac{1}{2},k}^{n}\\
    & - \frac{c_0\Delta t}{\Delta x}\left(\widetilde{E}_y|_{i+1,j+\frac{1}{2},k}^{n+\frac{1}{2}} - \widetilde{E}_y|_{i,j+\frac{1}{2},k}^{n+\frac{1}{2}} - \widetilde{E}_x|_{i+\frac{1}{2},j+1,k}^{n+\frac{1}{2}} - \widetilde{E}_x|_{i+\frac{1}{2},j,k}^{n+\frac{1}{2}}\right)
  \end{split}
\end{equation}
This is complete update equations derived for 3D cases.
\begin{equation}
  k+\frac{1}{2}\rightarrow k\quad \mathrm{and} \quad
  k-\frac{1}{2}\rightarrow k-1
\end{equation}
Pseudo code
\begin{code}
Points.each do 
  Dx[i,j,k] += 0.5 * ( Hz[i,j,k] - Hz[i,j-1,k] 
                     - Hy[i,j,k] + Hy[i,j,k-1] )
  Dy[i,j,k] += 0.5 * ( Hx[i,j,k] - Hx[i,j,k-1] 
                     - Hz[i,j,k] + Hz[i-1,j,k] )
  Dz[i,j,k] += 0.5 * ( Hy[i,j,k] - Hy[i-1,j,k] 
                     - Hx[i,j,k] + Hx[i,j-1,k] )
  Bx[i,j,k] -= 0.5 * ( Ez[i,j+1,k] - Ez[i,j,k] 
                     - Ey[i,j,k+1] + Ey[i,j,k] )
  By[i,j,k] -= 0.5 * ( Ex[i,j,k+1] - Ex[i,j,k] 
                     - Ez[i+1,j,k] + Ez[i,j,k] )
  Bz[i,j,k] -= 0.5 * ( Ey[i+1,j,k] - Ey[i,j,k] 
                     - Ex[i,j+1,k] + Ex[i,j,k] )
end
\end{code}


\subsection{Stability}
Courant Conditions, Courant Number
\begin{equation}
  \Delta t \le \frac{\Delta x}{\sqrt{n}\cdot c_0}
\end{equation}
where n is the dimension of the simulation. For the convenience of designing mentioned latter, throughout this thesis we determine
$\Delta t$ by
\begin{equation}
  \Delta t = \frac{\Delta x}{2 \cdot c_0}
\end{equation}


\subsection{Reduction to One Dimensions}
There are three selections to choose a one dimension EM string: $\mathrm{TEM_x}$ ($\mathrm{E_{y}}$, $\mathrm{H_{z}}$,
$\mathrm{k_x}$), $\mathrm{TEM_y}$ ($\mathrm{E_z}$, $\mathrm{H_x}$, $\mathrm{k_y}$), and $\mathrm{TEM_z}$
($\mathrm{E_x}$, $\mathrm{H_y}$, $\mathrm{k_z}$). Similarly, $\mathrm{TEM_z}$ is the default choice when saying
TEM. Following the definition of TEM, Eq.\ref{eq:up_d_x} and Eq.\ref{eq:up_b_y} were picked out for reduction of 1-D
case.
\begin{gather*}
  \frac{\partial}{\partial t}\widetilde{D}_x = \frac{1}{\sqrt{\mu_0\epsilon_0}}\left(\frac{\partial H_z}{\partial y} - \frac{\partial H_y}{\partial z}\right)\\
  \frac{\partial}{\partial t}\widetilde{B}_y =-\frac{1}{\sqrt{\mu_0\epsilon_0}}\left(\frac{\partial \widetilde{E}_x}{\partial z} - \frac{\partial \widetilde{E}_z}{\partial x}\right)
\end{gather*}
The choice implies
\begin{displaymath}
  \frac{\partial}{\partial x} \rightarrow 0\quad \mathrm{and} \quad
  \frac{\partial}{\partial y} \rightarrow 0
\end{displaymath}
apply
\begin{gather}
  \frac{\partial}{\partial t}\widetilde{D}_x = \frac{1}{\sqrt{\mu_0\epsilon_0}}\left( - \frac{\partial H_y}{\partial z}\right)\\
  \frac{\partial}{\partial t}\widetilde{B}_y =-\frac{1}{\sqrt{\mu_0\epsilon_0}}\left(\frac{\partial \widetilde{E}_x}{\partial z} \right)
\end{gather}
Discrete
\begin{gather}
  \frac{\widetilde{D}_x|_k^{n+\frac{1}{2}} - \widetilde{D}_x|_k^{n-\frac{1}{2}}}{\Delta t} = -c_0\cdot\frac{H_y|_{k+\frac{1}{2}}^n - H_y|_{k-\frac{1}{2}}^n}{\Delta z}\\
  \frac{\widetilde{B}_y|_{k+\frac{1}{2}}^{n+1} - \widetilde{B}_y|_{k+\frac{1}{2}}^n}{\Delta t} = -c_0\cdot\frac{\widetilde{E}_x|_{k+1}^{n+\frac{1}{2}} - \widetilde{E}_x|_{k}^{n+\frac{1}{2}}}{\Delta z}
\end{gather}
That is
\begin{gather}
  \widetilde{D}_x|_k^{n+\frac{1}{2}} = \widetilde{D}_x|_k^{n-\frac{1}{2}} - \frac{c_0\Delta t}{\Delta z}\left( H_y|_{k+\frac{1}{2}}^n - H_y|_{k-\frac{1}{2}}^n \right)\\
  \widetilde{B}_y|_{k+\frac{1}{2}}^{n+1} = \widetilde{B}_y|_{k+\frac{1}{2}}^{n} = - \frac{c_0\Delta t}{\Delta z}\left( \widetilde{E}_x|_{k+1}^{n+\frac{1}{2}} - \widetilde{E}_x|_{k}^{n+\frac{1}{2}} \right)
\end{gather}
Here is the code
\begin{code}
Points.each do
  Dx[k] += 0.5 * ( Hy[k-1] - Hy[k] )
  Hy[k] += 0.5 * ( Ex[k] - Ex[k+1] )
end
\end{code}




\subsection{Reduction to Two Dimensions}
There are 6 selections for us to choose a two dimensions EM plane: $\mathrm{TM_{x}} $, $\mathrm{TE_{x}}$,
$\mathrm{TM_{y}}$, $\mathrm{TE_{y}}$, $\mathrm{TM_{z}}$, $\mathrm{TE_{z}}$. By default, the choice in this thesis follow
the book of Taflove using $\mathrm{TM_{z}}$ ($\mathrm{H_x}$, $\mathrm{H_y}$, and $\mathrm{E_z}$) and $\mathrm{TE_{z}}$
($\mathrm{E_x}$, $\mathrm{E_y}$, and $\mathrm{H_z}$) as convention when saying TM and TE.
\begin{displaymath}
  \frac{\partial}{\partial z} \rightarrow 0
\end{displaymath}

$\mathrm{TM_z}$
\begin{displaymath}
  \frac{\partial}{\partial t}\widetilde{D}_z = \frac{1}{\sqrt{\mu_0\epsilon_0}}\left(\frac{\partial H_y}{\partial x} - \frac{\partial H_x}{\partial y}\right)
\end{displaymath}
\begin{displaymath}
  \frac{\partial}{\partial t}\widetilde{B}_x =-\frac{1}{\sqrt{\mu_0\epsilon_0}}\left(\frac{\partial \widetilde{E}_z}{\partial y} - \frac{\partial \widetilde{E}_y}{\partial z}\right)
\end{displaymath}
\begin{displaymath}
  \frac{\partial}{\partial t}\widetilde{B}_y =-\frac{1}{\sqrt{\mu_0\epsilon_0}}\left(\frac{\partial \widetilde{E}_x}{\partial z} - \frac{\partial \widetilde{E}_z}{\partial x}\right)
\end{displaymath}
Discretize
\begin{displaymath}
  \frac{\widetilde{D}_z|_{i,j}^{n+\frac{1}{2}}-\widetilde{D}_z|_{i,j}^{n+\frac{1}{2}}}{\Delta t} =
  c_0 \left(\frac{H_y|_{i+\frac{1}{2},j}^{n} - H_y|_{i-\frac{1}{2},j}^n}{\Delta x} - \frac{H_x|_{i,j+\frac{1}{2}}^{n} - H_x|_{i,j-\frac{1}{2}}^{n}}{\Delta y}\right)
\end{displaymath}
\begin{displaymath}
  \frac{\widetilde{B}_x|_{i,j+\frac{1}{2}}^{n+1} - \widetilde{B}_x|_{i,j+\frac{1}{2}}^{n}}{\Delta t} = 
  - c_0\left(\frac{\widetilde{E}_z|_{i,j+1}^{n+\frac{1}{2}} - \widetilde{E}_z|_{i,j}^{n+\frac{1}{2}}}{\Delta y}\right)
\end{displaymath}
\begin{displaymath}
  \frac{\widetilde{B}_y|_{i+\frac{1}{2},j}^{n+1} - \widetilde{B}_y|_{i+\frac{1}{2},j}^{n}}{\Delta t} =
  - c_0\left( - \frac{\widetilde{E}_z|_{i+1,j}^{n+\frac{1}{2}} - \widetilde{E}_z|_{i,j}^{n+\frac{1}{2}}}{\Delta x}\right)
\end{displaymath}
Pseudo code for $\mathrm{TM_z}$ polarization
\begin{code}
Points.each do 
  
end
\end{code}

$\mathrm{TE_z}$
\begin{displaymath}
    \frac{\partial}{\partial t}\widetilde{B}_z =-\frac{1}{\sqrt{\mu_0\epsilon_0}}\left(\frac{\partial \widetilde{E}_y}{\partial x} - \frac{\partial \widetilde{E}_x}{\partial y}\right)\label{eq:up_b_z}
\end{displaymath}
\begin{displaymath}
  \frac{\partial}{\partial t}\widetilde{D}_x = \frac{1}{\sqrt{\mu_0\epsilon_0}}\left(\frac{\partial H_z}{\partial y} - \frac{\partial H_y}{\partial z}\right)
\end{displaymath}
\begin{displaymath}
  \frac{\partial}{\partial t}\widetilde{D}_y = \frac{1}{\sqrt{\mu_0\epsilon_0}}\left(\frac{\partial H_x}{\partial z} - \frac{\partial H_z}{\partial x}\right)
\end{displaymath}
Discretize
\begin{displaymath}
  \frac{\widetilde{B}_z|_{i+\frac{1}{2},j+\frac{1}{2}}^{n} - \widetilde{B}_z|_{i+\frac{1}{2},j+\frac{1}{2}}^{n}}{\Delta t} =
  c_0\left(\frac{\widetilde{E}_y|_{i+1,j+\frac{1}{2}}^{} - \widetilde{E}_y|_{i,j+\frac{1}{2}}^{}}{\Delta x} - \frac{\widetilde{E}_x|_{i+\frac{1}{2},j+1}^{} - \widetilde{E}_x|_{i+\frac{1}{2},j}^{}}{\Delta y}\right)
\end{displaymath}
\begin{displaymath}
  \frac{\widetilde{D}_x|_{i+\frac{1}{2},j}^{n+\frac{1}{2}} - \widetilde{D}_x|_{i+\frac{1}{2},j}^{n-\frac{1}{2}}}{\Delta t} =
  c_0\left(\frac{H_z|_{i+\frac{1}{2},j+\frac{1}{2}}^{n} - H_z|_{i+\frac{1}{2},j-\frac{1}{2}}^{n}}{\Delta y} - \right)
\end{displaymath}
\begin{displaymath}
  \frac{\widetilde{D}_y|_{i,j+\frac{1}{2}}^{n+\frac{1}{2}} - \widetilde{D}_x|_{i,j+\frac{1}{2}}^{n-\frac{1}{2}}}{\Delta t} =
  c_0\left( - \frac{H_z|_{i+\frac{1}{2},j+\frac{1}{2}}^{n} - H_z|_{i-\frac{1}{2},j+\frac{1}{2}}^{n}}{\Delta x}\right)
\end{displaymath}

Pseudocode for $\mathrm{TE_z}$ polarization
\begin{code}
Points.each do
end
\end{code}








\section{Incident Source Conditions}
Primarily there are two kinds of sources in the FDTD method: point sources and plane wave sources.

Point source can be classified as hard source or soft source. If the source value is directly assigned to the certain
point, it is referred to as a hard source. Oppositely, it can be called a soft source if the source value is added to
the field at that point. A hard source would lead to some reflection between certain points and adjacent points and a
soft source allows the wave just passes through. Point source is not common used in 2D and 3D simulations but plays an
important role on constructing plane-wave source conditions.

For investigating nano structures, it is often of interest to impinge plane wave source upon structures and measure
transmission and reflection. The calculation of radar cross section also deals with the plane wave. Total Field /
Scatter Field (TFSF) [Taflove and Hagness, 2005] is a technique to simulate a plane wave by dividing the problem space
into the Totol Field region and the Scatter Field region, as shown in Fig 2.3.

The TFSF formulas can be observed through the complete update equations and the totol/scatter field theory as
\begin{displaymath}\label{eq:e_ts}
  E_{total}=E_{scatter}+E_{incident}
\end{displaymath}
\begin{displaymath}\label{eq:h_ts}
  H_{total}=H_{scatter}+H_{incident}  
\end{displaymath}
As an example, consider a the y-incident plane wave of TM polarization in the TFSF region $x=ia:ib, y=ja:jb, z=ka:kb$,
(\ref{eq:dz3d}) performed on the edage $y=ja$ of the main computation domain is actually
\begin{equation}\label{eq:tsedge}
  \begin{split}
    \widetilde{D}_{z,total}|_{i,ja,k} = \widetilde{D}_{z,total}|_{i,ja,k} &+ 0.5 \cdot \left( H_{y,total}|_{i+\frac{1}{2},ja,k+\frac{1}{2}} - H_{y,total}|_{i-\frac{1}{2},ja,k+\frac{1}{2}} \right) \\
    &- 0.5 \cdot \left( H_{x,total}|_{i,ja+\frac{1}{2},k+\frac{1}{2}} - H_{x,scatter}|_{i,ja-\frac{1}{2},k+\frac{1}{2}} \right)    
  \end{split}
\end{equation}
In (\ref{eq:tsedge}), the last $H_x$ belongs to the scattered field outside the TFSF boundary. By applying
(\ref{eq:h_ts}) to complete the curl operation, the insertion of the TFSF source of $\widetilde{D}_z$ is described by
\begin{displaymath}
  \widetilde{D}_z|_{i,ja,k} = \widetilde{D}_z|_{i,ja,k} + 0.5 \cdot H_{inc}|_{ja-\frac{1}{2}}
\end{displaymath}
The rest equations for the insertion of the TFSF can also be dervied as
\begin{displaymath}
  \widetilde{D}_z|_{i,jb,k} = \widetilde{D}_z|_{i,jb,k} - 0.5 \cdot H_{inc}|_{jb+\frac{1}{2}}  
\end{displaymath}
\begin{displaymath}
  \widetilde{B}_x|_{i,ja-\frac{1}{2},k+\frac{1}{2}}=\widetilde{B}_x|_{i,ja-\frac{1}{2},k+\frac{1}{2}}+0.5 \cdot E_{inc}|_{ja}
\end{displaymath}
\begin{displaymath}
  \widetilde{B}_x|_{i,jb+\frac{1}{2},k+\frac{1}{2}}=\widetilde{B}_x|_{i,jb+\frac{1}{2},k+\frac{1}{2}}-0.5 \cdot E_{inc}|_{jb}
\end{displaymath}
\begin{displaymath}
  \widetilde{B}_y|_{ia-\frac{1}{2},ja:jb,k+\frac{1}{2}}=\widetilde{B}_y|_{ia-\frac{1}{2},ja:jb,k+\frac{1}{2}}-0.5 \cdot E_{inc}|_{ja:jb}
\end{displaymath}
\begin{displaymath}
  \widetilde{B}_y|_{ib+\frac{1}{2},ja:jb,k+\frac{1}{2}}=\widetilde{B}_y|_{ib+\frac{1}{2},ja:jb,k+\frac{1}{2}}+0.5 \cdot E_{inc}|_{ja:jb}
\end{displaymath}







\section{Boundary Conditions}

\subsection{Absorbing Boundary Conditions}

The Absorbing Boundary Conditions were proposed

\subsection{Perfectly Matched Layer}

For 2-D and 3-D simulation, the ABC discussed above is no longer useful due to the reason the ABC can only absorb
normally incident wave.

\subsubsection{Berenger's PML}
the implementation of Berenger's PML

aka Split-Field PML
\subsubsection{Unaxial PML}
the implementations of Unaxial PML
\subsubsection{Convolution PML}
the implementations of Convolution PML

\subsection{Periodic Boundary Conditions}

For simulating Bloch periodic structure such as Photonic Crystals, the periodic boundary conditions (PBCs) were useful
to reduce the simulation region

\section{Near-to-Far-Field Transformation}

\section{Dispersive Material}
\label{sec:dispersive}
This section conerns of how to retrieve the update equations could apply in dispersive material. What would be discussed
below are only about material having dispersive electic permittivity. The material having dispersive magnetic
permeability could be investigated via duality.

In fundamental Electromagnetics, an indispersive material having a electrical polarization $P$ when there is a foreign
electric field and the phasor of electric flux would become $D(\omega) = \epsilon_0 E(\omega) + P$. $P$ was also defined
as $\epsilon_0 \chi_e E(\omega)$, so that we could rewrite $D(\omega)$ as $\epsilon_0 (1+\chi_e)E(\omega)$ and give the middle term a notation
$\epsilon_r$ named relative permittivity.

By definition, dispersive material is the material having different electic permeability when encountering EM wave
having different frequency that means the relative permittivity should be a function of frequency ordinarily coming with
image part. So, $\epsilon_r^*(\omega)$ was given as the notation of dispersive electic permittivity and the phasor of
electic flux in dispersive material was defined as $D(\omega) = \epsilon_0 \epsilon_r^*(\omega)E(\omega)$ or
$\widetilde{D} = \epsilon_r^*(\omega)\widetilde{E}$ in Gaussian Unit.
\subsection{Common Isotropic Dispersive Material}
The dispersive permittivity is defined as $\epsilon_r^*(\omega) = 1 + \chi_e + \sum \chi_p(\omega)$ and usually, $1 +
\chi_e$ are written as $\epsilon_r$ to be identical with fundamental electromagnetics.
\subsubsection{Simple Lossy Media}
Simple Lossy Media is a material with constant electric conductivity $\sigma_e$.
\begin{gather}
  \chi_p(\omega) = \frac{\sigma_e}{j\omega\epsilon_0}\\
  \epsilon_r^*(\omega) = \epsilon_r + \frac{\sigma_e}{j\omega\epsilon_0}
\end{gather}

\subsubsection{Debye Media}
\begin{equation}
  \label{eq:debye_chi}
  \chi_p(\omega) = \frac{\Delta\epsilon_p}{1+j\omega\tau_p}  
\end{equation}
\begin{equation}
  \epsilon_r^*(\omega) = \epsilon_r + \sum_{p=1}^P \frac{\Delta\epsilon_p}{1+j\omega\tau_p}  
\end{equation}


\subsubsection{Drude Media}
\begin{equation}
  \label{eq:drude_chi}
  \chi_p(\omega) = -\frac{\omega_p^2}{\omega^2 - j\omega\gamma_p}  
\end{equation}
\begin{equation}
  \epsilon_r^*(\omega) = \epsilon_r - \sum_{p=1}^P \frac{\omega_p^2}{\omega^2-j\omega\gamma_p}
\end{equation}
where $\omega_p$ is the Drude pole frequency and $\gamma_p$ is the inverse of the pole relaxation time, also known as the
electron oscillision frequency.
\begin{equation}
  m_e\frac{\partial^2 x(t)}{\partial t^2} + m_e\gamma_p\frac{\partial x(t)}{\partial t} = -eE(t)
\end{equation}
$x(t) = \mathrm{Re}\{x_0e^{j\omega t}\}$ and $E(t) = \mathrm{Re}\{E_0e^{j\omega t}\}$
\begin{equation}
  x(t) = \frac{e}{m_e(\omega^2 - j\omega\gamma_p)}E(t)
\end{equation}
Because polarization $P = -Ne\cdot x(t)$ should also satisfy $P = \epsilon_0\chi_pE(t)$, $\chi_p$ can be solved
\begin{equation}
  \chi_p = \frac{-Ne^2/m_e\epsilon_0}{\omega^2 - j\omega\gamma_p}
\end{equation}
With denoting $-Ne^2/m_e\epsilon_0$ by $\omega_p^2$. it becomes Eq.\ref{eq:drude_chi}


\subsubsection{Lorentz Media}
\begin{equation}
  \label{eq:lorentz_chi}
  \chi_p(\omega) = \frac{\Delta\epsilon_p\omega_p^2}{\omega_p^2 + 2j\omega\delta_p - \omega^2}  
\end{equation}
\begin{equation}
  \epsilon_r^*(\omega) = \epsilon_r + \sum_{p=1}^P \frac{\Delta\epsilon_p\omega_p^2}{\omega_p^2 + 2j\omega\delta_p - \omega^2}  
\end{equation}
start from
\begin{equation}
    m_e\frac{\partial^2 x(t)}{\partial t^2} + 2m_e\delta_L\frac{\partial x(t)}{t} + m_e\omega_L^2x(t) = -eE(t)
\end{equation}







\subsection{Dispersion-Compatible Update Equations}
For the analysis of dispersive materials shown above, numbers of algorithms have already been proposed in literature.
Most of these frequency-dependent algorithms can be categorized into three types: 
\begin{inparaenum}[(1)]
\item the Recursive Convolution (RC) method
\item the Auxiliary Differential Equation (ADE) method
\item the Z-transform (ZT) method
\end{inparaenum}.

The RC method is the most basic method inheriting the convolution theorem in Laplace transform and Fourier
transform. The pros and cons are easy to understand with background of engineering mathematics and difficult to handle
complex material with multiple poles.

The ADE method offer high flexibility in fitting arbitrary permittivity functions, modeling nonlinear effects and
arbitrary numbers of poles.

Any of these methods focuses on tranforming the dispersive relations between $D(\omega))$ and $E(\omega)$ in frequency domain
back to the time doamin for discretization. By applying different discretizing scheme at each step, many varieties of
them were given. Following words are trying to give an overview and examples as many as possible to be a reference
during implementing.

\subsubsection{The Recursive Convolution Method}
There are many varieties published using convolution including 
\begin{inparaenum}[(1)]
\item Recursive Convolution (RC) Method
\item Piecewise-Linear Recursive Convolution (PLRC) Method
\item Trapezoidal Recursive Convolution(TRC) Method
\end{inparaenum}

\paragraph{\msjh Simple Conductive Media - RC}
Apply RC to simple conductive material having definition as
\begin{displaymath}
  \epsilon_r^{*}(\omega) = \epsilon_r + \frac{\sigma}{j \omega \epsilon_0}
\end{displaymath}
\begin{displaymath}
  \begin{split}
    \widetilde{D}(\omega) & = \epsilon_r^{*}(\omega)\widetilde{E}(\omega)\\
    & = \epsilon_r\widetilde{E}(\omega) + \frac{\sigma}{j\omega\epsilon_0}\widetilde{E}(\omega)
  \end{split}
\end{displaymath}
\begin{displaymath}
  D(t) = \epsilon_r\widetilde{E}(t) + \frac{\sigma}{\epsilon_0}\int_0^t\widetilde{E}(t')dt'
\end{displaymath}
\begin{displaymath}
    D^n = \epsilon_r\widetilde{E}^n + \frac{\sigma}{\epsilon_0}\Delta t \sum_{i=0}^{n}\widetilde{E}^i
\end{displaymath}
The second term: dissipated displacement -> $I^n$, in recursive form
\begin{equation}
  I^n = \frac{\sigma}{\epsilon_0}\Delta t\cdot\widetilde{E}^n + \frac{\sigma}{\epsilon_0}\Delta t\sum_{i=0}^{n-1}\widetilde{E}^i = \frac{\sigma}{\epsilon_0}\Delta t\cdot\widetilde{E}^n + I^{n-1}
\end{equation}
\begin{equation}
  \widetilde{D}^n = \epsilon_r\widetilde{E}^n + I^n = (\epsilon_r + \frac{\sigma}{\epsilon_0}\Delta t)\widetilde{E}^n + I^{n-1}
\end{equation}
Finally, the update equations becomes
\begin{gather*}
  \widetilde{E}^n = \frac{\widetilde{D}^n - I^{n-1}}{\displaystyle \epsilon_r +\frac{\sigma}{\epsilon_0}\Delta t}\\
  I^n = I^{n-1} + \frac{\sigma}{\epsilon_0}\Delta t\cdot E^n
\end{gather*}
implementation
\begin{code}
  dx[k] = dx[k]
  ex[k] = ex[k]
  i[k]  = i[k]
\end{code}


\paragraph{{\msjh Debye Model - RC}}
\begin{displaymath}
  \epsilon_r^*(\omega) = \epsilon_r + \frac{\sigma}{j\omega \epsilon_0} + \frac{\Delta \epsilon_p}{1+j\omega \tau_p}
\end{displaymath}
\begin{displaymath}
  \begin{split}
    \widetilde{D}(\omega) & = \epsilon_r^*(\omega)\widetilde{E}(\omega)\\
    & = \epsilon_r\widetilde{E}(\omega) + \frac{\sigma}{j\omega\epsilon_0}\widetilde{E}(\omega) + \frac{\Delta \epsilon_p}{1+j\omega \tau_p}\widetilde{E}(\omega)
  \end{split}
\end{displaymath}
\begin{displaymath}
  \widetilde{D}(t) = \epsilon_r\widetilde{E}(t) + \frac{\sigma}{\epsilon_0}\int_0^t\widetilde{E}(t')dt' + \int_0^t\frac{\Delta \epsilon_p}{\tau_p}e^{-\frac{t'-t}{\tau_p}}\widetilde{E}(t')dt'
\end{displaymath}
\begin{equation}
  \widetilde{D}^n = \epsilon_r\widetilde{E}^n + \frac{\sigma}{\epsilon_0}\Delta t\sum_{i=0}^{n}\widetilde{E}^i + \frac{\Delta \epsilon_p}{\tau_p}\Delta t \sum_{i=0}^{n} e^{-\frac{n-i}{\tau_p}\Delta t}\widetilde{E}^i
\end{equation}
The second term, dissipated displacement, $I^n$ , the same as simple conductive media.
The third term, phasor polarization displacement, $J_p^n$. $J_p^n$ is slightly complex in recursive form.
\begin{equation}
  \begin{split}
    J_p^n & = \frac{\Delta\epsilon_p}{\tau_p}\Delta t \sum_{i=0}^ne^{-\frac{n-i}{\tau_p}\Delta t}\widetilde{E}^i\\
    & = \frac{\Delta\epsilon_p}{\tau_p}\Delta t \left(\widetilde{E}^n + \sum_{i=0}^{n-1}e^{-\frac{n-i}{\tau_p}\Delta t}\widetilde{E}^i\right)
  \end{split}
\end{equation}
and 
\begin{equation}
  \begin{split}
    J_p^{n-1} & = \frac{\Delta\epsilon_p}{\tau_p}\Delta t \sum_{i=0}^{n-1}e^{-\frac{n-1-i}{\tau_p}\Delta t}\widetilde{E}^i\\
    & = \frac{\Delta\epsilon_p}{\tau_p}\Delta t \left( e^{\frac{\Delta t}{\tau_p}} \right) \sum_{i=0}^{n-1}e^{-\frac{n-i}{\tau_p}\Delta t}\widetilde{E}^i
  \end{split}
\end{equation}
substituted into $J_p^n$
\begin{equation}
  J_p^n = \frac{\Delta\epsilon_p}{\tau_p}\Delta t\cdot\widetilde{E}^n + e^{-\frac{\Delta t}{\tau_p}} J_p^{n-1}
\end{equation}
\begin{equation}
  \begin{split}
    \widetilde{D}^n & = \epsilon_r\widetilde{E}^n + \left[\frac{\sigma}{\epsilon_0}\Delta t\cdot\widetilde{E}^n + I^{n-1}\right] + \left[\frac{\Delta \epsilon_p}{\tau_p}\Delta t\cdot\widetilde{E}^n + e^{-\frac{\Delta t}{\tau_p}} J_p^{n-1}\right]\\
    & = \left(\epsilon_r + \frac{\sigma}{\epsilon_0}\Delta t + \frac{\Delta\epsilon_p}{\tau_p}\Delta t\right)\widetilde{E}^n + I^{n-1} + e^{-\frac{\Delta t}{\tau_p}} J_p^{n-1}
  \end{split}
\end{equation}
Finally the discrete constitute relations in time domain becomes
\begin{gather}
  \begin{array}{@{}l@{}}
    \widetilde{E}^n =  \frac{\displaystyle \widetilde{D}^n - I^{n-1} - e^{-\frac{\Delta t}{\tau_p}}J_p^{n-1} }{\displaystyle \epsilon_r + \frac{\sigma}{\epsilon_0}\Delta t + \frac{\Delta \epsilon_p}{\tau_p} \Delta t}\\    
    I^n = \frac{\sigma}{\epsilon_0}\Delta t\cdot\widetilde{E}^n + I^{n-1}\\
    J_p^n = \frac{\Delta\epsilon_p}{\tau_p}\Delta t\cdot\widetilde{E}^n + e^{-\frac{\Delta t}{\tau_p}} J_p^{n-1}
  \end{array}
\end{gather}
implement
\begin{code}
  dx[k] = dx[k] + 0.5 * ( hy[k-1] - hy[k])
  ex[k] = ( dx[k] - i[k] - exp(-dt/tau_p) * j[k] ) 
  * ( epsilon_0 * epsilon_r[k] + sigma_e[k] * dt +  )
  i[k] = i[k] + sigma_e[k] 
  j[k] = j[k] + del
\end{code}






\paragraph{{\msjh Lorentz Model - RC}}


\paragraph{{\msjh Drude Model - RC}}
\begin{equation}
  D(\omega) = \epsilon_0\epsilon_r^*(\omega)E(\omega) = \epsilon_0\epsilon_rE(\omega) - \epsilon_0\frac{\omega_p^2}{\omega^2-j\omega\gamma_p}E(\omega)
\end{equation}
\begin{equation}
  D(t) = \epsilon_0\epsilon_rE(t) + \epsilon_0\int_0^t -\frac{\omega_p^2}{\gamma_p}(1 - e^{-\gamma_p(t'-t)})E(t')dt'
\end{equation}
\begin{equation}
  D^n = \epsilon_0\epsilon_rE^n + \epsilon_0\frac{\omega_p^2}{\gamma_p}\Delta t\sum_{i=0}^{n}E^i + \epsilon_0\frac{\omega_p^2}{\gamma_p}\Delta t \sum_{i=0}^{n}e^{-\gamma_p(n-i)\Delta t} E^i
\end{equation}



\subsubsection{The Auxiliary Differential Equation Method}
The central idea of the Auxiliary Differential Equation (ADE) Method is to detach the phasor polarization displacement
$J_p(\omega)$, that is, $\chi_p(\omega)E(\omega)$, from original dispersive relation.
\begin{equation}
  \begin{split}
    \widetilde{D}(\omega) &= \epsilon_r^*(\omega)\widetilde{E}(\omega)\\
    & = \epsilon_r\widetilde{E}(\omega) + \sum_{p=1}^P\chi_p(\omega)\widetilde{E}(\omega)\\
    & = \epsilon_r\widetilde{E}(\omega) + \sum_{p=1}^{P}J_p(\omega)
  \end{split}
\end{equation}
For muiltipole material composed of different dispersions, $J_p$ were solved for each pole individually to attend the
updating loop, and then $E$ can be solved via rearrangment of previous relation.
\begin{equation}
  \widetilde{E}^n = \frac{1}{\epsilon_r}\left(\widetilde{D}^n - \sum_{p=1}^PJ_p^n\right)
\end{equation}
Primitive ADE method shown in literature require deriving formulations for each dispersion type, however, a generalized
way \textit{Alsunaidi et al.} proposed finds its strength in unifying the formulation of different dispersion models
into one form.

Starting with the most general form of $\chi_p(\omega)$, the phasor polarization displacement can be written
as
\begin{equation}
  J_p(\omega) = \frac{a}{b + j\omega c - d\omega^2}\widetilde{E}(\omega)
\end{equation}
Rearranging and performming inverse Fourier transform 
\begin{equation}
  bJ_p(t) + c \frac{\partial}{\partial t}J_p(t) + d \frac{\partial ^2}{\partial t^2}J_p(t) = a\widetilde{E}(t)
\end{equation}
apply leapfrog scheme 
\begin{equation}
  bJ_p^{n-1} + c \frac{J_p^n - J_p^{n-1}}{2\Delta t} + d \frac{J_p^n + 2J_p^{n-1} + J_p^{n-2}}{(\Delta t)^2} = a\widetilde{E}^{n-1}
\end{equation}
Solving 
\begin{equation}
  J_p^n = \frac{4d-2b(\Delta t)^2}{2d+c\Delta t}J_p^{n-1} + \frac{-2d+c\Delta t}{2d+c\Delta t}J_p^{n-2} + \frac{2a(\Delta t)^2}{2d+c\Delta t}\widetilde{E}^{n-1}
\end{equation}
which can be written in the form 
\begin{equation}
  J_p^n = C_1 J_p^{n-1} + C_2 J_p^{n-2} + C_3 \widetilde{E}^{n-1}
\end{equation}
where $C_1$, $C_2$ and $C_3$ can be found for any form of dispersion relation.



\paragraph{{\msjh Debye Model - ADE}}
Debye model is a fun case can be induced the same form of $J_p(t)$ as RC if the semi-implicit scheme was chose. However,
leapfrog scheme provide better accruacy. Just for verifying, result of apply semi-implicit scheme was also derived
here. In the real wrold implementation \textit{\uwave{yaFDTD}}, leapfrog was picked out.

Starting with constitute relation as before
\begin{equation}
  \begin{split}
    \widetilde{D}(\omega) & = \epsilon_r^*(\omega)\widetilde{E}(\omega)\\
    & = \epsilon_r\widetilde{E}(\omega) + \frac{\sigma}{j\omega\epsilon_0}\widetilde{E}(\omega) + \frac{\Delta \epsilon_p}{1+j\omega \tau_p}\widetilde{E}(\omega)\label{eq:debye_ade_start}
  \end{split}
\end{equation}
detach $J_p$
\begin{equation}
  J_p(\omega) = \frac{\Delta \epsilon_p}{1+j\omega \tau_p}\widetilde{E}(\omega)
\end{equation}
This is the ADE of Debye Model
\begin{equation}
  J_p(\omega) + j\omega\tau_{p}J_p(\omega) = \Delta\epsilon_p\widetilde{E}(\omega)
\end{equation}
performming IFT 
\begin{equation}
  J_p(t) + \tau_p\frac{\partial}{\partial t}J_p(t) = \Delta\epsilon_p\widetilde{E}(t)
\end{equation}
apply semi-implicit scheme
\begin{equation}
  \left( \frac{J_p^n - J_p^{n-1}}{2} \right) + \tau_p \left( \frac{J_p^n - J_p^{n-1}}{\Delta t}\right) = \Delta\epsilon_p\widetilde{E}^n
\end{equation}
Solving $J_p$
\begin{equation}
  J_p^n = \frac{\left(1-\frac{\Delta t}{2\tau_p}\right)}{\left(1+\frac{\Delta t}{2\tau_p}\right)}J_p^{n-1} 
  + \frac{\left(\frac{\Delta\epsilon_p}{\tau_p}\right)\Delta t}{\left(1+\frac{\Delta t}{2\tau_p}\right)}\widetilde{E}^n
\end{equation}
It should be noted
\begin{equation}
  \begin{array}{@{}lp{0.5cm}r@{}}
    \frac{\displaystyle1-\delta}{\displaystyle1+\delta} \cong e^{-2\delta} && if\ \delta \ll 1
  \end{array}
\end{equation}
\begin{equation}
  \frac{\left(1-\frac{\displaystyle\Delta t}{\displaystyle2\tau_p}\right)}{\left(1+\frac{\displaystyle\Delta t}{\displaystyle2\tau_p}\right)} \cong e^{-\frac{\Delta t}{\tau_p}}\quad  because\ 1 \gg \frac{\Delta t}{2\tau_p}
\end{equation}
That is 
\begin{equation}
  J_p^n \cong e^{-\frac{\Delta t}{\tau_p}}J^{n-1} + \frac{\Delta\epsilon_p}{\tau_p}\Delta t\cdot\widetilde{E}^n
\end{equation}
And by performming inverse Fourier transform on Eq.\ref{eq:debye_ade_start}
\begin{equation}
  \widetilde{D}(t) = \epsilon_r\widetilde{E}(t) + \frac{\sigma}{\epsilon_0} \int_0^t\widetilde{E}(t')dt' + J_p(t)
\end{equation}
\begin{equation}
  \begin{split}
    \widetilde{D}^n & = \epsilon_r\widetilde{E}^n + \frac{\sigma}{\epsilon_0}\Delta t\sum_{i=0}^n\widetilde{E}^i + J_p^n\\
    & = \epsilon_r\widetilde{E}^n + \frac{\sigma}{\epsilon_0}\Delta t\cdot\widetilde{E}^n + I^{n-1} + \frac{\Delta\epsilon_p}{\tau_p}\Delta t\cdot\widetilde{E}^n + e^{-\frac{\Delta t}{\tau_p}}J^{n-1}
  \end{split}
\end{equation}
The same result as Recursive Convolution Method.






\paragraph{\msjh Lorentz Model - ADE} Lorentz pole totally matches the general form used in previous introduction of ADE method.
The coefficients are as following.
\begin{equation*}
  \begin{array}{@{}llll@{}}
    a = \Delta\epsilon_p\omega_p^2 &
    b = \omega_p^2 &
    c = 2\delta_p &
    d = 1
  \end{array}
\end{equation*}
substitute into ...
\begin{gather*}
  \begin{array}{@{}lll@{}}
    C_1 = \frac{\displaystyle 2-\omega_p^2\Delta t^2}{\displaystyle 1+\delta_p\Delta t} &
    C_2 = \frac{\displaystyle -1 + \delta_p\Delta t}{\displaystyle 1+\delta_p\Delta t} &
    C_3 = \frac{\displaystyle \Delta\epsilon_p\omega_p^2\Delta t^2}{\displaystyle 1+\delta_p\Delta t}
  \end{array}
\end{gather*}
Formal derivation is also written down here.
\begin{equation}
  \begin{split}
    \widetilde{D}(\omega) & = \epsilon_r^*(\omega)\widetilde{E}(\omega)\\
    & = \epsilon_r\widetilde{E}(\omega) +  \frac{\Delta \epsilon_p \omega_p^2}{\omega_p^2+2j\omega\delta_p-\omega^2}\widetilde{E}(\omega)
  \end{split}
\end{equation}
\begin{equation}
  J_p(\omega) =  \frac{\Delta \epsilon_p \omega_p^2}{\omega_p^2+2j\omega\delta_p-\omega^2}\widetilde{E}(\omega)
\end{equation}
rearrange and IFT
\begin{equation}
  \omega_p^2J_p(t) + 2\delta_p\frac{\partial}{\partial t}J_p(t) + \frac{\partial^2}{\partial t^2}J_p(t) = \Delta\epsilon_p\omega_p^2\widetilde{E}(t)
\end{equation}
apply leapfrog scheme 
\begin{equation}
  \omega_p^2J_p^{n-1} + \delta_p\frac{J_p^n - J_p^{n-2}}{\Delta t} + \frac{J_p^n - 2 J_p^{n-1} + J_p^{n-2}}{\Delta t^2} = \Delta\epsilon_p\omega_p^2\widetilde{E}^{n-1}
\end{equation}
rearrange 
\begin{equation}
  J_p^n = 
  \frac{ 2-\omega_p^2\Delta t^2}{ 1+\delta_p\Delta t} J_p^{n-1} +
  \frac{ -1 + \delta_p\Delta t}{ 1+\delta_p\Delta t} J_p^{n-2} + 
  \frac{ \Delta\epsilon_p\omega_p^2\Delta t^2}{ 1+\delta_p\Delta t}\widetilde{E}^{n-1}
\end{equation}
as the result showed by general form of $C_1$, $C_2$, $C_3$.



\paragraph{\msjh Drude Model - ADE} Drude pole lacks the b coefficient when comparing to general form.
\begin{equation}
  \begin{split}
    \widetilde{D}(\omega) & = \epsilon_r^*(\omega)\widetilde{E}(\omega)\\
    & =  \epsilon_r\widetilde{E}(\omega) - \frac{\omega_p^2}{\omega(\omega-j\gamma_p)}\widetilde{E}(\omega)
  \end{split}
\end{equation}
Defining the last term as $J_p$, the phasor polarization current in physics,
\begin{equation}
  J_p(\omega) = -\frac{\omega_p^2}{\omega(\omega-j\gamma_p)}\widetilde{E}(\omega)
\end{equation}
Rearrange
\begin{equation}
  j\omega\gamma_pJ_p(\omega) - \omega^2J_p(\omega) = \omega_p^2\widetilde{E}(\omega)
\end{equation}
Performming inverse Fourier transformation
\begin{equation}
  \frac{\partial^2 J_p(t)}{\partial t^2} + \gamma_p \frac{\partial J_p(t)}{\partial t} = \omega_p^2\frac{\partial\widetilde{E}(t)}{\partial t}
\end{equation}
This is the ADE for $J_p$ for Drude Model. Then apply leapfrog scheme as general solution.
\begin{equation}
  \gamma_p\frac{J_p^n-J_p^{n-2}}{2\Delta t} + \frac{J_p^n - 2J_p^{n-1} + J_p^{n-2}}{(\Delta t)^2} = \omega_p^2\widetilde{E}^{n-1}
\end{equation}
rearrange 
\begin{equation}
  J_p^n = \frac{4}{2+ \gamma_p\Delta t} J_p^{n-1} + \frac{-2+\gamma_p\Delta t}{2+\gamma_p\Delta t}J_p^{n-2} + \frac{2\omega_p^2(\Delta t)^2}{2+\gamma_p\Delta t}\widetilde{E}^{n-1}
\end{equation}



All coefficients of different dispersion types are summarized here
\begin{center}
  \begin{tabular}[c]{|r|l|c|c|c|}
    \hline
    Dispersion Type & $\chi_p(\omega)$ & $C_1$ & $C_2$ & $C_3$ \\
    \hline
    Lorentz & $\frac{\Delta \epsilon_p \omega_p^2}{\sqrt{\omega_p^2 - \delta_p^2}}e^{-\delta_p t}\sin\left(\sqrt{\omega_p^2-\delta_p^2}\ t\right)$ & $\frac{2-\omega_p^2\Delta t^2}{1+\delta_p\Delta t}$ & $\frac{-1 + \delta_p\Delta t}{1+\delta_p\Delta t}$  & $\frac{\Delta\epsilon_p\omega_p^2\Delta t^2}{1+\delta_p\Delta t}$ \\
    \hline
    Drude & $\frac{\epsilon_0\omega_p^2}{j\omega\gamma_p-\omega^2}$ & $\frac{ 4}{ 2+\gamma_p\Delta t}$ & $\frac{ -2+\gamma_p\Delta t}{ 2+\gamma_p\Delta t}$ & $\frac{ 2\omega_p^2\Delta t^2}{ 2+\gamma_p \Delta t}$\\
    \hline
  \end{tabular}
\end{center}




\subsubsection{The Z Transform Method}
\paragraph{\msjh Debye Model - ZT}

\paragraph{\msjh Lorentz Model - ZT}

\paragraph{\msjh Drude Model - ZT}

\section{Transmittance Spectrum}
\subsection{Fourier Transform}
The implement of Discrete Fourier Transform\\
The implement of Fast Fourier Transform

\section{The yaFDTD Framework}

In prior sections, all common components of FDTD were depicted. All the formula has been implemented in the Python programming
language to be a software framework named yet another Finite-Difference Time-Domain framework, The repository of this
framework is at \tt http://github.com/shelling/yafdtd \rm. A well-designed framework can reduce the time to developing and
modeling a new simulation in the future so that it's worthy to spend one section introducing the usages of the
framework.


\chapter{Modeling of Curved Surface}

\chapter{Modeling of Plasmonic Waveguides}

\chapter{Conclusion}

\end{document}

\documentclass[openany]{book}

\usepackage{fontspec}
\usepackage{paralist}
\usepackage{wallpaper}
\usepackage{color}
\usepackage{xunicode}
\usepackage{xltxtra}
\usepackage{graphicx}
\usepackage[margin=3cm]{geometry}
\usepackage{type1cm}
\usepackage{parskip}
\usepackage[fleqn]{amsmath}
\usepackage{amsthm}
\usepackage{url}
\usepackage{makeidx}
\usepackage[normalem]{ulem}
\usepackage{fancyvrb}
\usepackage{wasysym}


\setmonofont[SizeFeatures={Size={9}}]{Monaco}
\newfontfamily\msjh{Microsoft JhengHei}
\newfontfamily\kai{BiauKai}
\newfontfamily\arial{Arial}

\XeTeXlinebreaklocale "zh"
\DefineVerbatimEnvironment{code}{Verbatim}{fontsize=\small,frame=leftline,numbers=left,commandchars=\\\{\}}
\setlength{\parskip}{1em plus0.2em minus0em}
\pagestyle{plain}
\makeindex

\newcommand{\overPartialT}{\frac{\partial}{\partial t}}

\newcommand{\curlH}{\nabla\times H}
\newcommand{\curlE}{\nabla\times \widetilde{E}}

\newcommand{\curlHxThree}{\left( \frac{\partial H_z}{\partial y} - \frac{\partial H_y}{\partial z} \right)}
\newcommand{\curlHyThree}{\left( \frac{\partial H_x}{\partial z} - \frac{\partial H_z}{\partial x} \right)}
\newcommand{\curlHzThree}{\left( \frac{\partial H_y}{\partial x} - \frac{\partial H_x}{\partial y} \right)}

\newcommand{\curlExThree}{\left(\frac{\partial \widetilde{E}_z}{\partial y} - \frac{\partial \widetilde{E}_y}{\partial z}\right)}
\newcommand{\curlEyThree}{\left(\frac{\partial \widetilde{E}_x}{\partial z} - \frac{\partial \widetilde{E}_z}{\partial x}\right)}
\newcommand{\curlEzThree}{\left(\frac{\partial \widetilde{E}_y}{\partial x} - \frac{\partial \widetilde{E}_x}{\partial y}\right)}

\newcommand{\curlHxTwo}{\left(  \frac{\partial H_z}{\partial y}\right)}
\newcommand{\curlHyTwo}{\left(- \frac{\partial H_z}{\partial x}\right)}
\newcommand{\curlHzTwo}{\left(  \frac{\partial H_y}{\partial x} - \frac{\partial H_x}{\partial y}\right)}

\newcommand{\curlExTwo}{\left(  \frac{\partial \widetilde{E}_z}{\partial y}\right)}
\newcommand{\curlEyTwo}{\left(- \frac{\partial \widetilde{E}_z}{\partial x}\right)}
\newcommand{\curlEzTwo}{\left(  \frac{\partial \widetilde{E}_y}{\partial x} - \frac{\partial \widetilde{E}_x}{\partial y}\right)}

\begin{document}
\fontsize{12}{2.0em}\selectfont


\frontmatter

\title{
  \kai 國立台灣大學電機資訊學院光電工程學研究所\\碩士論文\\
  \rm Graduate Institute of Photonics and Optoelectronics\\College of Electrical Engineering and Computer Science\\National Taiwan University\\Master Dissertation\\[1cm]
  \kai 有限時域差分法之軟體架構與應用\\
  \rm Modern Software Architecture of the Finite-Difference Time-Domain Numerical Model and Its Applications
}

\author{
  \kai 許家瑋\\ \rm Jia-Wei Hsu\\\\
  \kai 指導教授:張宏鈞 博士\\ \rm Advisor: Hung-Chun Chang, Ph.D.
}

\date{
  \kai 中華民國 100 年 7 月\\ 
  \rm July 2011
}

\CenterWallPaper{0.5}{"ntu.jpg"}
\maketitle

\cleardoublepage
\chapter{Acknownledgement}
Thanks to ... $\heartsuit$

\cleardoublepage
\chapter{\kai 致謝}
{\kai
謝謝爸媽在求學這段不短的日子中無盡的支持,沒有你們的愛,我沒有辦法一路走來。
謝謝張宏鈞老師給予的指導與包容,老師嚴謹的作學態度,詼諧風趣的談吐,都是我還有待學習之處。
謝謝這兩年中其他所有給予我協助的人,點滴都在心頭,在此羅列您們的大名,似乎也太過俗套,因此請容我套句陳之藩大師的話:要謝的人太多了,那就謝天吧。\blacksmiley

兩年的時間是如此倉促,在進入實驗室不久所訂定的目標:嘗試完成一套有限差分時域法的軟體框架,又是如此浩大的工程,以一人之力實在難以竟功。
雖然遺憾未能畢其於一役,在即將畢業的此刻,也祇好將之留待茶餘飯後續以狗尾。套句改編自陳之藩大師的話:要改的程式碼太多了,祇好改天吧。\smiley

\begin{flushright}
中華民國一百年七月二十二日午夜\\於實驗室
\end{flushright}
}

\cleardoublepage
\chapter{Abstract}
The thesis is splited into two part. The first part is aimed at being a quick reference depicting the formulas of common
FDTD components as well as the base of my own real world implementation, yaFDTD. The second part discusses some surface
plasmonic structures via the result of its simulations.
\cleardoublepage
\chapter{\kai 摘要}
{
\kai

兄弟們到這樣時候,和狺狺的狗吠,也不知立的四周是否危險,現在不高興了,昨晚曾賜過觀覽,能夠合官廳的意思,不過隨意做作而
已,富豪是先天所賦與,了。拭過似的、萬里澄碧的天空,坐著閒談。好,什麼樣子,趕快走下山去!人們怎地在心境上,現在只有覺
悟,接著又說,誰甘白受人家的欺負,又恍惚坐在卸帆的舟中,憑這雙腕!也不是什麼大不了的事,我們婦女竟是消遣品,過年,實在想
不到!又復濃濃密密屯集起來,既不能把它倆,把他倆埋沒在可怕的黑暗之下。兄弟們到這樣時候,和狺狺的狗吠,也不知立的四周是否
危險,現在不高興了,昨晚曾賜過觀覽,能夠合官廳的意思,不過隨意做作而已,富豪是先天所賦與,了。拭過似的、萬里澄碧的天空,
坐著閒談。好,什麼樣子,趕快走下山去!人們怎地在心境上,現在只有覺悟,接著又說,誰甘白受人家的欺負,又恍惚坐在卸帆的舟
中,憑這雙腕!也不是什麼大不了的事,我們婦女竟是消遣品,過年,實在想不到!又復濃濃密密屯集起來,既不能把它倆,把他倆埋沒
在可怕的黑暗之下。

}

\tableofcontents





\mainmatter

\chapter{Introduction}
\section{Motivations}
More Lorem ipsum dolor sit amet, consectetur adipiscing elit. Phasellus nec ligula a tortor mattis
consectetur. Phasellus eget dictum quam. Pellentesque cursus, lacus ut rutrum suscipit, dolor sapien varius nisl, sed
aliquam mi augue non magna. Maecenas dignissim aliquet porta. Sed elit purus, vestibulum a posuere ut, volutpat eget
nibh. Phasellus eu dolor ante, a tincidunt massa. Pellentesque porttitor pharetra risus. Sed a sem neque. Etiam varius
rutrum consequat. Donec sagittis nulla sed lectus tristique iaculis. Aliquam placerat sagittis enim nec
aliquam. Phasellus in erat metus. Cum sociis natoque penatibus et magnis dis parturient montes, nascetur ridiculus mus.

Nullam nisl erat, pulvinar a fermentum ut, tempus eget enim. Vivamus vel odio id urna ultricies sollicitudin. Curabitur
lobortis augue rhoncus purus sollicitudin ornare. Nullam sit amet quam quis neque mollis elementum. Cras dapibus felis
eu mi vulputate ut rutrum dui molestie. Ut at venenatis purus. Nunc eget lacus blandit enim pharetra congue posuere in
arcu. Duis pharetra, mi vitae venenatis pellentesque, quam nulla vulputate dui, at facilisis tortor tortor non orci. Sed
lectus erat, suscipit quis aliquam et, lobortis at lorem. Quisque feugiat neque eros. Nunc rutrum adipiscing dolor eu
pulvinar. Nunc elit diam, tincidunt at vulputate ornare, eleifend id ante. Aliquam a augue augue, a hendrerit eros.

Praesent lectus enim, tincidunt id volutpat sit amet, varius sit amet lectus. Nam bibendum consequat tellus accumsan
pellentesque. Nullam enim dolor, eleifend eu faucibus non, convallis congue nibh. Pellentesque vel urna ac lorem congue
vehicula non ut massa. Quisque a metus et est faucibus sollicitudin viverra ut purus. Nullam a leo sit amet ipsum
sagittis volutpat ut at tortor. Nullam quis adipiscing arcu. Nulla facilisi. Integer at dui turpis. Suspendisse luctus
rutrum dui ac mattis. In hac habitasse platea dictumst. Duis pharetra fringilla nulla, at molestie arcu fermentum
eu. Integer id metus vel enim auctor rutrum.

In hac habitasse platea dictumst. Pellentesque elementum dolor vel felis tincidunt in mollis lorem ornare. Fusce non
ligula massa, varius tempus eros. Quisque nunc magna, facilisis at luctus ut, venenatis vel dolor. Mauris nec accumsan
nulla. Fusce in lorem velit, non egestas mauris. Cras laoreet erat eu leo commodo aliquam. In eget mauris lacus, vitae
scelerisque lacus. Proin id massa augue. Ut ac euismod risus. Integer vulputate turpis eu urna pulvinar sit amet ornare
elit sagittis. Nunc mattis enim vel lectus mollis vitae viverra mauris sollicitudin. Proin id dui nunc. Ut ut quam
iaculis est tincidunt venenatis. Nunc at felis in augue suscipit scelerisque. Pellentesque eget tellus at odio malesuada
mollis aliquet ut magna. Suspendisse velit augue, rhoncus ut tempor non, tincidunt vel ligula. Proin elementum arcu sit
amet eros hendrerit lacinia eget et lectus. Donec non justo vitae ipsum consequat scelerisque. Nunc tincidunt, lacus non
fermentum molestie, tellus nibh aliquet diam, ac congue velit lacus eu diam.

Intro to \TeX\footnote{footnote}\marginpar{margin par}\\
\textit{The TeX Book}\footnote{The TeX Book}\\
\msjh 中文\rmfamily \\
\kai 楷書\rmfamily \\
\textcolor{blue}{blue}\\
\fbox{float box}\\ \\
baseline \raisebox{1ex}{upward}\raisebox{-1ex}{downward}\\\\
math expression $ (y^m)^n $ \\\\

tabbing
\begin{tabbing}
  column1 \= column2 \= column3 \\
  item1   \> item2   \> item3   \\
\end{tabbing}

tabular\\
\begin{tabular}{|l|c|r|}
  \hline
  \multicolumn{3}{|c|}{Sample Tabular}\\
  \hline
  first & second & third \\
  \cline{2-3}
  left  & centered & right \\
  \hline
\end{tabular}
\\\\
one -\\
two --\\
three ---\\
minus $ - $\\
control\ space\\
final\\
\TeX\ ignore space behind it.\\
\XeTeX\ does.\\
'\TeX'\\
Here is $\pi$ and $\Pi$\\
the angle of whole circle is 2$\pi$.\\
P\'olya\\





\section{Chapter Outline}

In chapter 2, we go through all common components should be implemented. Starting from the basic update equations in
free space, we adjust the formulas to suit the additional component such as Perfectly Matched Layer, Dispersive
Material, etc.



\chapter{The Finite-Difference Time-Domain Method}
\section{The Algorithm}

In 1966, The algorithm of the FDTD method was first introduced by Yee.

\subsection{Finite Difference}
The first thing being concerned is how to discrete space and time in FDTD, in other word, how to turn differential
equations to algerba equations.
\subsubsection{Explicit Leapfrog Scheme}
Laylor's series expansion 
\begin{equation}
  u(x_i+\Delta x) = u|_{x_i} + 
  \Delta x\cdot\left.\frac{\partial u}{\partial x}\right|_{x_i} + 
  \frac{(\Delta x)^2}{2}\cdot\left.\frac{\partial ^2 u}{\partial x^2}\right|_{x_i} + 
  \frac{(\Delta x)^3}{6}\cdot\left.\frac{\partial ^3 u}{\partial x^3}\right|_{x_i} + ...
\end{equation}
\begin{equation}
  u(x_i-\Delta x) = u|_{x_i} -
  \Delta x\cdot\left.\frac{\partial u}{\partial x}\right|_{x_i} + 
  \frac{(\Delta x)^2}{2}\cdot\left.\frac{\partial ^2 u}{\partial x^2}\right|_{x_i} -
  \frac{(\Delta x)^3}{6}\cdot\left.\frac{\partial ^3 u}{\partial x^3}\right|_{x_i} + ...
\end{equation}
difference 
\begin{equation}
  u(x_i+\Delta x) - u(x_i-\Delta x) = 2\Delta x\cdot\left.\frac{\partial u}{\partial x}\right|_{x_i}+...
\end{equation}
\begin{equation}
  \left.\frac{\partial u}{\partial x}\right|_{x_i} = \frac{u(x_i+\Delta x) - u(x_i-\Delta x)}{2\Delta x} = \frac{u^{i+1} - u^{i-1}}{2\Delta x} + O[(\Delta x)^3]
\end{equation}
sum 
\begin{equation}
  u(x_i+\Delta x) + u(x_i-\Delta x) = \left.2u\right|_{x_i} + (\Delta x)^2\cdot\left.\frac{\partial ^2 u}{\partial x^2}\right|_{x_i} + ...
\end{equation}
\begin{equation}
  \left.\frac{\partial^2 u}{\partial x^2}\right|_{x_i} = \frac{u(x_i+\Delta x) - 2u(x_i) + u(x_i-\Delta x)}{(\Delta x)^2} = \frac{u^{i+1} - 2u^i + u^{i-1}}{(\Delta x)^2} % + O[(\Delta x)^2]
\end{equation}
\subsubsection{Semi-Implicit Scheme}
\begin{equation}
  \left.\frac{\partial u}{\partial x}\right|_{x_i} = \frac{u(x_i+\Delta x) - u(x_i)}{\Delta x}
\end{equation}

\subsection{The Update Equations}

The update equations is the core of the FDTD method. In every interation on the timeline, the update equations refresh
the value of field of every point in the simulation region.

In this section, we are going to derive update equation from Maxwell's Equations.Here is the most well-known form of
Maxwell's Equations\index{Maxwell's Equations}:
\begin{gather}
  \label{eq:maxwell}
  \begin{array}{@{}rclr@{}}
    \nabla \cdot D & = & \rho_{\nu} & \mathrm{(Gaussian's\ Law)}\\
    \nabla \times E & = & {\displaystyle -\frac{\partial B}{\partial t}} & \mathrm{(Faraday's\ Law)}\\
    \nabla \cdot B & = & 0 & \\
    \nabla \times H & = & {\displaystyle J_s + \frac{\partial D}{\partial t}} & \mathrm{(Amp\`ere's\ Law)}
  \end{array}
\end{gather}
the equations shown above is for simple conductive media, the simplest lossy media having constant conductivity.

for getting sysmetrical form, novel magnetic current $M$ was added into Faraday's Law
\begin{gather}
  \frac{\partial D}{\partial t} = \nabla \times H - J\\
  \frac{\partial B}{\partial t} =  - \nabla \times E - M
\end{gather}
in terms of E and H
\begin{gather}
  \epsilon\frac{\partial E}{\partial t} = \nabla \times H - \sigma_eE\\
  \mu\frac{\partial H}{\partial t} = - \nabla \times E - \sigma_hH
\end{gather}
gather coefficient 
\begin{gather}
  \frac{\partial E}{\partial t} = \frac{1}{\epsilon_r\epsilon_0}\nabla\times H - \frac{\sigma_e}{\epsilon_r\epsilon_0}E\\
  \frac{\partial H}{\partial t} = - \frac{1}{\mu_r\mu_0}\nabla\times E - \frac{\sigma_h}{\mu_r\mu_0}H
\end{gather}
turn to Gaussian Unit 
\begin{equation}
  \label{eq:gaussian_unit}
  \begin{array}{@{}l@{}}
    {\displaystyle\widetilde{E} = \sqrt{\frac{\epsilon_0}{\mu_0}}E}\\
    {\displaystyle\widetilde{D} = \frac{1}{\sqrt{\epsilon_0\mu_0}}D}\\
    {\displaystyle\widetilde{B} = \frac{1}{\mu_0}B}
  \end{array}
\end{equation}
\begin{gather}
  \frac{\partial \widetilde{E}}{\partial t} = \frac{1}{\epsilon_r\sqrt{\mu_0\epsilon_0}}\nabla\times H - \frac{\sigma_e}{\epsilon_r\epsilon_0}\widetilde{E}\\
  \frac{\partial H}{\partial t} = - \frac{1}{\mu_r\sqrt{\mu_0\epsilon_0}}\nabla\times\widetilde{E} - \frac{\sigma_h}{\mu_r\mu_0}H
\end{gather}
\begin{gather}
  \left(\epsilon_r\frac{\partial}{\partial t} + \frac{\sigma_e}{\epsilon_0}\right)\widetilde{E} = \frac{1}{\sqrt{\mu_0\epsilon_0}}\nabla\times H\\
  \left(\mu_r\frac{\partial}{\partial t} + \frac{\sigma_h}{\mu_0}\right)H = - \frac{1}{\sqrt{\mu_0\epsilon_0}}\nabla\times\widetilde{E}
\end{gather}
$\frac{\partial}{\partial t} \rightarrow j\omega$
\begin{gather}
  j\omega\left(\epsilon_r + \frac{\sigma_e}{j\omega\epsilon_0}\right)\widetilde{E} = c\ \nabla\times H\\
  j\omega\left(\mu_r + \frac{\sigma_h}{j\omega\mu_0}\right)H = - c\ \nabla\times\widetilde{E}
\end{gather}
update equations and constitute relations in simple conductive media under Gaussian Units.
\begin{gather}
  \frac{\partial}{\partial t}\widetilde{D} = \frac{1}{\sqrt{\mu_0\epsilon_0}}\nabla\times H\label{eq:up_d}\\
  \widetilde{D}(\omega) = \left(\epsilon_r + \frac{\sigma_e}{j\omega\epsilon_0}\right)\widetilde{E} = \epsilon_r^*(\omega)\widetilde{E}(\omega)\label{eq:cr_d}\\
  \frac{\partial}{\partial t}\widetilde{B} = -\frac{1}{\sqrt{\mu_0\epsilon_0}}\nabla\times\widetilde{E}\label{eq:up_b}\\
  \widetilde{B}(\omega) = \left(\mu_r + \frac{\sigma_h}{j\omega\mu_0}\right)H = \mu_r^*(\omega)H(\omega)\label{eq:cr_b}
\end{gather}
It's obvious separating constitute relations from updating of electric flux $D$ and magnetic flux $B$. The
material-related coefficients were collected into constitute relations to handle different objects, so that no matter
what object was changed in region of simulation Eq.\ref{eq:up_d} and Eq.\ref{eq:up_b} keep in this form.

For example, The formulas can be simpilified to describe loseless dielectric media by setting $\sigma_e$, $\sigma_h$ as
zero in Eq.\ref{eq:cr_d} and Eq.\ref{eq:cr_b} to be
\begin{gather*}
  \widetilde{D}(\omega) = \epsilon_r\cdot\widetilde{E}(\omega)\\
  \widetilde{B}(\omega) = \mu_r\cdot H(\omega)
\end{gather*}
Performing inverse Fourier Transformation
\begin{gather*}
  \widetilde{D}(t) = \epsilon_r\cdot\widetilde{E}(t)\\
  \widetilde{B}(t) = \mu_r\cdot H(t)
\end{gather*}
but Eq.\ref{eq:up_d} and Eq.\ref{eq:up_b} need not any modification.

Or to describe freespace by setting $\sigma_e$, $\sigma_h$ as zero and $\epsilon_r$, $\mu_r$ as one 
\begin{gather*}
  \widetilde{D}(\omega) = \widetilde{E}(\omega)\\
  \widetilde{B}(\omega) = H(\omega)
\end{gather*}
Performing inverse Fourier Transformation again
\begin{gather*}
  \widetilde{D}(t) = \widetilde{E}(t)\\
  \widetilde{B}(t) = H(t)
\end{gather*}
In general, every material has its own $\epsilon_r^*(\omega)$ varying through whole frequency spectrum duo to its own
characters. By applying some mathematical trick Eq.\ref{eq:cr_d} and Eq.\ref{eq:cr_b} can be specialized for different
material to retrieve $E$ from $D$ in every time step but Eq.\ref{eq:up_d} and Eq.\ref{eq:up_b} can be applied directly
on every kinds of material. That's the best advanteage separating constitute relations out of the two update equations
would be introduce in \ref{sec:dispersive}.

This way also shows some advantagewhen handling perfecly matched layer which would be discussed in \ref{subsec:pml}.

Extend to Cartesian coordinate system.
\begin{gather}
  \frac{\partial}{\partial t}\widetilde{D}_x = \frac{1}{\sqrt{\mu_0\epsilon_0}}\left(\frac{\partial H_z}{\partial y} - \frac{\partial H_y}{\partial z}\right)\label{eq:up_d_x}\\
  \frac{\partial}{\partial t}\widetilde{D}_y = \frac{1}{\sqrt{\mu_0\epsilon_0}}\left(\frac{\partial H_x}{\partial z} - \frac{\partial H_z}{\partial x}\right)\label{eq:up_d_y}\\
  \frac{\partial}{\partial t}\widetilde{D}_z = \frac{1}{\sqrt{\mu_0\epsilon_0}}\left(\frac{\partial H_y}{\partial x} - \frac{\partial H_x}{\partial y}\right)\label{eq:up_d_z}\\
  \frac{\partial}{\partial t}\widetilde{B}_x =-\frac{1}{\sqrt{\mu_0\epsilon_0}}\left(\frac{\partial \widetilde{E}_z}{\partial y} - \frac{\partial \widetilde{E}_y}{\partial z}\right)\label{eq:up_b_x}\\
  \frac{\partial}{\partial t}\widetilde{B}_y =-\frac{1}{\sqrt{\mu_0\epsilon_0}}\left(\frac{\partial \widetilde{E}_x}{\partial z} - \frac{\partial \widetilde{E}_z}{\partial x}\right)\label{eq:up_b_y}\\
  \frac{\partial}{\partial t}\widetilde{B}_z =-\frac{1}{\sqrt{\mu_0\epsilon_0}}\left(\frac{\partial \widetilde{E}_y}{\partial x} - \frac{\partial \widetilde{E}_x}{\partial y}\right)\label{eq:up_b_z}
\end{gather}
\clearpage
Discrete use semi-implicit scheme
\begin{gather}
  \begin{array}{@{}l@{}}
    \displaystyle \frac{\widetilde{D}_x|_{i+\frac{1}{2},j,k}^{n+\frac{1}{2}} - \widetilde{D}_x|_{i+\frac{1}{2},j,k}^{n-\frac{1}{2}}}{\Delta t} = \\
    \displaystyle c_0\left(\frac{H_z|_{i+\frac{1}{2},j+\frac{1}{2},k}^{n} - H_z|_{i+\frac{1}{2},j-\frac{1}{2},k}^{n}}{\Delta y} - \frac{H_y|_{i+\frac{1}{2},j,k+\frac{1}{2}}^{n} - H_y|_{i+\frac{1}{2},j,k-\frac{1}{2}}^{n}}{\Delta z}\right)\\[3em]
    \displaystyle \frac{\widetilde{D}_y|_{i,j+\frac{1}{2},k}^{n+\frac{1}{2}} - \widetilde{D}_x|_{i,j+\frac{1}{2},k}^{n-\frac{1}{2}}}{\Delta t} = \\
    \displaystyle c_0\left(\frac{H_x|_{i,j+\frac{1}{2},k+\frac{1}{2}}^{n} - H_x|_{i,j+\frac{1}{2},k-\frac{1}{2}}^{n}}{\Delta z} - \frac{H_z|_{i+\frac{1}{2},j+\frac{1}{2},k}^{n} - H_z|_{i-\frac{1}{2},j+\frac{1}{2},k}^{n}}{\Delta x}\right)\\[3em]
    \displaystyle \frac{\widetilde{D}_z|_{i,j,k+\frac{1}{2}}^{n+\frac{1}{2}} - \widetilde{D}_z|_{i,j,k+\frac{1}{2}}^{n-\frac{1}{2}}}{\Delta t} = \\
    \displaystyle c_0\left(\frac{H_y|_{i+\frac{1}{2},j,k+\frac{1}{2}}^{n} - H_y|_{i-\frac{1}{2},j,k+\frac{1}{2}}^{n}}{\Delta x} - \frac{H_x|_{i,j+\frac{1}{2},k+\frac{1}{2}}^{n} - H_x|_{i,j-\frac{1}{2},k+\frac{1}{2}}^{n}}{\Delta y}\right)\\[3em]
    \displaystyle \frac{\widetilde{B}_x|_{i,j+\frac{1}{2},k+\frac{1}{2}}^{n+1} - \widetilde{B}_x|_{i,j+\frac{1}{2},k+\frac{1}{2}}^{n}}{\Delta t} = \\
    \displaystyle - c_0\left(\frac{\widetilde{E}_z|_{i,j+1,k+\frac{1}{2}}^{n+\frac{1}{2}} - \widetilde{E}_z|_{i,j,k+\frac{1}{2}}^{n+\frac{1}{2}}}{\Delta y} - \frac{\widetilde{E}_y|_{i,j+\frac{1}{2},k+1}^{n+\frac{1}{2}} - \widetilde{E}_y|_{i,j+\frac{1}{2},k}^{n+\frac{1}{2}}}{\Delta z}\right)\\[3em]
    \displaystyle \frac{\widetilde{B}_y|_{i+\frac{1}{2},j,k+\frac{1}{2}}^{n+1} - \widetilde{B}_y|_{i+\frac{1}{2},j,k+\frac{1}{2}}^{n}}{\Delta t} = \\
    \displaystyle - c_0\left(\frac{\widetilde{E}_x|_{i+\frac{1}{2},j,k+1}^{n+\frac{1}{2}} - \widetilde{E}_x|_{i+\frac{1}{2},j,k}^{n+\frac{1}{2}}}{\Delta z} - \frac{\widetilde{E}_z|_{i+1,j,k+\frac{1}{2}}^{n+\frac{1}{2}} - \widetilde{E}_z|_{i,j,k+\frac{1}{2}}^{n+\frac{1}{2}}}{\Delta x}\right)\\[3em]
    \displaystyle \frac{\widetilde{B}_z|_{i+\frac{1}{2},j+\frac{1}{2},k}^{n+1} - \widetilde{B}_z|_{i+\frac{1}{2},j+\frac{1}{2},k}^{n}}{\Delta t} = \\
    \displaystyle - c_0\left(\frac{\widetilde{E}_y|_{i+1,j+\frac{1}{2},k}^{n+\frac{1}{2}} - \widetilde{E}_y|_{i,j+\frac{1}{2},k}^{n+\frac{1}{2}}}{\Delta x} - \frac{\widetilde{E}_x|_{i+\frac{1}{2},j+1,k}^{n+\frac{1}{2}} - \widetilde{E}_x|_{i+\frac{1}{2},j,k}^{n+\frac{1}{2}}}{\Delta y}\right)
  \end{array}
\end{gather}
Throughout this thesis, we use regular grid, that is, $\Delta x = \Delta y = \Delta z$. That means we could rewrite update equations as
\begin{equation}
  \begin{split}
    \widetilde{D}_x|_{i+\frac{1}{2},j,k}^{n+\frac{1}{2}} & = \widetilde{D}_x|_{i+\frac{1}{2},j,k}^{n-\frac{1}{2}}\\
    & + \frac{c_0\Delta t}{\Delta x}\left(H_z|_{i+\frac{1}{2},j+\frac{1}{2},k}^{n} - H_z|_{i+\frac{1}{2},j-\frac{1}{2},k}^{n} - H_y|_{i+\frac{1}{2},j,k+\frac{1}{2}}^{n} + H_y|_{i+\frac{1}{2},j,k-\frac{1}{2}}^{n}\right)
  \end{split}
\end{equation}
\begin{equation}
  \begin{split}
    \widetilde{D}_y|_{i,j+\frac{1}{2},k}^{n+\frac{1}{2}} & = \widetilde{D}_x|_{i,j+\frac{1}{2},k}^{n-\frac{1}{2}}\\
    & + \frac{c_0\Delta t}{\Delta x}\left(H_x|_{i,j+\frac{1}{2},k+\frac{1}{2}}^{n} - H_x|_{i,j+\frac{1}{2},k-\frac{1}{2}}^{n} - H_z|_{i+\frac{1}{2},j+\frac{1}{2},k}^{n} + H_z|_{i-\frac{1}{2},j+\frac{1}{2},k}^{n}\right)
  \end{split}
\end{equation}
\begin{equation}
  \begin{split}
    \widetilde{D}_z|_{i,j,k+\frac{1}{2}}^{n+\frac{1}{2}} & = \widetilde{D}_z|_{i,j,k+\frac{1}{2}}^{n-\frac{1}{2}}\\
    & + \frac{c_0\Delta t}{\Delta x}\left(H_y|_{i+\frac{1}{2},j,k+\frac{1}{2}}^{n} - H_y|_{i-\frac{1}{2},j,k+\frac{1}{2}}^{n} - H_x|_{i,j+\frac{1}{2},k+\frac{1}{2}}^{n} + H_x|_{i,j-\frac{1}{2},k+\frac{1}{2}}^{n}\right)
  \end{split}
\end{equation}
\begin{equation}
  \begin{split}
    \widetilde{B}_x|_{i,j+\frac{1}{2},k+\frac{1}{2}}^{n+1} & = \widetilde{B}_x|_{i,j+\frac{1}{2},k+\frac{1}{2}}^{n}\\
    & - \frac{c_0\Delta t}{\Delta x}\left(\widetilde{E}_z|_{i,j+1,k+\frac{1}{2}}^{n+\frac{1}{2}} - \widetilde{E}_z|_{i,j,k+\frac{1}{2}}^{n+\frac{1}{2}} - \widetilde{E}_y|_{i,j+\frac{1}{2},k+1}^{n+\frac{1}{2}} - \widetilde{E}_y|_{i,j+\frac{1}{2},k}^{n+\frac{1}{2}}\right)
  \end{split}
\end{equation}
\begin{equation}
  \begin{split}
    \widetilde{B}_y|_{i+\frac{1}{2},j,k+\frac{1}{2}}^{n+1} & = \widetilde{B}_y|_{i+\frac{1}{2},j,k+\frac{1}{2}}^{n}\\
    & - \frac{c_0\Delta t}{\Delta x}\left(\widetilde{E}_x|_{i+\frac{1}{2},j,k+1}^{n+\frac{1}{2}} - \widetilde{E}_x|_{i+\frac{1}{2},j,k}^{n+\frac{1}{2}} - \widetilde{E}_z|_{i+1,j,k+\frac{1}{2}}^{n+\frac{1}{2}} - \widetilde{E}_z|_{i,j,k+\frac{1}{2}}^{n+\frac{1}{2}}\right)
  \end{split}
\end{equation}
\begin{equation}
  \begin{split}
    \widetilde{B}_z|_{i+\frac{1}{2},j+\frac{1}{2},k}^{n+1} & = \widetilde{B}_z|_{i+\frac{1}{2},j+\frac{1}{2},k}^{n}\\
    & - \frac{c_0\Delta t}{\Delta x}\left(\widetilde{E}_y|_{i+1,j+\frac{1}{2},k}^{n+\frac{1}{2}} - \widetilde{E}_y|_{i,j+\frac{1}{2},k}^{n+\frac{1}{2}} - \widetilde{E}_x|_{i+\frac{1}{2},j+1,k}^{n+\frac{1}{2}} - \widetilde{E}_x|_{i+\frac{1}{2},j,k}^{n+\frac{1}{2}}\right)
  \end{split}
\end{equation}
This is complete update equations derived for 3D cases.
\begin{equation}
  k+\frac{1}{2}\rightarrow k\quad \mathrm{and} \quad
  k-\frac{1}{2}\rightarrow k-1
\end{equation}
Pseudo code
\begin{code}
Points.each do 
  Dx[i,j,k] += 0.5 * ( Hz[i,j,k] - Hz[i,j-1,k] 
                     - Hy[i,j,k] + Hy[i,j,k-1] )
  Dy[i,j,k] += 0.5 * ( Hx[i,j,k] - Hx[i,j,k-1] 
                     - Hz[i,j,k] + Hz[i-1,j,k] )
  Dz[i,j,k] += 0.5 * ( Hy[i,j,k] - Hy[i-1,j,k] 
                     - Hx[i,j,k] + Hx[i,j-1,k] )
  Bx[i,j,k] -= 0.5 * ( Ez[i,j+1,k] - Ez[i,j,k] 
                     - Ey[i,j,k+1] + Ey[i,j,k] )
  By[i,j,k] -= 0.5 * ( Ex[i,j,k+1] - Ex[i,j,k] 
                     - Ez[i+1,j,k] + Ez[i,j,k] )
  Bz[i,j,k] -= 0.5 * ( Ey[i+1,j,k] - Ey[i,j,k] 
                     - Ex[i,j+1,k] + Ex[i,j,k] )
end
\end{code}


\subsection{Stability}
Courant Conditions, Courant Number
\begin{equation}
  \Delta t \le \frac{\Delta x}{\sqrt{n}\cdot c_0}
\end{equation}
where n is the dimension of the simulation. For the convenience of designing mentioned latter, throughout this thesis we determine
$\Delta t$ by
\begin{equation}
  \Delta t = \frac{\Delta x}{2 \cdot c_0}
\end{equation}


\subsection{Reduction to One Dimensions}
There are three selections to choose a one dimension EM string: $\mathrm{TEM_x}$ ($\mathrm{E_{y}}$, $\mathrm{H_{z}}$,
$\mathrm{k_x}$), $\mathrm{TEM_y}$ ($\mathrm{E_z}$, $\mathrm{H_x}$, $\mathrm{k_y}$), and $\mathrm{TEM_z}$
($\mathrm{E_x}$, $\mathrm{H_y}$, $\mathrm{k_z}$). Similarly, $\mathrm{TEM_z}$ is the default choice when saying
TEM. Following the definition of TEM, Eq.\ref{eq:up_d_x} and Eq.\ref{eq:up_b_y} were picked out for reduction of 1-D
case.
\begin{gather*}
  \frac{\partial}{\partial t}\widetilde{D}_x = \frac{1}{\sqrt{\mu_0\epsilon_0}}\left(\frac{\partial H_z}{\partial y} - \frac{\partial H_y}{\partial z}\right)\\
  \frac{\partial}{\partial t}\widetilde{B}_y =-\frac{1}{\sqrt{\mu_0\epsilon_0}}\left(\frac{\partial \widetilde{E}_x}{\partial z} - \frac{\partial \widetilde{E}_z}{\partial x}\right)
\end{gather*}
The choice implies
\begin{displaymath}
  \frac{\partial}{\partial x} \rightarrow 0\quad \mathrm{and} \quad
  \frac{\partial}{\partial y} \rightarrow 0
\end{displaymath}
apply
\begin{gather}
  \frac{\partial}{\partial t}\widetilde{D}_x = \frac{1}{\sqrt{\mu_0\epsilon_0}}\left( - \frac{\partial H_y}{\partial z}\right)\\
  \frac{\partial}{\partial t}\widetilde{B}_y =-\frac{1}{\sqrt{\mu_0\epsilon_0}}\left(\frac{\partial \widetilde{E}_x}{\partial z} \right)
\end{gather}
Discrete
\begin{gather}
  \frac{\widetilde{D}_x|_k^{n+\frac{1}{2}} - \widetilde{D}_x|_k^{n-\frac{1}{2}}}{\Delta t} = -c_0\cdot\frac{H_y|_{k+\frac{1}{2}}^n - H_y|_{k-\frac{1}{2}}^n}{\Delta z}\\
  \frac{\widetilde{B}_y|_{k+\frac{1}{2}}^{n+1} - \widetilde{B}_y|_{k+\frac{1}{2}}^n}{\Delta t} = -c_0\cdot\frac{\widetilde{E}_x|_{k+1}^{n+\frac{1}{2}} - \widetilde{E}_x|_{k}^{n+\frac{1}{2}}}{\Delta z}
\end{gather}
That is
\begin{gather}
  \widetilde{D}_x|_k^{n+\frac{1}{2}} = \widetilde{D}_x|_k^{n-\frac{1}{2}} - \frac{c_0\Delta t}{\Delta z}\left( H_y|_{k+\frac{1}{2}}^n - H_y|_{k-\frac{1}{2}}^n \right)\\
  \widetilde{B}_y|_{k+\frac{1}{2}}^{n+1} = \widetilde{B}_y|_{k+\frac{1}{2}}^{n} = - \frac{c_0\Delta t}{\Delta z}\left( \widetilde{E}_x|_{k+1}^{n+\frac{1}{2}} - \widetilde{E}_x|_{k}^{n+\frac{1}{2}} \right)
\end{gather}
Here is the code
\begin{code}
Points.each do
  Dx[k] += 0.5 * ( Hy[k-1] - Hy[k] )
  Hy[k] += 0.5 * ( Ex[k] - Ex[k+1] )
end
\end{code}




\subsection{Reduction to Two Dimensions}
There are 6 selections for us to choose a two dimensions EM plane: $\mathrm{TM_{x}} $, $\mathrm{TE_{x}}$,
$\mathrm{TM_{y}}$, $\mathrm{TE_{y}}$, $\mathrm{TM_{z}}$, $\mathrm{TE_{z}}$. By default, the choice in this thesis follow
the book of Taflove using $\mathrm{TM_{z}}$ ($\mathrm{H_x}$, $\mathrm{H_y}$, and $\mathrm{E_z}$) and $\mathrm{TE_{z}}$
($\mathrm{E_x}$, $\mathrm{E_y}$, and $\mathrm{H_z}$) as convention when saying TM and TE.
\begin{displaymath}
  \frac{\partial}{\partial z} \rightarrow 0
\end{displaymath}

$\mathrm{TM_z}$
\begin{displaymath}
  \frac{\partial}{\partial t}\widetilde{D}_z = \frac{1}{\sqrt{\mu_0\epsilon_0}}\left(\frac{\partial H_y}{\partial x} - \frac{\partial H_x}{\partial y}\right)
\end{displaymath}
\begin{displaymath}
  \frac{\partial}{\partial t}\widetilde{B}_x =-\frac{1}{\sqrt{\mu_0\epsilon_0}}\left(\frac{\partial \widetilde{E}_z}{\partial y} - \frac{\partial \widetilde{E}_y}{\partial z}\right)
\end{displaymath}
\begin{displaymath}
  \frac{\partial}{\partial t}\widetilde{B}_y =-\frac{1}{\sqrt{\mu_0\epsilon_0}}\left(\frac{\partial \widetilde{E}_x}{\partial z} - \frac{\partial \widetilde{E}_z}{\partial x}\right)
\end{displaymath}
Discretize
\begin{displaymath}
  \frac{\widetilde{D}_z|_{i,j}^{n+\frac{1}{2}}-\widetilde{D}_z|_{i,j}^{n+\frac{1}{2}}}{\Delta t} =
  c_0 \left(\frac{H_y|_{i+\frac{1}{2},j}^{n} - H_y|_{i-\frac{1}{2},j}^n}{\Delta x} - \frac{H_x|_{i,j+\frac{1}{2}}^{n} - H_x|_{i,j-\frac{1}{2}}^{n}}{\Delta y}\right)
\end{displaymath}
\begin{displaymath}
  \frac{\widetilde{B}_x|_{i,j+\frac{1}{2}}^{n+1} - \widetilde{B}_x|_{i,j+\frac{1}{2}}^{n}}{\Delta t} = 
  - c_0\left(\frac{\widetilde{E}_z|_{i,j+1}^{n+\frac{1}{2}} - \widetilde{E}_z|_{i,j}^{n+\frac{1}{2}}}{\Delta y}\right)
\end{displaymath}
\begin{displaymath}
  \frac{\widetilde{B}_y|_{i+\frac{1}{2},j}^{n+1} - \widetilde{B}_y|_{i+\frac{1}{2},j}^{n}}{\Delta t} =
  - c_0\left( - \frac{\widetilde{E}_z|_{i+1,j}^{n+\frac{1}{2}} - \widetilde{E}_z|_{i,j}^{n+\frac{1}{2}}}{\Delta x}\right)
\end{displaymath}
Pseudo code for $\mathrm{TM_z}$ polarization
\begin{code}
Points.each do 
  
end
\end{code}

$\mathrm{TE_z}$
\begin{displaymath}
    \frac{\partial}{\partial t}\widetilde{B}_z =-\frac{1}{\sqrt{\mu_0\epsilon_0}}\left(\frac{\partial \widetilde{E}_y}{\partial x} - \frac{\partial \widetilde{E}_x}{\partial y}\right)\label{eq:up_b_z}
\end{displaymath}
\begin{displaymath}
  \frac{\partial}{\partial t}\widetilde{D}_x = \frac{1}{\sqrt{\mu_0\epsilon_0}}\left(\frac{\partial H_z}{\partial y} - \frac{\partial H_y}{\partial z}\right)
\end{displaymath}
\begin{displaymath}
  \frac{\partial}{\partial t}\widetilde{D}_y = \frac{1}{\sqrt{\mu_0\epsilon_0}}\left(\frac{\partial H_x}{\partial z} - \frac{\partial H_z}{\partial x}\right)
\end{displaymath}
Discretize
\begin{displaymath}
  \frac{\widetilde{B}_z|_{i+\frac{1}{2},j+\frac{1}{2}}^{n} - \widetilde{B}_z|_{i+\frac{1}{2},j+\frac{1}{2}}^{n}}{\Delta t} =
  c_0\left(\frac{\widetilde{E}_y|_{i+1,j+\frac{1}{2}}^{} - \widetilde{E}_y|_{i,j+\frac{1}{2}}^{}}{\Delta x} - \frac{\widetilde{E}_x|_{i+\frac{1}{2},j+1}^{} - \widetilde{E}_x|_{i+\frac{1}{2},j}^{}}{\Delta y}\right)
\end{displaymath}
\begin{displaymath}
  \frac{\widetilde{D}_x|_{i+\frac{1}{2},j}^{n+\frac{1}{2}} - \widetilde{D}_x|_{i+\frac{1}{2},j}^{n-\frac{1}{2}}}{\Delta t} =
  c_0\left(\frac{H_z|_{i+\frac{1}{2},j+\frac{1}{2}}^{n} - H_z|_{i+\frac{1}{2},j-\frac{1}{2}}^{n}}{\Delta y} - \right)
\end{displaymath}
\begin{displaymath}
  \frac{\widetilde{D}_y|_{i,j+\frac{1}{2}}^{n+\frac{1}{2}} - \widetilde{D}_x|_{i,j+\frac{1}{2}}^{n-\frac{1}{2}}}{\Delta t} =
  c_0\left( - \frac{H_z|_{i+\frac{1}{2},j+\frac{1}{2}}^{n} - H_z|_{i-\frac{1}{2},j+\frac{1}{2}}^{n}}{\Delta x}\right)
\end{displaymath}

Pseudocode for $\mathrm{TE_z}$ polarization
\begin{code}
Points.each do
end
\end{code}








\section{Incident Source Conditions}
Primarily there are two kinds of sources in the FDTD method: point sources and plane wave sources.

Point source can be classified as hard source or soft source. If the source value is directly assigned to the certain
point, it is referred to as a hard source. Oppositely, it can be called a soft source if the source value is added to
the field at that point. A hard source would lead to some reflection between certain points and adjacent points and a
soft source allows the wave just passes through. Point source is not common used in 2D and 3D simulations but plays an
important role on constructing plane-wave source conditions.

For investigating nano structures, it is often of interest to impinge plane wave source upon structures and measure
transmission and reflection. The calculation of radar cross section also deals with the plane wave. Total Field /
Scatter Field (TFSF) [Taflove and Hagness, 2005] is a technique to simulate a plane wave by dividing the problem space
into the Totol Field region and the Scatter Field region, as shown in Fig 2.3.

The TFSF formulas can be observed through the complete update equations and the totol/scatter field theory as
\begin{displaymath}\label{eq:e_ts}
  E_{total}=E_{scatter}+E_{incident}
\end{displaymath}
\begin{displaymath}\label{eq:h_ts}
  H_{total}=H_{scatter}+H_{incident}  
\end{displaymath}
As an example, consider a the y-incident plane wave of TM polarization in the TFSF region $x=ia:ib, y=ja:jb, z=ka:kb$,
(\ref{eq:dz3d}) performed on the edage $y=ja$ of the main computation domain is actually
\begin{equation}\label{eq:tsedge}
  \begin{split}
    \widetilde{D}_{z,total}|_{i,ja,k} = \widetilde{D}_{z,total}|_{i,ja,k} &+ 0.5 \cdot \left( H_{y,total}|_{i+\frac{1}{2},ja,k+\frac{1}{2}} - H_{y,total}|_{i-\frac{1}{2},ja,k+\frac{1}{2}} \right) \\
    &- 0.5 \cdot \left( H_{x,total}|_{i,ja+\frac{1}{2},k+\frac{1}{2}} - H_{x,scatter}|_{i,ja-\frac{1}{2},k+\frac{1}{2}} \right)    
  \end{split}
\end{equation}
In (\ref{eq:tsedge}), the last $H_x$ belongs to the scattered field outside the TFSF boundary. By applying
(\ref{eq:h_ts}) to complete the curl operation, the insertion of the TFSF source of $\widetilde{D}_z$ is described by
\begin{displaymath}
  \widetilde{D}_z|_{i,ja,k} = \widetilde{D}_z|_{i,ja,k} + 0.5 \cdot H_{inc}|_{ja-\frac{1}{2}}
\end{displaymath}
The rest equations for the insertion of the TFSF can also be dervied as
\begin{displaymath}
  \widetilde{D}_z|_{i,jb,k} = \widetilde{D}_z|_{i,jb,k} - 0.5 \cdot H_{inc}|_{jb+\frac{1}{2}}  
\end{displaymath}
\begin{displaymath}
  \widetilde{B}_x|_{i,ja-\frac{1}{2},k+\frac{1}{2}}=\widetilde{B}_x|_{i,ja-\frac{1}{2},k+\frac{1}{2}}+0.5 \cdot E_{inc}|_{ja}
\end{displaymath}
\begin{displaymath}
  \widetilde{B}_x|_{i,jb+\frac{1}{2},k+\frac{1}{2}}=\widetilde{B}_x|_{i,jb+\frac{1}{2},k+\frac{1}{2}}-0.5 \cdot E_{inc}|_{jb}
\end{displaymath}
\begin{displaymath}
  \widetilde{B}_y|_{ia-\frac{1}{2},ja:jb,k+\frac{1}{2}}=\widetilde{B}_y|_{ia-\frac{1}{2},ja:jb,k+\frac{1}{2}}-0.5 \cdot E_{inc}|_{ja:jb}
\end{displaymath}
\begin{displaymath}
  \widetilde{B}_y|_{ib+\frac{1}{2},ja:jb,k+\frac{1}{2}}=\widetilde{B}_y|_{ib+\frac{1}{2},ja:jb,k+\frac{1}{2}}+0.5 \cdot E_{inc}|_{ja:jb}
\end{displaymath}







\section{Boundary Conditions}

\subsection{Absorbing Boundary Conditions}

The Absorbing Boundary Conditions were proposed

\subsection{Perfectly Matched Layer}

For 2-D and 3-D simulation, the ABC discussed above is no longer useful due to the reason the ABC can only absorb
normally incident wave.

\subsubsection{Berenger's PML}
the implementation of Berenger's PML

aka Split-Field PML
\subsubsection{Unaxial PML}
the implementations of Unaxial PML
\subsubsection{Convolution PML}
the implementations of Convolution PML

\subsection{Periodic Boundary Conditions}

For simulating Bloch periodic structure such as Photonic Crystals, the periodic boundary conditions (PBCs) were useful
to reduce the simulation region

\section{Dispersive Material}
\label{sec:dispersive}
This section conerns of how to retrieve the update equations could apply in dispersive material. What would be discussed
below are only about material having dispersive electic permittivity. The material having dispersive magnetic
permeability could be investigated via duality.

In fundamental Electromagnetics, an indispersive material having a electrical polarization $P$ when there is a foreign
electric field and the phasor of electric flux would become $D(\omega) = \epsilon_0 E(\omega) + P$. $P$ was also defined
as $\epsilon_0 \chi_e E(\omega)$, so that we could rewrite $D(\omega)$ as $\epsilon_0 (1+\chi_e)E(\omega)$ and give the middle term a notation
$\epsilon_r$ named relative permittivity.

By definition, dispersive material is the material having different electic permeability when encountering EM wave
having different frequency that means the relative permittivity should be a function of frequency ordinarily coming with
image part. So, $\epsilon_r^*(\omega)$ was given as the notation of dispersive electic permittivity and the phasor of
electic flux in dispersive material was defined as $D(\omega) = \epsilon_0 \epsilon_r^*(\omega)E(\omega)$ or
$\widetilde{D} = \epsilon_r^*(\omega)\widetilde{E}$ in Gaussian Unit.
\subsection{Common Isotropic Dispersive Material}
The dispersive permittivity is defined as $\epsilon_r^*(\omega) = 1 + \chi_e + \sum \chi_p(\omega)$ and usually, $1 +
\chi_e$ are written as $\epsilon_r$ to be identical with fundamental electromagnetics.
\subsubsection{Simple Lossy Media}
Simple Lossy Media is a material with constant electric conductivity $\sigma_e$.
\begin{gather}
  \chi_p(\omega) = \frac{\sigma_e}{j\omega\epsilon_0}\\
  \epsilon_r^*(\omega) = \epsilon_r + \frac{\sigma_e}{j\omega\epsilon_0}
\end{gather}

\subsubsection{Debye Media}
\begin{equation}
  \label{eq:debye_chi}
  \chi_p(\omega) = \frac{\Delta\epsilon_p}{1+j\omega\tau_p}  
\end{equation}
\begin{equation}
  \epsilon_r^*(\omega) = \epsilon_r + \sum_{p=1}^P \frac{\Delta\epsilon_p}{1+j\omega\tau_p}  
\end{equation}


\subsubsection{Drude Media}
\begin{equation}
  \label{eq:drude_chi}
  \chi_p(\omega) = -\frac{\omega_p^2}{\omega^2 - j\omega\gamma_p}  
\end{equation}
\begin{equation}
  \epsilon_r^*(\omega) = \epsilon_r - \sum_{p=1}^P \frac{\omega_p^2}{\omega^2-j\omega\gamma_p}
\end{equation}
where $\omega_p$ is the Drude pole frequency and $\gamma_p$ is the inverse of the pole relaxation time, also known as the
electron oscillision frequency.
\begin{equation}
  m_e\frac{\partial^2 x(t)}{\partial t^2} + m_e\gamma_p\frac{\partial x(t)}{\partial t} = -eE(t)
\end{equation}
$x(t) = \mathrm{Re}\{x_0e^{j\omega t}\}$ and $E(t) = \mathrm{Re}\{E_0e^{j\omega t}\}$
\begin{equation}
  x(t) = \frac{e}{m_e(\omega^2 - j\omega\gamma_p)}E(t)
\end{equation}
Because polarization $P = -Ne\cdot x(t)$ should also satisfy $P = \epsilon_0\chi_pE(t)$, $\chi_p$ can be solved
\begin{equation}
  \chi_p = \frac{-Ne^2/m_e\epsilon_0}{\omega^2 - j\omega\gamma_p}
\end{equation}
With denoting $-Ne^2/m_e\epsilon_0$ by $\omega_p^2$. it becomes Eq.\ref{eq:drude_chi}


\subsubsection{Lorentz Media}
\begin{equation}
  \label{eq:lorentz_chi}
  \chi_p(\omega) = \frac{\Delta\epsilon_p\omega_p^2}{\omega_p^2 + 2j\omega\delta_p - \omega^2}  
\end{equation}
\begin{equation}
  \epsilon_r^*(\omega) = \epsilon_r + \sum_{p=1}^P \frac{\Delta\epsilon_p\omega_p^2}{\omega_p^2 + 2j\omega\delta_p - \omega^2}  
\end{equation}
start from
\begin{equation}
    m_e\frac{\partial^2 x(t)}{\partial t^2} + 2m_e\delta_L\frac{\partial x(t)}{t} + m_e\omega_L^2x(t) = -eE(t)
\end{equation}







\subsection{Dispersion-Compatible Update Equations}
For the analysis of dispersive materials shown above, numbers of algorithms have already been proposed in literature.
Most of these frequency-dependent algorithms can be categorized into three types: 
\begin{inparaenum}[(1)]
\item the Recursive Convolution (RC) method
\item the Auxiliary Differential Equation (ADE) method
\item the Z-transform (ZT) method
\end{inparaenum}.

The RC method is the most basic method inheriting the convolution theorem in Laplace transform and Fourier
transform. The pros and cons are easy to understand with background of engineering mathematics and difficult to handle
complex material with multiple poles.

The ADE method offer high flexibility in fitting arbitrary permittivity functions, modeling nonlinear effects and
arbitrary numbers of poles.

Any of these methods focuses on tranforming the dispersive relations between $D(\omega))$ and $E(\omega)$ in frequency domain
back to the time doamin for discretization. By applying different discretizing scheme at each step, many varieties of
them were given. Following words are trying to give an overview and examples as many as possible to be a reference
during implementing.

\subsubsection{The Recursive Convolution Method}
There are many varieties published using convolution including 
\begin{inparaenum}[(1)]
\item Recursive Convolution (RC) Method
\item Piecewise-Linear Recursive Convolution (PLRC) Method
\item Trapezoidal Recursive Convolution(TRC) Method
\end{inparaenum}

\paragraph{\msjh Simple Conductive Media - RC}
Apply RC to simple conductive material having definition as
\begin{displaymath}
  \epsilon_r^{*}(\omega) = \epsilon_r + \frac{\sigma}{j \omega \epsilon_0}
\end{displaymath}
\begin{displaymath}
  \begin{split}
    \widetilde{D}(\omega) & = \epsilon_r^{*}(\omega)\widetilde{E}(\omega)\\
    & = \epsilon_r\widetilde{E}(\omega) + \frac{\sigma}{j\omega\epsilon_0}\widetilde{E}(\omega)
  \end{split}
\end{displaymath}
\begin{displaymath}
  D(t) = \epsilon_r\widetilde{E}(t) + \frac{\sigma}{\epsilon_0}\int_0^t\widetilde{E}(t')dt'
\end{displaymath}
\begin{displaymath}
    D^n = \epsilon_r\widetilde{E}^n + \frac{\sigma}{\epsilon_0}\Delta t \sum_{i=0}^{n}\widetilde{E}^i
\end{displaymath}
The second term: dissipated displacement -> $I^n$, in recursive form
\begin{equation}
  I^n = \frac{\sigma}{\epsilon_0}\Delta t\cdot\widetilde{E}^n + \frac{\sigma}{\epsilon_0}\Delta t\sum_{i=0}^{n-1}\widetilde{E}^i = \frac{\sigma}{\epsilon_0}\Delta t\cdot\widetilde{E}^n + I^{n-1}
\end{equation}
\begin{equation}
  \widetilde{D}^n = \epsilon_r\widetilde{E}^n + I^n = (\epsilon_r + \frac{\sigma}{\epsilon_0}\Delta t)\widetilde{E}^n + I^{n-1}
\end{equation}
Finally, the update equations becomes
\begin{gather*}
  \widetilde{E}^n = \frac{\widetilde{D}^n - I^{n-1}}{\displaystyle \epsilon_r +\frac{\sigma}{\epsilon_0}\Delta t}\\
  I^n = I^{n-1} + \frac{\sigma}{\epsilon_0}\Delta t\cdot E^n
\end{gather*}
implementation
\begin{code}
  dx[k] = dx[k]
  ex[k] = ex[k]
  i[k]  = i[k]
\end{code}


\paragraph{{\msjh Debye Model - RC}}
\begin{displaymath}
  \epsilon_r^*(\omega) = \epsilon_r + \frac{\sigma}{j\omega \epsilon_0} + \frac{\Delta \epsilon_p}{1+j\omega \tau_p}
\end{displaymath}
\begin{displaymath}
  \begin{split}
    \widetilde{D}(\omega) & = \epsilon_r^*(\omega)\widetilde{E}(\omega)\\
    & = \epsilon_r\widetilde{E}(\omega) + \frac{\sigma}{j\omega\epsilon_0}\widetilde{E}(\omega) + \frac{\Delta \epsilon_p}{1+j\omega \tau_p}\widetilde{E}(\omega)
  \end{split}
\end{displaymath}
\begin{displaymath}
  \widetilde{D}(t) = \epsilon_r\widetilde{E}(t) + \frac{\sigma}{\epsilon_0}\int_0^t\widetilde{E}(t')dt' + \int_0^t\frac{\Delta \epsilon_p}{\tau_p}e^{-\frac{t'-t}{\tau_p}}\widetilde{E}(t')dt'
\end{displaymath}
\begin{equation}
  \widetilde{D}^n = \epsilon_r\widetilde{E}^n + \frac{\sigma}{\epsilon_0}\Delta t\sum_{i=0}^{n}\widetilde{E}^i + \frac{\Delta \epsilon_p}{\tau_p}\Delta t \sum_{i=0}^{n} e^{-\frac{n-i}{\tau_p}\Delta t}\widetilde{E}^i
\end{equation}
The second term, dissipated displacement, $I^n$ , the same as simple conductive media.
The third term, phasor polarization displacement, $J_p^n$. $J_p^n$ is slightly complex in recursive form.
\begin{equation}
  \begin{split}
    J_p^n & = \frac{\Delta\epsilon_p}{\tau_p}\Delta t \sum_{i=0}^ne^{-\frac{n-i}{\tau_p}\Delta t}\widetilde{E}^i\\
    & = \frac{\Delta\epsilon_p}{\tau_p}\Delta t \left(\widetilde{E}^n + \sum_{i=0}^{n-1}e^{-\frac{n-i}{\tau_p}\Delta t}\widetilde{E}^i\right)
  \end{split}
\end{equation}
and 
\begin{equation}
  \begin{split}
    J_p^{n-1} & = \frac{\Delta\epsilon_p}{\tau_p}\Delta t \sum_{i=0}^{n-1}e^{-\frac{n-1-i}{\tau_p}\Delta t}\widetilde{E}^i\\
    & = \frac{\Delta\epsilon_p}{\tau_p}\Delta t \left( e^{\frac{\Delta t}{\tau_p}} \right) \sum_{i=0}^{n-1}e^{-\frac{n-i}{\tau_p}\Delta t}\widetilde{E}^i
  \end{split}
\end{equation}
substituted into $J_p^n$
\begin{equation}
  J_p^n = \frac{\Delta\epsilon_p}{\tau_p}\Delta t\cdot\widetilde{E}^n + e^{-\frac{\Delta t}{\tau_p}} J_p^{n-1}
\end{equation}
\begin{equation}
  \begin{split}
    \widetilde{D}^n & = \epsilon_r\widetilde{E}^n + \left[\frac{\sigma}{\epsilon_0}\Delta t\cdot\widetilde{E}^n + I^{n-1}\right] + \left[\frac{\Delta \epsilon_p}{\tau_p}\Delta t\cdot\widetilde{E}^n + e^{-\frac{\Delta t}{\tau_p}} J_p^{n-1}\right]\\
    & = \left(\epsilon_r + \frac{\sigma}{\epsilon_0}\Delta t + \frac{\Delta\epsilon_p}{\tau_p}\Delta t\right)\widetilde{E}^n + I^{n-1} + e^{-\frac{\Delta t}{\tau_p}} J_p^{n-1}
  \end{split}
\end{equation}
Finally the discrete constitute relations in time domain becomes
\begin{gather}
  \begin{array}{@{}l@{}}
    \widetilde{E}^n =  \frac{\displaystyle \widetilde{D}^n - I^{n-1} - e^{-\frac{\Delta t}{\tau_p}}J_p^{n-1} }{\displaystyle \epsilon_r + \frac{\sigma}{\epsilon_0}\Delta t + \frac{\Delta \epsilon_p}{\tau_p} \Delta t}\\    
    I^n = \frac{\sigma}{\epsilon_0}\Delta t\cdot\widetilde{E}^n + I^{n-1}\\
    J_p^n = \frac{\Delta\epsilon_p}{\tau_p}\Delta t\cdot\widetilde{E}^n + e^{-\frac{\Delta t}{\tau_p}} J_p^{n-1}
  \end{array}
\end{gather}
implement
\begin{code}
  dx[k] = dx[k] + 0.5 * ( hy[k-1] - hy[k])
  ex[k] = ( dx[k] - i[k] - exp(-dt/tau_p) * j[k] ) 
  * ( epsilon_0 * epsilon_r[k] + sigma_e[k] * dt +  )
  i[k] = i[k] + sigma_e[k] 
  j[k] = j[k] + del
\end{code}






\paragraph{{\msjh Lorentz Model - RC}}


\paragraph{{\msjh Drude Model - RC}}
\begin{equation}
  D(\omega) = \epsilon_0\epsilon_r^*(\omega)E(\omega) = \epsilon_0\epsilon_rE(\omega) - \epsilon_0\frac{\omega_p^2}{\omega^2-j\omega\gamma_p}E(\omega)
\end{equation}
\begin{equation}
  D(t) = \epsilon_0\epsilon_rE(t) + \epsilon_0\int_0^t -\frac{\omega_p^2}{\gamma_p}(1 - e^{-\gamma_p(t'-t)})E(t')dt'
\end{equation}
\begin{equation}
  D^n = \epsilon_0\epsilon_rE^n + \epsilon_0\frac{\omega_p^2}{\gamma_p}\Delta t\sum_{i=0}^{n}E^i + \epsilon_0\frac{\omega_p^2}{\gamma_p}\Delta t \sum_{i=0}^{n}e^{-\gamma_p(n-i)\Delta t} E^i
\end{equation}



\subsubsection{The Auxiliary Differential Equation Method}
The central idea of the Auxiliary Differential Equation (ADE) Method is to detach the phasor polarization displacement
$J_p(\omega)$, that is, $\chi_p(\omega)E(\omega)$, from original dispersive relation.
\begin{equation}
  \begin{split}
    \widetilde{D}(\omega) &= \epsilon_r^*(\omega)\widetilde{E}(\omega)\\
    & = \epsilon_r\widetilde{E}(\omega) + \sum_{p=1}^P\chi_p(\omega)\widetilde{E}(\omega)\\
    & = \epsilon_r\widetilde{E}(\omega) + \sum_{p=1}^{P}J_p(\omega)
  \end{split}
\end{equation}
For muiltipole material composed of different dispersions, $J_p$ were solved for each pole individually to attend the
updating loop, and then $E$ can be solved via rearrangment of previous relation.
\begin{equation}
  \widetilde{E}^n = \frac{1}{\epsilon_r}\left(\widetilde{D}^n - \sum_{p=1}^PJ_p^n\right)
\end{equation}
Primitive ADE method shown in literature require deriving formulations for each dispersion type, however, a generalized
way \textit{Alsunaidi et al.} proposed finds its strength in unifying the formulation of different dispersion models
into one form.

Starting with the most general form of $\chi_p(\omega)$, the phasor polarization displacement can be written
as
\begin{equation}
  J_p(\omega) = \frac{a}{b + j\omega c - d\omega^2}\widetilde{E}(\omega)
\end{equation}
Rearranging and performming inverse Fourier transform 
\begin{equation}
  bJ_p(t) + c \frac{\partial}{\partial t}J_p(t) + d \frac{\partial ^2}{\partial t^2}J_p(t) = a\widetilde{E}(t)
\end{equation}
apply leapfrog scheme 
\begin{equation}
  bJ_p^{n-1} + c \frac{J_p^n - J_p^{n-1}}{2\Delta t} + d \frac{J_p^n + 2J_p^{n-1} + J_p^{n-2}}{(\Delta t)^2} = a\widetilde{E}^{n-1}
\end{equation}
Solving 
\begin{equation}
  J_p^n = \frac{4d-2b(\Delta t)^2}{2d+c\Delta t}J_p^{n-1} + \frac{-2d+c\Delta t}{2d+c\Delta t}J_p^{n-2} + \frac{2a(\Delta t)^2}{2d+c\Delta t}\widetilde{E}^{n-1}
\end{equation}
which can be written in the form 
\begin{equation}
  J_p^n = C_1 J_p^{n-1} + C_2 J_p^{n-2} + C_3 \widetilde{E}^{n-1}
\end{equation}
where $C_1$, $C_2$ and $C_3$ can be found for any form of dispersion relation.



\paragraph{{\msjh Debye Model - ADE}}
Debye model is a fun case can be induced the same form of $J_p(t)$ as RC if the semi-implicit scheme was chose. However,
leapfrog scheme provide better accruacy. Just for verifying, result of apply semi-implicit scheme was also derived
here. In the real wrold implementation \textit{\uwave{yaFDTD}}, leapfrog was picked out.

Starting with constitute relation as before
\begin{equation}
  \begin{split}
    \widetilde{D}(\omega) & = \epsilon_r^*(\omega)\widetilde{E}(\omega)\\
    & = \epsilon_r\widetilde{E}(\omega) + \frac{\sigma}{j\omega\epsilon_0}\widetilde{E}(\omega) + \frac{\Delta \epsilon_p}{1+j\omega \tau_p}\widetilde{E}(\omega)\label{eq:debye_ade_start}
  \end{split}
\end{equation}
detach $J_p$
\begin{equation}
  J_p(\omega) = \frac{\Delta \epsilon_p}{1+j\omega \tau_p}\widetilde{E}(\omega)
\end{equation}
This is the ADE of Debye Model
\begin{equation}
  J_p(\omega) + j\omega\tau_{p}J_p(\omega) = \Delta\epsilon_p\widetilde{E}(\omega)
\end{equation}
performming IFT 
\begin{equation}
  J_p(t) + \tau_p\frac{\partial}{\partial t}J_p(t) = \Delta\epsilon_p\widetilde{E}(t)
\end{equation}
apply semi-implicit scheme
\begin{equation}
  \left( \frac{J_p^n - J_p^{n-1}}{2} \right) + \tau_p \left( \frac{J_p^n - J_p^{n-1}}{\Delta t}\right) = \Delta\epsilon_p\widetilde{E}^n
\end{equation}
Solving $J_p$
\begin{equation}
  J_p^n = \frac{\left(1-\frac{\Delta t}{2\tau_p}\right)}{\left(1+\frac{\Delta t}{2\tau_p}\right)}J_p^{n-1} 
  + \frac{\left(\frac{\Delta\epsilon_p}{\tau_p}\right)\Delta t}{\left(1+\frac{\Delta t}{2\tau_p}\right)}\widetilde{E}^n
\end{equation}
It should be noted
\begin{equation}
  \begin{array}{@{}lp{0.5cm}r@{}}
    \frac{\displaystyle1-\delta}{\displaystyle1+\delta} \cong e^{-2\delta} && if\ \delta \ll 1
  \end{array}
\end{equation}
\begin{equation}
  \frac{\left(1-\frac{\displaystyle\Delta t}{\displaystyle2\tau_p}\right)}{\left(1+\frac{\displaystyle\Delta t}{\displaystyle2\tau_p}\right)} \cong e^{-\frac{\Delta t}{\tau_p}}\quad  because\ 1 \gg \frac{\Delta t}{2\tau_p}
\end{equation}
That is 
\begin{equation}
  J_p^n \cong e^{-\frac{\Delta t}{\tau_p}}J^{n-1} + \frac{\Delta\epsilon_p}{\tau_p}\Delta t\cdot\widetilde{E}^n
\end{equation}
And by performming inverse Fourier transform on Eq.\ref{eq:debye_ade_start}
\begin{equation}
  \widetilde{D}(t) = \epsilon_r\widetilde{E}(t) + \frac{\sigma}{\epsilon_0} \int_0^t\widetilde{E}(t')dt' + J_p(t)
\end{equation}
\begin{equation}
  \begin{split}
    \widetilde{D}^n & = \epsilon_r\widetilde{E}^n + \frac{\sigma}{\epsilon_0}\Delta t\sum_{i=0}^n\widetilde{E}^i + J_p^n\\
    & = \epsilon_r\widetilde{E}^n + \frac{\sigma}{\epsilon_0}\Delta t\cdot\widetilde{E}^n + I^{n-1} + \frac{\Delta\epsilon_p}{\tau_p}\Delta t\cdot\widetilde{E}^n + e^{-\frac{\Delta t}{\tau_p}}J^{n-1}
  \end{split}
\end{equation}
The same result as Recursive Convolution Method.






\paragraph{\msjh Lorentz Model - ADE} Lorentz pole totally matches the general form used in previous introduction of ADE method.
The coefficients are as following.
\begin{equation*}
  \begin{array}{@{}llll@{}}
    a = \Delta\epsilon_p\omega_p^2 &
    b = \omega_p^2 &
    c = 2\delta_p &
    d = 1
  \end{array}
\end{equation*}
substitute into ...
\begin{gather*}
  \begin{array}{@{}lll@{}}
    C_1 = \frac{\displaystyle 2-\omega_p^2\Delta t^2}{\displaystyle 1+\delta_p\Delta t} &
    C_2 = \frac{\displaystyle -1 + \delta_p\Delta t}{\displaystyle 1+\delta_p\Delta t} &
    C_3 = \frac{\displaystyle \Delta\epsilon_p\omega_p^2\Delta t^2}{\displaystyle 1+\delta_p\Delta t}
  \end{array}
\end{gather*}
Formal derivation is also written down here.
\begin{equation}
  \begin{split}
    \widetilde{D}(\omega) & = \epsilon_r^*(\omega)\widetilde{E}(\omega)\\
    & = \epsilon_r\widetilde{E}(\omega) +  \frac{\Delta \epsilon_p \omega_p^2}{\omega_p^2+2j\omega\delta_p-\omega^2}\widetilde{E}(\omega)
  \end{split}
\end{equation}
\begin{equation}
  J_p(\omega) =  \frac{\Delta \epsilon_p \omega_p^2}{\omega_p^2+2j\omega\delta_p-\omega^2}\widetilde{E}(\omega)
\end{equation}
rearrange and IFT
\begin{equation}
  \omega_p^2J_p(t) + 2\delta_p\frac{\partial}{\partial t}J_p(t) + \frac{\partial^2}{\partial t^2}J_p(t) = \Delta\epsilon_p\omega_p^2\widetilde{E}(t)
\end{equation}
apply leapfrog scheme 
\begin{equation}
  \omega_p^2J_p^{n-1} + \delta_p\frac{J_p^n - J_p^{n-2}}{\Delta t} + \frac{J_p^n - 2 J_p^{n-1} + J_p^{n-2}}{\Delta t^2} = \Delta\epsilon_p\omega_p^2\widetilde{E}^{n-1}
\end{equation}
rearrange 
\begin{equation}
  J_p^n = 
  \frac{ 2-\omega_p^2\Delta t^2}{ 1+\delta_p\Delta t} J_p^{n-1} +
  \frac{ -1 + \delta_p\Delta t}{ 1+\delta_p\Delta t} J_p^{n-2} + 
  \frac{ \Delta\epsilon_p\omega_p^2\Delta t^2}{ 1+\delta_p\Delta t}\widetilde{E}^{n-1}
\end{equation}
as the result showed by general form of $C_1$, $C_2$, $C_3$.



\paragraph{\msjh Drude Model - ADE} Drude pole lacks the b coefficient when comparing to general form.
\begin{equation}
  \begin{split}
    \widetilde{D}(\omega) & = \epsilon_r^*(\omega)\widetilde{E}(\omega)\\
    & =  \epsilon_r\widetilde{E}(\omega) - \frac{\omega_p^2}{\omega(\omega-j\gamma_p)}\widetilde{E}(\omega)
  \end{split}
\end{equation}
Defining the last term as $J_p$, the phasor polarization current in physics,
\begin{equation}
  J_p(\omega) = -\frac{\omega_p^2}{\omega(\omega-j\gamma_p)}\widetilde{E}(\omega)
\end{equation}
Rearrange
\begin{equation}
  j\omega\gamma_pJ_p(\omega) - \omega^2J_p(\omega) = \omega_p^2\widetilde{E}(\omega)
\end{equation}
Performming inverse Fourier transformation
\begin{equation}
  \frac{\partial^2 J_p(t)}{\partial t^2} + \gamma_p \frac{\partial J_p(t)}{\partial t} = \omega_p^2\frac{\partial\widetilde{E}(t)}{\partial t}
\end{equation}
This is the ADE for $J_p$ for Drude Model. Then apply leapfrog scheme as general solution.
\begin{equation}
  \gamma_p\frac{J_p^n-J_p^{n-2}}{2\Delta t} + \frac{J_p^n - 2J_p^{n-1} + J_p^{n-2}}{(\Delta t)^2} = \omega_p^2\widetilde{E}^{n-1}
\end{equation}
rearrange 
\begin{equation}
  J_p^n = \frac{4}{2+ \gamma_p\Delta t} J_p^{n-1} + \frac{-2+\gamma_p\Delta t}{2+\gamma_p\Delta t}J_p^{n-2} + \frac{2\omega_p^2(\Delta t)^2}{2+\gamma_p\Delta t}\widetilde{E}^{n-1}
\end{equation}



All coefficients of different dispersion types are summarized here
\begin{center}
  \begin{tabular}[c]{|r|l|c|c|c|}
    \hline
    Dispersion Type & $\chi_p(\omega)$ & $C_1$ & $C_2$ & $C_3$ \\
    \hline
    Lorentz & $\frac{\Delta \epsilon_p \omega_p^2}{\sqrt{\omega_p^2 - \delta_p^2}}e^{-\delta_p t}\sin\left(\sqrt{\omega_p^2-\delta_p^2}\ t\right)$ & $\frac{2-\omega_p^2\Delta t^2}{1+\delta_p\Delta t}$ & $\frac{-1 + \delta_p\Delta t}{1+\delta_p\Delta t}$  & $\frac{\Delta\epsilon_p\omega_p^2\Delta t^2}{1+\delta_p\Delta t}$ \\
    \hline
    Drude & $\frac{\epsilon_0\omega_p^2}{j\omega\gamma_p-\omega^2}$ & $\frac{ 4}{ 2+\gamma_p\Delta t}$ & $\frac{ -2+\gamma_p\Delta t}{ 2+\gamma_p\Delta t}$ & $\frac{ 2\omega_p^2\Delta t^2}{ 2+\gamma_p \Delta t}$\\
    \hline
  \end{tabular}
\end{center}




\subsubsection{The Z Transform Method}
\paragraph{\msjh Debye Model - ZT}

\paragraph{\msjh Lorentz Model - ZT}

\paragraph{\msjh Drude Model - ZT}

\section{Modeling of Objects}
\label{sec:modeling}

\subsection{Center shift}
In the previous section, (\ref{eq:dispersive}) has provided clear view for modeling of objects: by marking $\epsilon_r$
and $a, b, c,d$ coefficients of $J_p$ for each point of the computational domain, it does. However, in consequence of
coordinate transformation done with (\ref{eq:coordinate_transform}), the center should be shifted for different
field components.

For example, there is a dielectric sphere with center $(i_0,j_0,k_0)$ and radius $r_0$. Due to the convenience, the
$E_x|_{i+\frac{1}{2},j,k}$ field is saved as $E_x[i,j,k]$ in computer memory. Corresponding
$\epsilon_{rx}|_{i+\frac{1}{2},j,k}$ or $\epsilon_{rx}[i,j,k]$ have to be marked with the sphere with the center $(i_0-1/2,
j_0, k_0)$ and radius $r_0$ in the coordinate in computer memory. That is, center of $\epsilon_{rx}$ has a shift
quantity $(-1/2,0,0)$.

The rest five field components should also be shifted as 
\begin{displaymath}
  E_y \rightarrow (0,-1/2,0)
\end{displaymath}
\begin{displaymath}
  E_z \rightarrow (0,0,-1/2)
\end{displaymath}
\begin{displaymath}
  H_x \rightarrow (0,-1/2,-1/2)
\end{displaymath}
\begin{displaymath}
  H_y \rightarrow (-1/2,0,-1/2)
\end{displaymath}
\begin{displaymath}
  H_z \rightarrow (-1/2,-1/2,0)
\end{displaymath}

\subsection{Modeling scheme}
Another issue of modeling is about the curved surface. The finite-difference strategy makes the surface to be a shape of
sawteeth in the original Stair-Case scheme: it is a flip-flop view on dispersive coefficents. The high reflection and
non-linear effect on the surface increases the error in the stair-case approximation. Index-average scheme is a common
impovement to stair-case by assigning average coefficents to points at the edge of objects through the proportion. The
Conformal scheme is another solution proposed by [\textit{Mohzmmadi et al}, 2009]. By importing a fourth order
time-stepping scheme, it performs better that the index-average.


%% \section{Transmittance Spectrum}
\subsection{Fourier Transform}
The implement of Discrete Fourier Transform\\
The implement of Fast Fourier Transform

\clearpage
\begin{center}
  \includegraphics[scale=0.5]{images/yee-grid.jpg}\\
  Fig 2.1
\end{center}

\begin{center}
\includegraphics[scale=0.5]{images/tfsf.jpg}\\
Fig 2.2
\end{center}

\begin{center}
\includegraphics[scale=0.5]{images/pbc.jpg}\\
Fig 2.3
\end{center}



\chapter{The yaFDTD Framework}
\section{Yet Another FDTD Framework}

Don't reinventing the wheels. This is a famous idiom handed down from the era of industrial society. In software
engineering, there is also a common derivation of the concept: Don't Repeat Yourself (DRY). It shows understanding how
to reuse existed codes to accelerate new projects is just like earning a silver bullet for software developing. That is
why the history of software devloping is also the history of seeking better ways to reuse softwares. A function as a
resuable unit became well-known after the Structural Programming languages, such as Pascal, Ada and C, had become the
mainstream. Not a long time after that, Object-Oriented Programming (OOP) languages took over the world with Class and
Object, the more general reusable units containing functions and data structures. [Sebesta, 2007]

OOP encourages developers grouping data structures under a namespace where it's not directly accessible by the rest of
the program and bundling relative functions to form classes. By instancing a class, objects are created to contain
different data but have identical data structures and behaviors. OOP promised a vision: through finding pertinent
objects and factoring them into classes at right granularity, your program would be able to address future requirements
without change too much code. [Gamma et al, 1994] Projects are also speeded up by spliting workforces for each
independent piece of a program. The set of cooperating classes making up a reusable design can form a framewok for
targeted class of software. With a framework, creating a particular application is just creating some specific
subclasses of abstract class from the framework. Actually, the road to fit the prerequisite is not flat.

The hard part about designing a framework is decomposing a system into objects due to many factor: encapsulation,
granularity, flexibility, evolution, reusability and so on [Gamma et al, 1994]. A framework designer has to gambles that
one architecture will work for all applications in the domain so that a framework should be as flexible and extensible
as possible. Loose coupling is another imperative issue for preventing major repercussion in applications when doing a
minor change to the framework. Facing challenges one after another, few researchers of FDTD are willing to spend time on
proposing a design of FDTD framework of which they can take advantage.

Fortunately, Components of FDTD simulator has given us a native prototype of architecture. Every component can just
corresponds to a class. The emergent problem is they have some functionalities overlapping for simple or complex
situations respectively. How to design a extensible way making complex components can be adopted when needed becomes the
main delimma. This chapter is aimed at proposing a experimental framework of FDTD. Whole project named Yet Another FDTD
(yaFDTD) framework was released under GPL v2 license and hosted at
\texttt{\url{http://github.com/shelling/yafdtd}}. Even it's a implementation in Python programming language, the
concepts about the split of data structures and classes can be transplanted into other languages easily.

This experimental framework has some goals here. The first is to assemle components easily. We may hope components can
be convened like following codes in a 2D simulation.
\begin{code}
  plane = PBC(plane)
  plane = UPML(plane)
  plane = TFSF(plane)
\end{code}
In this way, a usable plane are standby in a few lines. All field variables are hidden in the object
\texttt{plane}. Unnecessary components can be dropped without marking as commnets along with the variables used by these
components.

Next, configuration should be done through calling the interfaces to wrap the logic inside the components. Following are
some possible usages of the interfaces. Configuring through calling interfaces helps developers thinking the arrangement
in abstract, reducing possibility to make errors, and starting simulations soon.
\begin{code}
  plane.pml(
    x = False,
    y = True
    thick = 20
  )
  plane.pbc(
    x = True,
    y = False
  )
  plane.tfsf(
    x = 0,
    y = 20
  )
\end{code}
Finally, to allow developers extend this framework with arbitrary compoments for complex situations, it needs a class to
play this role to connect components together. A new Design Pattern named Once Decorator is proposed here due to the
fact there is no proper pattern recorded in popular lists.

\section{Once Decoractor}
The Decoractor Pattern is a well-known Design Pattern listed in the bestseller of Gang of Four, \textit{Design Patterns:
  Elements of Reusable Object-Oriented Software}. The book depicts that the purpose of the Decorator Pattern is
attaching additional responsibilities to an object dynamically. The role of Decorators in this pattern provides a
flexible alternative to subclassing for extending functionality.

Similar name implies similar behaviors. The Once Decorator Pattern is the mixing of Singleton Pattern and Dynamic
Inheritance, which is acting like the Decorator Pattern but actually not. The most remarkable difference is the
Decorator Pattern delegates all method calls to what it decorates, i.e. all subclasses should provide the same
interfaces as main component when the Once Decorator inherits the main component dynamically to retrieve the interfaces
from what it decorates when initializing. Dynamic Inheritance also allows subclasses overwriting interfaces to adapt
more complex situations as well as attaching additional interfaces without providing in every related class. These are
advantages of the Once Decorator Pattern.

Dynamic Inheritance induces another problems: once two instances perform inheritance, the latter would change the
behaviors of the former. Because a FDTD simulaion needs only one main component, Singleton Pattern acts inside of the
Once Decorator for preventing modification of inheritance on identical subclass. Moreover, all subclasses save the
singleton instance in main component to ensure one and only one singleton instance can be allowed to join the
calculation.

The architecture of Once Decorator is also similar to Decorator as shown in Fig 3.1. The magic is hidden in constructor
of Decorator. Assuming the main component of two dimension is named as \texttt{Plane}, the Decorator can be implemented
as following.
\begin{code}
  class PlaneDecorator(Plane):
    def __new__(clz):
        if not clz.singleton:
            clz.singleton = object.__new__(clz)
            Plane.singleton = singleton
        return clz.singleton

    def __init__(self, orig):
        self.__dict__ = orig.__dict__
        if orig.__class__ != Plane:
            self.__class__.__bases__ = (orig.__class__,)
        return None
\end{code}
The concrete decorator subclasses would inherit the abstract decorator class and invoke the constructor of the abstract
decorator class when initializing. The constructor does two things: exporting all instance variables to a new instance
created from any concrete decorator and making the concrete decorator inheriting the another concrete decorator from the
original instance.

The two actions save the instance variables and make later decorators can overwrite interfaces of prior decorators
through inheritance in concise relation.

\section{Freespace Component}
The freespace component serves as main component in the architecture of the Once Decorator Pattern. It stores fields
variables (arrays for storing $D, E, B, H$) as instance variables and provides the interfaces to update fields by the
rules in freespace. The fields variables are also utilized by concrete decorators which inherit the main component. And
about interfaces, there is a big point here. Comparing the Eq.\ref{eq:pmldx} and Eq.\ref{eq:dx3d}, the curl parts are
are not changed. Apparently it is necessary to provide interfaces for these reusable parts. The $curl\_ex$ and
$curl\_hx$ is a example here.
\begin{code}
    def curl_ex(self):
        res = numpy.zeros(self.shape)
        (x,y,z) = self.shape
        for i in range(x):
            for j in range(y-1):
                for k in range(z-1):
                    res[i,j,k] = self.ezfield[i,j+1,k] - self.ezfield[i,j,k] 
                               - self.eyfield[i,j,k+1] + self.eyfield[i,j,k]
        return res

    def curl_hx(self):
        res = numpy.zeros(self.shape)
        (x,y,z) = self.shape
        for i in range(x):
            for j in range(y):
                for k in range(z):
                    res[i,j,k] = self.hzfield[i,j,k] - self.hzfield[i,j-1,k] 
                               - self.hyfield[i,j,k] + self.hyfield[i,j,k-1]
        return res
\end{code}
The return value is a 2D array saving the values of curl for each point. with thees interfaces, Eq.\ref{eq:dx3d} and
Eq.\ref{eq:bx3d} appear a elegant view in source code.
\begin{code}
    def update_dfield(self):
        self.dxfield += 0.5 * self.curl_hx()
        self.dyfield += 0.5 * self.curl_hy()
        self.dzfield += 0.5 * self.curl_hz()
        return self
    def update_bfield(self):
        self.bxfield -= 0.5 * self.curl_ex()
        self.byfield -= 0.5 * self.curl_ey()
        self.bzfield -= 0.5 * self.curl_ez()
        return self
\end{code}
And the E field and H field are just copied from D and B in freespace. Similar handlers for material are left to
implement in Dispersion concrete decorator to overwrite these two interfaces.
\begin{code}
    def update_efield(self):
        self.exfield = self.dxfield.copy()
        self.eyfield = self.dyfield.copy()
        self.ezfield = self.dzfield.copy()
        return self

    def update_hfield(self):
        self.hxfield = self.bxfield.copy()
        self.hyfield = self.byfield.copy()
        self.hzfield = self.bzfield.copy()
        return self
\end{code}



\section{PBC Component}
To make PBC component work, surrounded edges must be appended as instance variables and participate the calculation of
the curl parts. For minimizing trivial codes, this parts can be implenented in Freespace component. For example, the
$curl\_ex$ in 2D may invoke edges as following.
\begin{code}
    def curl_ex(self):
        res = numpy.zeros(self.shape)
        (x,y) = self.shape
        for i in range(x):
            for j in range(y-1):
                res[i,j] = self.ezfield[i,j+1] - self.ezfield[i,j]
        for i in range(x):
            res[i,y-1] = self.ezedgey[i] - self.ezfield[i,y-1]  # invoke edges
        return res
\end{code}
The duty of PBC component now remains locating the the PBC and updating the edge components. Instance variables
\texttt{pbcx} and \texttt{pbcy} are used to verify which edges should serve and manipulated in the interface
\texttt{pbc}.
\begin{code}
    def pbc(self, x=True, y=True):
        self.pbcx = x
        self.pbcy = y
        return self

    def update_epbc(self):
        if self.pbcx:
            self.ezedgex = self.ezfield[0,:] # pbc x, TM
            self.eyedgex = self.eyfield[0,:] # pbc x, TE
        if self.pbcy:
            self.ezedgey = self.ezfield[:,0] # pbc y, TM
            self.exedgey = self.exfield[:,0] # pbc y, TE
        return self

    def update_hpbc(self):
        xmax = self.shape[0]-1
        ymax = self.shape[1]-1
        if self.pbcx:
            self.hyedgex = self.hyfield[xmax,:] # pbc x, TM
            self.hzedgex = self.hzfield[xmax,:] # pbc x, TE
        if self.pbcy:
            self.hxedgey = self.hxfield[:,ymax] # pbc y, TM
            self.hzedgey = self.hzfield[:,ymax] # pbc y, TE
        return self
\end{code}


\section{UPML Component}
The hard part of UPML implementation is the summation term. The solution is deploying auxiliary variables to sum current
values of curl and participate updating. In the case of 2D, Eq.\ref{eq:pmldx}, Eq.\ref{eq:pmldy}, and Eq.\ref{eq:pmldz}
was implemented as following.
\begin{code}
    def update_dfield(self):
        self.idx += self.curl_hx()
        self.idy += self.curl_hy()
        self.dxfield = self.j3 * self.dxfield 
                     + self.j2 * 0.5 * ( self.curl_hx() + self.i1 * self.idx )
        self.dyfield = self.i3 * self.dyfield 
                     + self.i2 * 0.5 * ( self.curl_hy() + self.j1 * self.idy )
        self.dzfield = self.i3 * self.j3 * self.dzfield 
                     + self.i2 * self.j2 * 0.5 * self.curl_hz()
        return self
\end{code}
The summation of $curl\_hx$ and $curl\_hy$ are saved in \texttt{idx} and \texttt{idy}. The variables defined in
Eq.\ref{eq:v3}, Eq.\ref{eq:v2}, Eq.\ref{eq:v1} are saved as arrays \texttt{i1}, \texttt{i2}, \texttt{i3}, \texttt{j1},
\texttt{j2}, \texttt{j3} which are mapping coefficients to every point in computational domain. In Sullivan's example,
he used one dimension arrays to save memory here. However, it's more intuitive and programmable using array having
dimension the same as fields component.

The multiplication operator of numpy ndarray do multiplication for each pair having the same index like inner
product. That makes the code clean as in mathematics. In any programming language allowing operator overloading,
including Perl, Ruby, C++ and Java, it is a better alternative using a function as middleware to enclose for loops
rather than writing for loop here directly.

\section{TF/SF Component}
Concepts of TF/SF Component are easy to grasp. However, the theory left a big blank canvas for implementation because
there are eight types in two dimension and twelve types in three dimension. A intuitive implementation is categorizing
by incident direction to be XTFSF, YTFSF, and ZTFSF (only in 3D). An XTFSF can be TM or TE mode and each mode can be
rectangle layout or infinite layout working with PBC (Fig.???), YTFSF and ZTFSF too. By example of YTFSF in two
dimension, the constructor appends two auxiliary 1D source as well as masks boundary of TFSF in two direction. 
\begin{code}
    def __init__(self, orig):
        super(YTFSFPlane, self).__init__(orig)
        self.tminc = String(self.shape[1])
        self.teinc = String(self.shape[1])
        self.xtfsf = [10, self.shape[0]-10]
        self.ytfsf = [10, self.shape[1]-10]
        return None
\end{code}
The rest is two more interfaces to insert D and B at TFSF boundary. To simplify, TE and TM mode are inserted at the same
time. If only one mode is needed, no updating on the unnecessary auxiliary 1D source in main script could disable
it. This design leads to more overhead but creates elegant depiction. The next big trick is to examinate whether the
boundaries in x direction are \texttt{[None, None]}. If the assertion is true, regard TFSF itself as a layout infinite
at x direction. Otherwise, it's a rectangle layout should update boundaries in x direction.
\begin{code}
    def update_dtfsf(self):
        if self.xtfsf == [None, None]:
            self.dzfield[:, self.ytfsf[0]]
            += 0.5 * self.tminc.hfield[self.ytfsf[0]-1]
            self.dzfield[:, self.ytfsf[1]] 
            -= 0.5 * self.tminc.hfield[self.ytfsf[1]]
            self.dxfield[:, self.ytfsf[0]]  
            -= 0.5 * self.teinc.hfield[self.ytfsf[0]-1]
            self.dxfield[:, self.ytfsf[1]]  
            += 0.5 * self.teinc.hfield[self.ytfsf[1]]
        else:
            # y edge
            self.dzfield[self.xtfsf[0]:self.xtfsf[1]+1, self.ytfsf[0]]  
            += 0.5 * self.tminc.hfield[self.ytfsf[0]-1]
            self.dzfield[self.xtfsf[0]:self.xtfsf[1]+1, self.ytfsf[1]]  
            -= 0.5 * self.tminc.hfield[self.ytfsf[1]]
            self.dxfield[self.xtfsf[0]:self.xtfsf[1]+1, self.ytfsf[0]]  
            -= 0.5 * self.teinc.hfield[self.ytfsf[0]-1]
            self.dxfield[self.xtfsf[0]:self.xtfsf[1]+1, self.ytfsf[1]]  
            += 0.5 * self.teinc.hfield[self.ytfsf[1]]

            # x edge
            self.dyfield[self.xtfsf[0],   self.ytfsf[0]:self.ytfsf[1]+1] 
            += 0.5 * self.teinc.hfield[self.ytfsf[0]:self.ytfsf[1]+1]
            self.dyfield[self.xtfsf[1]+1, self.ytfsf[0]:self.ytfsf[1]+1] 
            -= 0.5 * self.teinc.hfield[self.ytfsf[0]:self.ytfsf[1]+1]
        return self


    def update_btfsf(self):
        if self.xtfsf == [None, None]:
            # y edge
            self.bxfield[:, self.ytfsf[0]-1]
            += 0.5 * self.tminc.efield[self.ytfsf[0]]
            self.bxfield[:, self.ytfsf[1]]  
            -= 0.5 * self.tminc.efield[self.ytfsf[1]]
            self.bzfield[:, self.ytfsf[0]-1]
            += 0.5 * self.teinc.efield[self.ytfsf[0]]
            self.bzfield[:, self.ytfsf[1]]  
            -= 0.5 * self.teinc.efield[self.ytfsf[1]]
        else:
            # y edge
            self.bxfield[self.xtfsf[0]:self.xtfsf[1]+1, self.ytfsf[0]-1] 
            += 0.5 * self.tminc.efield[self.ytfsf[0]]
            self.bxfield[self.xtfsf[0]:self.xtfsf[1]+1, self.ytfsf[1]]   
            -= 0.5 * self.tminc.efield[self.ytfsf[1]]
            self.bzfield[self.xtfsf[0]:self.xtfsf[1]+1, self.ytfsf[0]-1] 
            += 0.5 * self.teinc.efield[self.ytfsf[0]]
            self.bzfield[self.xtfsf[0]:self.xtfsf[1]+1, self.ytfsf[1]]   
            -= 0.5 * self.teinc.efield[self.ytfsf[1]]

            # x edge
            self.byfield[self.xtfsf[0]-1, self.ytfsf[0]:self.ytfsf[1]+1] 
            -= 0.5 * self.tminc.efield[self.ytfsf[0]:self.ytfsf[1]+1]
            self.byfield[self.xtfsf[1],   self.ytfsf[0]:self.ytfsf[1]+1]
            += 0.5 * self.tminc.efield[self.ytfsf[0]:self.ytfsf[1]+1]
        return self
\end{code}

\section{Dispersion Component}
Owing to the demand of supports to multiple poles in (\ref{eq:dispersive}), the dispersion component has to be splited
into two part. The $\epsilon_r$ are saved in main decorated component to participate calculation after retrieving the
summation of $J_p$. Oppositely, $J_p$ saving $a,b,c,d$ coefficients is implemented as a independent class not involving
in Once Decorator inheritance architecture.

One more decorator is needed to append $\epsilon_r$ to main component and overwrite the \texttt{update\_efield} method
as following. 
\begin{code}
    def update_efield(self, *polar):
        self.exfield  = self.dxfield.copy()
        self.eyfield  = self.dyfield.copy()
        self.ezfield  = self.dzfield.copy()
        for p in polar:
            self.exfield -= p.x
            self.eyfield -= p.y
            self.ezfield -= p.z
        self.exfield /= self.epsilon_rx
        self.eyfield /= self.epsilon_ry
        self.ezfield /= self.epsilon_rz
        return self
\end{code}
The variable \texttt{*polar} with star declares it can receive arbitrary arguments and act as an array. In the middle
for loop, $E$ field uses the values copied from $D$ field to subtract the $J_p$. Then $\epsilon_r$ divides $E - \sum
J_p$ in latter three lines. The reason each E field component has its own appointment of $\epsilon_{r}$ has been
explained in Section \ref{sec:modeling}.

The second part of dispersion component is $J_p$. The constructor of $J_p$ does two things. The first is calculating
$C1$, $C2$, and $C3$ from $a,b,c,d$ coefficients. The other is appending instance variables for saving polarized
displacement of recent three steps. From (\ref{eq:polarized_displacement}), the method \texttt{update} was implemented
as following. The strategy to distinguish material is slightly different from modeling $\epsilon_r$. Logically, one pole
saves one set of coefficients, so that it's not necessary saving coefficients as arrays. Masks are a better design here.
\begin{code}
    def update(self, plane):
        self.xp2 = self.xp
        self.yp2 = self.yp
        self.zp2 = self.zp
        self.xp = self.x
        self.yp = self.y
        self.zp = self.z
        self.x = self.c1*self.xp + self.c2*self.xp2 + self.c3*plane.exfield*self.maskx
        self.y = self.c1*self.yp + self.c2*self.yp2 + self.c3*plane.eyfield*self.masky
        self.z = self.c1*self.zp + self.c2*self.zp2 + self.c3*plane.ezfield*self.maskz
        return self
\end{code}
In the main FDTD loop, the two parts constructed above collaborate after retrieving current $D$ field. All poles update
themselves with previous $E$ field and $E$ field finds its current values from all poles.
\begin{code}
  plane.update_dfield()

  # three poles are defined 
  drude_pole.update(plane)
  lorentz_pole1.update(plane)
  lorentz_pole2.update(plane)

  # allow arbitrary poles 
  plane.update_efield(drude_pole, lorentz_pole1, lorentz_pole2) 
\end{code}

\section{MPI Edge Component}
Because there is no possibility to overcome the time complexity $O(3)$ in 3D and $O(2)$ in 2D For
investigating some huge structure, parallelization is a imperative support in FDTD. Recently famous Tactics to
parallelization include Cuda, Pthreads API, OpenMP API, and MPI standard. For well integrating with numpy used in
yaFDTD, mpi4py, the brother package implementing MPI Standard, was pick out here. 

In the slant of MPI, computational domain of FDTD can be splited into many fragments on different memory block and
manipulated by different CPU. Because data fragments are independent with others, at edges of computational domain, it
needs to exchange data. APIs of MPI serve here.

Invoking APIs of MPI in main script is a quick and dirty solution in vary examples and tutorials. However, it's possible
wrapping MPI actions in a class as a Decorator in our architecture to hide ugly invoking of MPI. 
\begin{code}
    class MPIXEdgePlane(PlaneDecorator):
        def __init__(self, orig):
            super(MPIXEdgePlane, self).__init__(orig)
            return None

        def mpi(self, comm):
            self.mpi_comm = comm
            self.mpi_size = comm.Get_size
            self.mpi_rank = comm.Get_rank
            self.mpi_prev = self.mpi_rank-1
            self.mpi_next = self.mpi_rank+1
            return self

        def send_hpbc(self):
            self.mpi_comm.send(self.hyfield[self.shape[0]-1], dest=self.mpi_next, tag=0)
            return self

        def send_epbc(self):
            self.mpi_comm.send(self.ezfield[0], dest=self.mpi_prev, tag=1)
            return self
\end{code}

\clearpage
\begin{center}
\includegraphics[scale=0.5]{images/once-decorator.jpg}\\
Fig 3.1
\end{center}
\begin{center}
\includegraphics[scale=0.5]{images/tfsf-pbc.jpg}\\
Fig 3.2
\end{center}



\chapter{Applications about Surface Plasmon Structures}
\section{examination}
The surface plasmonic structures become popular in this two decades owing to plasmonic waveguides admit optical waves
transmitting in subwavelength structures having metal-dielectric interfaces specially in nano-scale. The conduction
electrons oscillate in the longitudinal direction and the Maxwell's equations show the solution is electromagnetic
fields confined near the interface of metal.

Single silver rod may one of the most simple metal-dielectric interface. Phasor distribution of single silver rod is
easy to obtain via frequency domain method such as Multiple-Scattering method. By performing Fourier transform on
timeline of FDTD result reaching steady state, a good approximation is expected. To ensure the yaFDTD framework to carry
out a credible result, comparison result between exact solution and FDTD is a good index.

The example here is a nano cylinder with radius 25 nm. The cylinder material is silver simulated using Drude Model
with following parameters: $\epsilon_r = 8.926$, $\omega_p = 9.39 \times 10^{15}$ rad/s, and $\gamma_p = 3.14 \times
10^{13}$ rad/s.  There would be one more thing remarkable before starting simulation to acquire better result. Although
the UPML has excellent performance for absorbing propagatin waves; however, the evanescent waves still grows field
intensity up inside the UPML. To prevent the reflection from PML interact with the scatters. It need extra space between
PML and scatters. Here we set the grid size to be 1 nm and make whole space being 500 nm $\times$ 500 nm.

Final arrangement is shown in Fig 4.1. Plane wave in $\mathrm{TE_z}$ mode is impinged in x-direction with wavelength
347.5 nm. Fig 4.2 shows the Re\{$H_z$\} profiles of phasor distribution when reaching steady state after 100 periods of
plane wave. Fig 4.3 shows the exact solution of Re\{$H_z$\} profiles. and numerical error shown in Fig 4.4 is calculated
by subtracting the result from FDTD with the result from exact solution. The inaccuracy is small than 0.1 with the
simplest Stair-Case scheme. In general, it good enough to apply the framework on more complex cases.

\section{Silver Rods Open Cavity}
Using nano-scale structure to design devices for confining or guiding light may be the major challenges in optical
researches. When the surface plasmon resonance provides dramatic local-field enhancement, a series of silver rods can
has the ability to guide light in finite length through near-field coupling. In additions, Confinement of light can be
also obseved by arrange silver rods to enclose space. Local-field enhancement plays other important role like gain media
in laser that mean the enclosed space can form an open cavity. Here we study the phasor characteristics of a silver rods
open cavity proposed recently.

The open cavity forms of three silver nanocylinder pairs. Fig 4.5 shows the geometry configuration of three pairs of
silver nanocylinders. The interparticle distance is $d$. The interpair distance is $d_p$. The radius of nanocylinders is
$r$. The three pair structures are illuminated with a plane wave of $\mathrm{TE_z}$ polarized mode. Drude Model is used
again in this simulation but parameters is set using the experimental data from Palik. Computational region is set to be
$500 \times 500 nm^2$ and 20 $nm$ and UPML is surrounded outside whole region.

Due to the fact the optical response of nanoparticles would act as dipoles when the size of nanoparticles are far
smaller than wavelength of impinged plane wave. The interaction between nanoparticles is similar to the interaction
between dipoles. The higher multipole behavior becomes apparent with increasing size of nanoparticles. For the reason,
the radius of nanocylinders should change the near-field optical response of the open cavity.

Previous study shows the maximum intensity in chamber occrurs when the ratio of wavelength and nanocylinder radius is
around 11 and 12. Two remarkable cases are also shown in phasor distribution. The first case impinges plane wave in
wavelength $460 nm$ and the six nanocylinders are set as $r = 36 nm$, $d = d_p = 20 nm$, which the ratio is $\lambda /r
= 12.7$. The second case impinges plane wave in wavelength $650 nm$ and nanocylinders are set as $r = 58 nm$, $d = d_p =
20 nm$, which the ratio is $\lambda /r = 11.2$ . For futher testing of performance of the yaFDTD framework we directly
perform this two cases again because the numerical simulation error usually arises as the field intensity is enhanced
severely.

Fig 4.6 and 4.7 shows the steady state total E field phasor distribution for the first and second case mentioned above
respectively. Comparing with the result shown in previous study (Fig 4.8), the phasor distribution of yaFDTD can fit it
good (Fig 4.8).

Fig 4.9 and 4.10 also shown the the partial E field phasor distribution for $\mathrm{E_x}$ and $\mathrm{E_y}$ in the
first case. It can be observed the $\mathrm{E_x}$ component is dominant in the gaps between pairs and the $\mathrm{E_y}$
component is dominant inside the gaps of each pairs. This shows that we can partially manipulate the $\mathrm{E_x}$ and
$\mathrm{E_y}$ by change the distance in pairs $d$ and distance between pairs $d_p$ to reduce the interaction of
dipoles.  For example, when we change the $d_p$ to be $40 nm$ in the first case, the $\mathrm{E_x}$ would almost
disappear.

\clearpage
\begin{center}
\includegraphics[scale=0.5]{images/single-rod-config.jpg}\\
Fig 4.1
configuration of single silver rods
\end{center}
\begin{center}
\includegraphics[scale=0.8]{images/phasor-exam-fdtd.png}\\
Fig 4.2
phasor distribution of $mathrm{H_z}$, calculated by yaFDTD
\end{center}
\begin{center}
\includegraphics[scale=0.8]{images/phasor-exam-exact.png}\\
Fig 4.3
phasor distribution of $mathrm{H_z}$, exact solution
\end{center}
\begin{center}
\includegraphics[scale=0.8]{images/errors.png}\\
Fig 4.4
numerical error between yaFDTD and exact solution
\end{center}
\begin{center}
\includegraphics[scale=0.5]{images/open-cavity-config.jpg}\\
Fig 4.5
configuration of three pairs of silver nanocylinders
\end{center}
\begin{center}
\includegraphics[scale=0.8]{images/etotal.png}\\
Fig 4.6
phasor distribution of $|E|$ for the case $\lambda = 460nm, r = 36nm, d = d_p = 20nm$, calculated by yaFDTD
\end{center}
\begin{center}
\includegraphics[scale=0.1]{images/r36.png}\\
Fig 4.8
phasor distribution of $|E|$ for the case $\lambda = 460nm, r = 36nm, d = d_p = 20nm$, in previous study
\end{center}
\begin{center}
\includegraphics[scale=0.8]{images/ex.png}\\
Fig 4.9
phasor distribution of $|E_x|$ for the case $\lambda = 460nm, r = 36nm, d = d_p = 20nm$, calculated by yaFDTD
\end{center}
\begin{center}
\includegraphics[scale=0.8]{images/ey.png}\\
Fig 4.10
phasor distribution of $|E_y|$ for the case $\lambda = 460nm, r = 36nm, d = d_p = 20nm$, calculated by yaFDTD
\end{center}




\chapter{Conclusion}
In this research, Maxwell's equations are reinvestigated to be a more symmetrical forms for being programmable easily
And the possibility to refactor Maxwell's equtions into a bunch of fragments is also be studied. By importing the
concepts of Structural Programming and Object-Oriented Programming, a blueprint of forming a FDTD simulator by assemble
well-prepared components appears before our eyes. It admits the newbies starting simulations without understand details
of each components. For design five common components in FDTD method: Freespace, Perfectly Matched Layers, Dispersive
Materials, Total Field / Scatter Field plane wave source, and Bloch Periodic Boundary Conditions, fragments of Maxwell's
equations are spreaded into different classes. For assembling all components well, a Design Pattern, Once Decorator, is
proposed. By the abilities of this Design Pattern, It's possible to append or exchange new components into future
simulation without considerable changes in other components have been written. Existed five components is also gathered
as a tiny framework.

The structures formed of silver nanocylinders are also studied. In single silver rods case, the phasor distribution
calculated by FDTD is compared with exact solution. This case shows that the accuracy of the tiny framework is
credible. An open cavity forms of three silver nanocylinders pairs are also studied. Phasor calculation shows the open
cavity has the maximum resonance when the ratio of wevelength of impinging plane wave and the radius of nanocylinders is
around 11 and 12. Partially phasor distribution of $|\mathrm{E_x}|$ and $|\mathrm{E_y}|$ also shows the details of
dipole interaction. It implies increasing distance between pairs $|\mathrm{E_y}|$ would be weaken and $|\mathrm{E_x}|$
has the same change when distance inside each pairs increase due to less coupling of near-field.




%% \appendix
%% \chapter{Comparing ADE, RC and ZT}
%% \paragraph{{\msjh Debye Model - RC}}
\begin{displaymath}
  \epsilon_r^*(\omega) = \epsilon_r + \frac{\sigma}{j\omega \epsilon_0} + \frac{\Delta \epsilon_p}{1+j\omega \tau_p}
\end{displaymath}
\begin{displaymath}
  \begin{split}
    \widetilde{D}(\omega) & = \epsilon_r^*(\omega)\widetilde{E}(\omega)\\
    & = \epsilon_r\widetilde{E}(\omega) + \frac{\sigma}{j\omega\epsilon_0}\widetilde{E}(\omega) + \frac{\Delta \epsilon_p}{1+j\omega \tau_p}\widetilde{E}(\omega)
  \end{split}
\end{displaymath}
\begin{displaymath}
  \widetilde{D}(t) = \epsilon_r\widetilde{E}(t) + \frac{\sigma}{\epsilon_0}\int_0^t\widetilde{E}(t')dt' + \int_0^t\frac{\Delta \epsilon_p}{\tau_p}e^{-\frac{t'-t}{\tau_p}}\widetilde{E}(t')dt'
\end{displaymath}
\begin{equation}
  \widetilde{D}^n = \epsilon_r\widetilde{E}^n + \frac{\sigma}{\epsilon_0}\Delta t\sum_{i=0}^{n}\widetilde{E}^i + \frac{\Delta \epsilon_p}{\tau_p}\Delta t \sum_{i=0}^{n} e^{-\frac{n-i}{\tau_p}\Delta t}\widetilde{E}^i
\end{equation}
The second term, dissipated displacement, $I^n$ , the same as simple conductive media.
The third term, phasor polarization displacement, $J_p^n$. $J_p^n$ is slightly complex in recursive form.
\begin{equation}
  \begin{split}
    J_p^n & = \frac{\Delta\epsilon_p}{\tau_p}\Delta t \sum_{i=0}^ne^{-\frac{n-i}{\tau_p}\Delta t}\widetilde{E}^i\\
    & = \frac{\Delta\epsilon_p}{\tau_p}\Delta t \left(\widetilde{E}^n + \sum_{i=0}^{n-1}e^{-\frac{n-i}{\tau_p}\Delta t}\widetilde{E}^i\right)
  \end{split}
\end{equation}
and 
\begin{equation}
  \begin{split}
    J_p^{n-1} & = \frac{\Delta\epsilon_p}{\tau_p}\Delta t \sum_{i=0}^{n-1}e^{-\frac{n-1-i}{\tau_p}\Delta t}\widetilde{E}^i\\
    & = \frac{\Delta\epsilon_p}{\tau_p}\Delta t \left( e^{\frac{\Delta t}{\tau_p}} \right) \sum_{i=0}^{n-1}e^{-\frac{n-i}{\tau_p}\Delta t}\widetilde{E}^i
  \end{split}
\end{equation}
substituted into $J_p^n$
\begin{equation}
  J_p^n = \frac{\Delta\epsilon_p}{\tau_p}\Delta t\cdot\widetilde{E}^n + e^{-\frac{\Delta t}{\tau_p}} J_p^{n-1}
\end{equation}
\begin{equation}
  \begin{split}
    \widetilde{D}^n & = \epsilon_r\widetilde{E}^n + \left[\frac{\sigma}{\epsilon_0}\Delta t\cdot\widetilde{E}^n + I^{n-1}\right] + \left[\frac{\Delta \epsilon_p}{\tau_p}\Delta t\cdot\widetilde{E}^n + e^{-\frac{\Delta t}{\tau_p}} J_p^{n-1}\right]\\
    & = \left(\epsilon_r + \frac{\sigma}{\epsilon_0}\Delta t + \frac{\Delta\epsilon_p}{\tau_p}\Delta t\right)\widetilde{E}^n + I^{n-1} + e^{-\frac{\Delta t}{\tau_p}} J_p^{n-1}
  \end{split}
\end{equation}
Finally the discrete constitute relations in time domain becomes
\begin{gather}
  \begin{array}{@{}l@{}}
    \widetilde{E}^n =  \frac{\displaystyle \widetilde{D}^n - I^{n-1} - e^{-\frac{\Delta t}{\tau_p}}J_p^{n-1} }{\displaystyle \epsilon_r + \frac{\sigma}{\epsilon_0}\Delta t + \frac{\Delta \epsilon_p}{\tau_p} \Delta t}\\    
    I^n = \frac{\sigma}{\epsilon_0}\Delta t\cdot\widetilde{E}^n + I^{n-1}\\
    J_p^n = \frac{\Delta\epsilon_p}{\tau_p}\Delta t\cdot\widetilde{E}^n + e^{-\frac{\Delta t}{\tau_p}} J_p^{n-1}
  \end{array}
\end{gather}
implement
\begin{code}
  dx[k] = dx[k] + 0.5 * ( hy[k-1] - hy[k])
  ex[k] = ( dx[k] - i[k] - exp(-dt/tau_p) * j[k] ) 
  * ( epsilon_0 * epsilon_r[k] + sigma_e[k] * dt +  )
  i[k] = i[k] + sigma_e[k] 
  j[k] = j[k] + del
\end{code}



\paragraph{{\msjh Debye Model - ADE}}
Debye model is a fun case can be induced the same form of $J_p(t)$ as RC if the semi-implicit scheme was chose. However,
leapfrog scheme provide better accruacy. Just for verifying, result of apply semi-implicit scheme was also derived
here. In the real wrold implementation \textit{\uwave{yaFDTD}}, leapfrog was picked out.

Starting with constitute relation as before
\begin{equation}
  \begin{split}
    \widetilde{D}(\omega) & = \epsilon_r^*(\omega)\widetilde{E}(\omega)\\
    & = \epsilon_r\widetilde{E}(\omega) + \frac{\sigma}{j\omega\epsilon_0}\widetilde{E}(\omega) + \frac{\Delta \epsilon_p}{1+j\omega \tau_p}\widetilde{E}(\omega)\label{eq:debye_ade_start}
  \end{split}
\end{equation}
detach $J_p$
\begin{equation}
  J_p(\omega) = \frac{\Delta \epsilon_p}{1+j\omega \tau_p}\widetilde{E}(\omega)
\end{equation}
This is the ADE of Debye Model
\begin{equation}
  J_p(\omega) + j\omega\tau_{p}J_p(\omega) = \Delta\epsilon_p\widetilde{E}(\omega)
\end{equation}
performming IFT 
\begin{equation}
  J_p(t) + \tau_p\frac{\partial}{\partial t}J_p(t) = \Delta\epsilon_p\widetilde{E}(t)
\end{equation}
apply semi-implicit scheme
\begin{equation}
  \left( \frac{J_p^n - J_p^{n-1}}{2} \right) + \tau_p \left( \frac{J_p^n - J_p^{n-1}}{\Delta t}\right) = \Delta\epsilon_p\widetilde{E}^n
\end{equation}
Solving $J_p$
\begin{equation}
  J_p^n = \frac{\left(1-\frac{\Delta t}{2\tau_p}\right)}{\left(1+\frac{\Delta t}{2\tau_p}\right)}J_p^{n-1} 
  + \frac{\left(\frac{\Delta\epsilon_p}{\tau_p}\right)\Delta t}{\left(1+\frac{\Delta t}{2\tau_p}\right)}\widetilde{E}^n
\end{equation}
It should be noted
\begin{equation}
  \begin{array}{@{}lp{0.5cm}r@{}}
    \frac{\displaystyle1-\delta}{\displaystyle1+\delta} \cong e^{-2\delta} && if\ \delta \ll 1
  \end{array}
\end{equation}
\begin{equation}
  \frac{\left(1-\frac{\displaystyle\Delta t}{\displaystyle2\tau_p}\right)}{\left(1+\frac{\displaystyle\Delta t}{\displaystyle2\tau_p}\right)} \cong e^{-\frac{\Delta t}{\tau_p}}\quad  because\ 1 \gg \frac{\Delta t}{2\tau_p}
\end{equation}
That is 
\begin{equation}
  J_p^n \cong e^{-\frac{\Delta t}{\tau_p}}J^{n-1} + \frac{\Delta\epsilon_p}{\tau_p}\Delta t\cdot\widetilde{E}^n
\end{equation}
And by performming inverse Fourier transform on Eq.\ref{eq:debye_ade_start}
\begin{equation}
  \widetilde{D}(t) = \epsilon_r\widetilde{E}(t) + \frac{\sigma}{\epsilon_0} \int_0^t\widetilde{E}(t')dt' + J_p(t)
\end{equation}
\begin{equation}
  \begin{split}
    \widetilde{D}^n & = \epsilon_r\widetilde{E}^n + \frac{\sigma}{\epsilon_0}\Delta t\sum_{i=0}^n\widetilde{E}^i + J_p^n\\
    & = \epsilon_r\widetilde{E}^n + \frac{\sigma}{\epsilon_0}\Delta t\cdot\widetilde{E}^n + I^{n-1} + \frac{\Delta\epsilon_p}{\tau_p}\Delta t\cdot\widetilde{E}^n + e^{-\frac{\Delta t}{\tau_p}}J^{n-1}
  \end{split}
\end{equation}
The same result as Recursive Convolution Method.



\backmatter
\cleardoublepage
\addcontentsline{toc}{chapter}{Bibliography}
\begin{thebibliography}{999}
  \bibitem
  {taflove}
  Computational Electrodynamics: The Finite-Difference Time-Domain Method, 3/e, Allen Taflove et. al.

  \bibitem
  {cheng}
  Fields and Wave Electromagnetics, 2/e, David K. Cheng

  \bibitem
  {rao} 
  the electromagnetic wave, 6/e, Rao

  \bibitem
  {sebesta}
  Concepts of Programming Languages, 8/e, Robert W. Sebesta
\end{thebibliography}

%% \clearpage
\addcontentsline{toc}{chapter}{Index}
\printindex


\end{document}

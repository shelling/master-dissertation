\section{Motivations}
When observing the histories of the Finite-Difference Time-Domain (FDTD) method [Yee, 1966], the UNIX operating system, and the Lisp and
C programming languages which are ancestors of most other modern programming languages, they look like friends growing
together. The flourishing of UNIX promotes the whole computer industry and reduces the cost of numerical simulation. As the
main language in UNIX environment, C gathers many many advocates in campuses with widespread distribution of licneses of
UNIX. Comparing with Assembly Language family, C provides protability and clear code style of Structural Programming
which can also benefit the programs of FDTD simulations.

However, at present all of them become more mature and complex. UNIX and its derivations go beyond campuses and live in
fabrication industries, business corporates, and families. Advocates come from different fields requiring and appending
more abilities on this omnipotent operating system. Bunches of theorems of designing programming languages are developing
on the operating system of rebirth. One of them is concepts of Object-Oriented Programming (OOP) introduced in Simula
language of 1960s and reinvestigated in Smalltalk language of 1970s \cite[Sebesta 2008]{sebesta}. Soon after 1980s, OOP
becomes popular on UNIX system along with the implementation on Common Lisp Object System (CLOS) and the birth of C++
and Java in 1983 and 1995 respectively because of the possibility to devide a huge system into small independent
fragments for different programmers. Programmers may not know details of the parts done by others any more but programs
can work with correct invoking of interfaces. Unquestionably, it is one gaint leap for software developing.

The FDTD method also had huge advances. Varieties of Absorbing Boundary Conditions and transformation tools for scattering systems
were developed during 70s and 80s. Methods for arbitrary material and non-linear systems were also proposed in 90s. Along
with more complex theorems came the obscure formulas. Complications were caused by bundling too many theorems in one
equation and requiring numerous variables in procedural programming environment. Reviewing the evolution of Structural
Programming and Object-Oriented Programming, we know those two problems are targeted by these two concepts. That is, a
modern FDTD simulator should be designed with modern concepts of programming. Unfortunately, almost all books and
articles about the FDTD method focus on fundamental theorems and show their algorithms in primitive procedural style like Assembly
languages. Apparently, the skill out-of-date before 30 years ago is bad smell in code. [Kerievsky, 2004]

Indeed, some people notice the inevitability to reform the architecture of the FDTD method. A famous implementation, Meep
(\texttt{http://ab-initio.mit.edu/wiki/index.php/Meep}), developed at Massachusetts Institute of Technology, shows us the
advantages to develop components for reusing in high-level languages such as Scheme. However, lack of good Periodic
Boundary Conditions and Plane Wave Source Conditions makes it impossible to be used in our environment. No support to
OOP also limits the development of this package. For example, it may be insuperable to replace the uniaxial perfectly matched layer (UPML) [Sack et al, 1995] with the convolution perfectly matched layer (CPML) [Roden and Gedney, 2000] in Meep
without considerable changes. Based on problems listed above, Developing our own general-purpose FDTD simulator is worthy of
giving a try rather than stacking upon Meep. 

In this thesis, we propose a elegant organization to theorems of the FDTD method. By applying the concepts of refactoring in OOP
upon Maxwell's equations or Update Equations, formulas are presented in a bunch of small atoms which are easy to
statisfy the concept of Structural Programming. Through the cutting, the way to arrange formulas into different
components in OOP would also rise before our mind. Paradigm of OOP not only helps us assemble proper simulation
environment with handy components but also eliminates the obstacle to newbies of the FDTD method. UPML and CPML can be exchanged in
seconds and newcomers may be able to do a correct simulation without understanding how a Perfectly Matched Layer works.

\section{Yet Another FDTD Framework}

Don't reinventing the wheels. This is a famous idiom handed down from the era of industrial society. In software
engineering, there is also a common derivation of the concept: Don't Repeat Yourself (DRY). It shows understanding how
to reuse existed codes to accelerate new projects is just like earning a silver bullet for software developing. That is
why the history of software devloping is also the history of seeking better ways to reuse softwares. A function as a
resuable unit became well-known after the Structural Programming languages, such as Pascal, Ada and C, had become the
mainstream. Not a long time after that, Object-Oriented Programming (OOP) languages took over the world with class and
object, the more general reusable units containing functions and data structures. [Sebesta, 2007]

OOP encourages developers grouping data structures under a namespace where it's not directly accessible by the rest of
the program and bundling relative functions to form classes. By instancing a class, objects are created to contain
different data but have identical data structures and behaviors. OOP promises a vision: through finding pertinent
objects and factoring them into classes at right granularity, your program would be able to address future requirements
without changing too much code. [Gamma et al, 1994] Projects are also speeded up by spliting workforces for each
independent piece of programs. The set of cooperating classes making up a reusable design can form a framewok for
targeted class of software. With a framework, creating a particular application is just creating some specific
subclasses of abstract classes from the framework. Actually, the road to fit the prerequisite is not flat.

The hard part about designing a framework is decomposing a system into objects due to many factor: encapsulation,
granularity, flexibility, evolution, reusability and so on [Gamma et al, 1994]. A framework designer has to gamble that
one architecture will work for all applications in the domain so that a framework should be as flexible and extensible
as possible. Loose coupling is another imperative issue for preventing major repercussion in applications when doing a
minor change to the framework. Facing challenges one after another, few researchers of FDTD are willing to spend time on
proposing a design of FDTD framework of which they can take advantage.

Fortunately, the components of FDTD simulator have given us a native prototype of architecture. Every component can just
corresponds to a class. The emergent problem is they have some functionalities overlapping for simple or complex
situations respectively. How to design a extensible way making components for complex situations can be adopted when
needed becomes the main delimma. This chapter is aimed at proposing a experimental framework of FDTD. The project
named Yet Another FDTD (yaFDTD) framework was released under GPL v2 license and hosted at
\texttt{\url{http://github.com/shelling/yafdtd}}. Even it's a implementation in Python programming language, the
concepts about the split of data structures and classes can be transplanted into other languages easily.

This experimental framework has some goals here. The first is to assemble components easily. We may hope components can
be convened like following codes in a 2D simulation.
\begin{code}
  plane = PBC(plane)
  plane = UPML(plane)
  plane = TFSF(plane)
\end{code}
In this way, a usable plane are standby in a few lines. All field variables are hidden in the object
\texttt{plane}. Unnecessary components can be dropped without marking as commnets along with the variables used by these
components.

Next, configuration should be done through calling the interfaces to wrap the logic inside the components. Following are
some possible usages of the interfaces. Configuring through calling interfaces helps developers thinking the arrangement
in abstract, reducing possibility to make errors, and starting simulations soon.
\begin{code}
  plane.pml(
    x = False,
    y = True
    thick = 20
  )
  plane.pbc(
    x = True,
    y = False
  )
  plane.tfsf(
    x = 0,
    y = 20
  )
\end{code}
Finally, to allow developers extend this framework with arbitrary compoments for complex situations, it needs a class to
play this role to connect components together. A new Design Pattern named Once Decorator is proposed here due to the
fact there is no proper pattern recorded in popular lists.

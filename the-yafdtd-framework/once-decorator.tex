\section{Once Decoractor}
Decoractor Pattern is a well-known Design Pattern listed in the bestseller of Gang of Four, \textit{Design Patterns:
  Elements of Reusable Object-Oriented Software}. The book describe the purpose of Decorator Pattern is attaching
additional responsibilities to an object dynamically. Decorators provide a flexible alternative to subclassing for
extending functinoality.

Similar name imply similar behavior. Once Decorator is the mixing of Singleton Pattern and Dynamic Inheritance acting
like Decorator but actually not. The most remarkable difference is Decorator Pattern delegates the method call to main
component, i.e. all subclasses should have the same interface as main component but the Once Decorator inherit the main
component dynamically when initializing. Dynamic Inheritance allows subclasses overwriting interfaces to adapt more
complex situation as well as attaching additional interfaces without providiong in every related class.

Dynamic Inheritance induces another problem: once two instances perform inheritance, the latter would change the
behaviors of the former. Because a FDTD simulaion need only one main component, Singleton Pattern act inside of the Once
Decorator for preventing modification of inheritance on identical subclass. Moreover, all subclasses save the singleton
instance in main component to ensure one and only one singleton instance can be allowed to join the calculation.

The architecture of Once Decorator is also similar to Decorator as shown in Fig.??. The trick is hidden in constructor
of Decorator. Assuming main component of two dimension is named as Plane, the Decorator can be implemented as following.
\begin{code}
  class PlaneDecorator(Plane):
    def __new__(clz):
        if not clz.singleton:
            clz.singleton = object.__new__(clz)
            Plane.singleton = singleton
        return clz.singleton

    def __init__(self, orig):
        self.__dict__ = orig.__dict__
        if orig.__class__ != Plane:
            self.__class__.__bases__ = (orig.__class__,)
        return None
\end{code}
The concrete decorator subclasses would inhert the abstract decorator class and invoke the constructor when
initializing. The constructor do two things: exporting all instance variables to new instance created from any concrete
decorator and make the concrete decorator inherting the another concrete decorator from original instance.

It's clear the two action save the instance variables and make later decorators can overwrite interfaces of prior
decorators through inheritance in concise relation.

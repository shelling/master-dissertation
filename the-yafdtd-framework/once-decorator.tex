\section{Once Decoractor}
The Decoractor Pattern is a well-known Design Pattern listed in the bestseller of Gang of Four, \textit{Design Patterns:
  Elements of Reusable Object-Oriented Software}. The book depicts that the purpose of the Decorator Pattern is
attaching additional responsibilities to an object dynamically. The role of Decorators in this pattern provides a
flexible alternative to subclassing for extending functionality.

Similar name implies similar behaviors. The Once Decorator Pattern is the mixing of Singleton Pattern and Dynamic
Inheritance, which is acting like the Decorator Pattern but actually not. The most remarkable difference is the
Decorator Pattern delegates all method calls to what it decorates, i.e. all subclasses should provide the same
interfaces as main component when the Once Decorator inherits the main component dynamically to retrieve the interfaces
from what it decorates when initializing. Dynamic Inheritance also allows subclasses overwriting interfaces to adapt
more complex situations as well as attaching additional interfaces without providing in every related class. These are
advantages of the Once Decorator Pattern.

Dynamic Inheritance induces another problems: once two instances perform inheritance, the latter would change the
behaviors of the former. Because a FDTD simulaion needs only one main component, Singleton Pattern acts inside of the
Once Decorator for preventing modification of inheritance on identical subclass. Moreover, all subclasses save the
singleton instance in main component to ensure one and only one singleton instance can be allowed to join the
calculation.

The architecture of Once Decorator is also similar to Decorator as shown in Fig 3.1. The magic is hidden in constructor
of Decorator. Assuming the main component of two dimension is named as \texttt{Plane}, the Decorator can be implemented
as following.
\begin{code}
  class PlaneDecorator(Plane):
    def __new__(clz):
        if not clz.singleton:
            clz.singleton = object.__new__(clz)
            Plane.singleton = singleton
        return clz.singleton

    def __init__(self, orig):
        self.__dict__ = orig.__dict__
        if orig.__class__ != Plane:
            self.__class__.__bases__ = (orig.__class__,)
        return None
\end{code}
The concrete decorator subclasses would inherit the abstract decorator class and invoke the constructor of the abstract
decorator class when initializing. The constructor does two things: exporting all instance variables to a new instance
created from any concrete decorator and making the concrete decorator inheriting the another concrete decorator from the
original instance.

The two actions save the instance variables and make later decorators can overwrite interfaces of prior decorators
through inheritance in concise relation.

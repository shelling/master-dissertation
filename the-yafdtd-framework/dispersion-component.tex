\section{Dispersion Component}
Owing to the demand of supports to multiple poles in Eq.\ref{eq:dispersive}, the dispersion component has to be splited
into two part. The $\epsilon_r$ are saved in main decorated component to participate calculation after retrieving the
summation of $J_p$. Oppositely, $J_p$ saving $a,b,c,d$ coefficients is implemented as a independent class not involving
in Once Decorator inheritance structure.

One more decorator is needed to append $\epsilon_r$ to main component and overwrite the \texttt{update\_efield} method
as following. 
\begin{code}
    def update_efield(self, *polar):
        self.exfield  = self.dxfield.copy()
        self.eyfield  = self.dyfield.copy()
        self.ezfield  = self.dzfield.copy()
        for p in polar:
            self.exfield -= p.x
            self.eyfield -= p.y
            self.ezfield -= p.z
        self.exfield /= self.epsilon_rx
        self.eyfield /= self.epsilon_ry
        self.ezfield /= self.epsilon_rz
        return self
\end{code}
The variable \texttt{*polar} with star declares it can receive arbitrary arguments and act as an array. In the middle
for loop, $E$ field uses the values copied from $D$ field to subtract the $J_p$. Then $\epsilon_r$ divides $E - \sum
J_p$ in latter three lines. The reason each E field component has its own appointment of $\epsilon_{r}$ has been
explained in Section \ref{sec:modeling}.

The second part of dispersion component is $J_p$. The constructor of $J_p$ does two things. The first is calculating
$C1$, $C2$, and $C3$ from $a,b,c,d$ coefficients. The other is appending instance variables for saving polarized
displacement of recent three steps. From Eq.\ref{eq:polarized_displacement}, the method \texttt{update} was implemented
as following. The strategy to distinguish material is slightly different from modeling $\epsilon_r$. Logically, one pole
saves one set of coefficients, so that it's not necessary saving coefficients as arrays. Masks are a better design here.
\begin{code}
    def update(self, plane):
        self.xp2 = self.xp
        self.yp2 = self.yp
        self.zp2 = self.zp
        self.xp = self.x
        self.yp = self.y
        self.zp = self.z
        self.x = self.c1*self.xp + self.c2*self.xp2 + self.c3*plane.exfield*self.maskx
        self.y = self.c1*self.yp + self.c2*self.yp2 + self.c3*plane.eyfield*self.masky
        self.z = self.c1*self.zp + self.c2*self.zp2 + self.c3*plane.ezfield*self.maskz
        return self
\end{code}
In the main FDTD loop, the two parts constructed above collaborate after retrieving current $D$ field. All poles update
themselves with previous $E$ field and $E$ field finds its current values from all poles.
\begin{code}
  plane.update_dfield()

  # three poles are defined 
  drude_pole.update(plane)
  lorentz_pole1.update(plane)
  lorentz_pole2.update(plane)

  plane.update_efield(drude_pole, lorentz_pole1, lorentz_pole2) # allow arbitrary poles 
\end{code}

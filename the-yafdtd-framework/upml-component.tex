\section{UPML Component}
The hard part of UPML implementation is the summation term. The solution is deploying auxiliary variables to sum current
values of curl and participate updating. In the case of 2D, (\ref{eq:pmldx}), (\ref{eq:pmldy}), and (\ref{eq:pmldz})
was implemented as following.
\begin{code}
    def update_dfield(self):
        self.idx += self.curl_hx()
        self.idy += self.curl_hy()
        self.dxfield = self.j3 * self.dxfield 
                     + self.j2 * 0.5 * ( self.curl_hx() + self.i1 * self.idx )
        self.dyfield = self.i3 * self.dyfield 
                     + self.i2 * 0.5 * ( self.curl_hy() + self.j1 * self.idy )
        self.dzfield = self.i3 * self.j3 * self.dzfield 
                     + self.i2 * self.j2 * 0.5 * self.curl_hz()
        return self
\end{code}
The summation of $curl\_hx$ and $curl\_hy$ are saved in \texttt{idx} and \texttt{idy}. The variables defined in
(\ref{eq:v3}), (\ref{eq:v2}), (\ref{eq:v1}) are saved as arrays \texttt{i1}, \texttt{i2}, \texttt{i3}, \texttt{j1},
\texttt{j2}, \texttt{j3} which are mapping coefficients to every point in computational domain. In Sullivan's example,
he used one dimension arrays to save memory. However, it's more intuitive and programmable using array having
dimension(s) the same as field variables.

The multiplication operator of numpy ndarray do multiplication for each pair having the same index like inner
product. That makes the code clean as in mathematics. In any programming language allowing operator overloading,
including Perl, Ruby, C++ and Java, it is a better alternative using a function as middleware to enclose for loops
rather than writing for loop here directly.

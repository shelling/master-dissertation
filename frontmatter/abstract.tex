\cleardoublepage
\chapter{Abstract}
In this thesis we propose a modern architecture of the Finite-Difference Time-Domain method through importing concepts
of Object-Oriented Programming and apply to real world structures. Most implementations are created in procedural style
even in a language supporting Object-Oriented Programming due to the difficulty to separate components from the main
program in traditional formulas. Procedural style is intuitive to transform formulas into codes. However, it needs
considerable changes to suit different cases. Modularized Maxwell's equations are discussed and transformed into codes in
this thesis. For assembling components well, we design a new Design Pattern to extend, overwrite, and delegate methods to
the main component. Finally this implementation is applied to simulations of some dispersive plasmonic structures.

\cleardoublepage
\chapter{\kai 摘要}
{
\kai
本篇論文試圖引入物件導向來重新設計有限差分時域法之架構,並應用於實際結構。
由於在傳統推導的公式下,各種元件難以分離運作,多數實做皆採用程序式寫法。
程序式寫法之優點為易於將數學公式程式化,相對代價則為在模擬條件不同的情況下,程式需要大幅改寫才能適用。
本論文論述將馬克士威方程式重新分割並拆解為不同程式片段之方法,並予以實作。
為了在實作中良好組合各元件,我們自行設計了一組新的設計模式單次裝飾器。
利用此設計模式能對主要元件自由進行方法擴充,覆寫,和委任。
最後此實作被實際利用在幾個色散電漿子介質結構的模擬。}

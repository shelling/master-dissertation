\section{Modeling of Objects}
\label{sec:modeling}

\subsection{Center Shift}
In the previous section, (\ref{eq:dispersive}) has provided clear view for modeling of objects: by marking $\epsilon_r$
and $a, b, c,d$ coefficients of $J_p$ for each point of the computational domain, it does. However, in consequence of
coordinate transformation done with (\ref{eq:coordinate_transform}), the center should be shifted for different
field components.

For example, there is a dielectric sphere with center $(i_0,j_0,k_0)$ and radius $r_0$. Due to the convenience, the
$E_x|_{i+\frac{1}{2},j,k}$ field is saved as $E_x[i,j,k]$ in computer memory. Corresponding
$\epsilon_{rx}|_{i+\frac{1}{2},j,k}$ or $\epsilon_{rx}[i,j,k]$ have to be marked with the sphere with the center $(i_0-1/2,
j_0, k_0)$ and radius $r_0$ in the coordinate in computer memory. That is, center of $\epsilon_{rx}$ has a shift
quantity $(-1/2,0,0)$.

The rest five field components should also be shifted as 
\begin{displaymath}
  E_y \rightarrow (0,-1/2,0)
\end{displaymath}
\begin{displaymath}
  E_z \rightarrow (0,0,-1/2)
\end{displaymath}
\begin{displaymath}
  H_x \rightarrow (0,-1/2,-1/2)
\end{displaymath}
\begin{displaymath}
  H_y \rightarrow (-1/2,0,-1/2)
\end{displaymath}
\begin{displaymath}
  H_z \rightarrow (-1/2,-1/2,0)
\end{displaymath}

\subsection{Modeling Scheme}
Another issue of modeling is about the curved surface. The finite-difference strategy makes the surface to be a shape of
sawteeth in the original Stair-Case scheme: it is a flip-flop view on dispersive coefficents. The high reflection and
non-linear effect on the surface increases the error in the stair-case approximation. Index-average scheme is a common
impovement to stair-case by assigning average coefficents to points at the edge of objects through the proportion. The
Conformal scheme is another solution proposed by [\textit{Mohzmmadi et al}, 2009]. By importing a fourth order
time-stepping scheme, it performs better that the index-average.


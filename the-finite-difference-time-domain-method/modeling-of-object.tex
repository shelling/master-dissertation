\section{Modeling of Objects}
\label{sec:modeling}

\subsection{center shift}
In previous section, Eq.\ref{eq:dispersive} has provide clear view for modeling of objects: by marking $\epsilon_r$ and
$a, b, c,d$ coefficients of $J_p$ for each point of computation domain, it does. However, in consequence of coordinate
transformation done with Eq.\ref{eq:coordinate_transform}, the center should be shifted for different component of
fields.

For example, there is a dielectric sphere with center $(i_0,j_0,k_0)$ and radius $r_0$. Due to the convenience, the
$E_x|_{i+\frac{1}{2},j,k}$ field was saved as $E_x[i,j,k]$ in computer memory. Corresponding $\epsilon_{rx}[i,j,k]$ have
to be marked with the sphere with center $(i_0-1/2, j_0, k_0)$ and radius $r_0$ in the coordinate in computer memory.

The rest five components of fields should also be shifted as following
\begin{displaymath}
  E_y \rightarrow (0,-1/2,0)
\end{displaymath}
\begin{displaymath}
  E_z \rightarrow (0,0,-1/2)
\end{displaymath}
\begin{displaymath}
  H_x \rightarrow (0,-1/2,-1/2)
\end{displaymath}
\begin{displaymath}
  H_y \rightarrow (-1/2,0,-1/2)
\end{displaymath}
\begin{displaymath}
  H_z \rightarrow (-1/2,-1/2,0)
\end{displaymath}

\section{modeling scheme}
Stair-Case scheme

Index-Average scheme

Conformal scheme


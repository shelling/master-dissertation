\section{Dispersive Material}


\subsection{Isotropic Dispersive Material}


\subsubsection{Debye Media}
\begin{equation}
  \chi_p(\omega) = \frac{\Delta\epsilon_p}{1+j\omega\tau_p}  
\end{equation}
\begin{equation}
  \chi_p(t) = \frac{\Delta \epsilon_p}{\tau_p} e^{-\frac{t}{\tau_p}}U(t)  
\end{equation}
\begin{equation}
  \epsilon_r(\omega) = \epsilon_{\infty} + \sum_{p=1}^P \frac{\Delta\epsilon_p}{1+j\omega\tau_p}  
\end{equation}



\subsubsection{Lorentz Media}
\begin{equation}
  \chi_p(\omega) = \frac{\Delta\epsilon_p\omega_p^2}{\omega_p^2 + 2j\omega\delta_p - \omega^2}  
\end{equation}
\begin{equation}
  \chi_p(t) = \frac{\Delta \epsilon_p \omega_p^2}{\sqrt{\omega_p^2 - \delta_p^2}}e^{-\delta_p t}\sin\left(\sqrt{\omega_p^2-\delta_p^2}\ t\right)U(t)
\end{equation}
\begin{equation}
  \epsilon_r(\omega) = \epsilon_{\infty} + \sum_{p=1}^P \frac{\Delta\epsilon_p\omega_p^2}{\omega_p^2 + 2j\omega\delta_p - \omega^2}  
\end{equation}


\subsubsection{Drude Media}
\begin{equation}
  \chi_p(\omega) = -\frac{\omega_p^2}{\omega^2 - j\omega\gamma_p}  
\end{equation}
\begin{equation}
  \chi_p(t) = -\frac{\omega_p^2}{\gamma_p}\left(1-e^{-\gamma_p t}\right) U(t)
\end{equation}
where U(t) is Heaviside
The single pole permittivity is 
\begin{equation}
  \epsilon_r(\omega) = \epsilon_{\infty} - \sum_{p=1}^P \frac{\omega_p^2}{\omega(\omega-j\gamma_p)}
\end{equation}
where $\omega_p$ is the Drude pole frequency and $\gamma_p$ is the inverse of the pole relaxation time, also known as the
electron oscillision frequency.



\subsection{Dispersion-Compatible Update Equations}
In this section, we would introduce three algorithm to adjust the update equations to suit the dispersive material shown
above.

We would see that the update equations would have the same forms after transformming no matter which way we choose. It
means, for some problems convenient to apply specified algorithm but not easy to apply another one, 

\subsubsection{The Piecewise-Linear Recursive Convolution Method}
Piecewise-Linear Recursive Convolution (PLRC) Method

\paragraph{{\msjh Debye Model - PLRC}}

\paragraph{{\msjh Lorentz Model - PLRC}}

\paragraph{{\msjh Drude Model - PLRC}}

\subsubsection{The Auxiliary Differential Equation Method}
The 

By defining the $\chi$-related term as the phasor polorization current $J_p(\omega)$.
\paragraph{{\msjh Debye Model - ADE}}
\begin{equation}
  \nabla \times H(\omega)
\end{equation}

\paragraph{{\msjh Lorentz Model - ADE}}
\begin{equation}
  \nabla \times
\end{equation}

\paragraph{{\msjh Drude Model - ADE}}
\begin{equation}
    \nabla \times H(\omega) = j\omega \epsilon_r(\omega)E(\omega) = j\omega\epsilon_0 \epsilon_{\infty} E(\omega) - j\omega\epsilon_0\frac{\omega_p^2}{\omega(\omega-j\gamma_p)}E(\omega)
\end{equation}

Defining the last term as $J_p$
\begin{equation}
  J_p(\omega) = -j\omega\epsilon_0\frac{\omega_p^2}{\omega(\omega-j\gamma_p)}E(\omega)
\end{equation}

Rearrange
\begin{equation}
  \omega^2J_p(\omega) - j\omega\gamma_pJ_p(\omega) = -j\omega\epsilon_0\omega_p^2 E(\omega)
\end{equation}

Performming inverse Fourier transformation
\begin{equation}
  \frac{\partial^2 J_p}{\partial t^2} + \gamma_p \frac{\partial J_p}{\partial t} = \epsilon_0\omega_p^2\frac{\partial E}{\partial t}
\end{equation}

Integrating once
\begin{equation}
  \frac{\partial J_p}{\partial t} + \gamma_p J_p = \epsilon_0 \omega_p^2 E(\omega)
\end{equation}

This is the ADE for $J_p$ for Drude Model

Apply the semi-implicit scheme
\begin{equation}
  \Huge(\frac{J_p^{n+1} - J_p^n}{\Delta t}\Huge) + \gamma_{p}\Huge(\frac{J_p^{n+1} + J_p^n}{2}\Huge) = \epsilon_0\omega_p^2\Huge(\frac{E^{n+1} + E^n}{2}\Huge)
\end{equation}

Solving
\begin{equation}
  J_p^{n+1} = \Huge(\frac{1-\gamma_p\Delta t /2}{1+\gamma_p\Delta t /2}\Huge)J_p^n + \Huge(\frac{\omega_p^2\epsilon_0\Delta t /2}{1+\gamma_p\Delta t /2}\Huge)(E^{n+1}+E^n)
\end{equation}


\begin{equation}
  E^{n+1} = \Huge(\frac{2\epsilon_0\epsilon_{\infty} - \Delta t \beta_p}{2\epsilon_0\epsilon_{\infty} + \Delta t \beta_p}\Huge)E^n + \Huge(\frac{2\Delta t}{2\epsilon_0\epsilon_{\infty} + \Delta t \beta_p}\Huge)\Huge(\nabla\times H^{n+1/2} - \frac{1}{2}(1+k_p)J_p^n\Huge)
\end{equation}



\subsubsection{The Z Transform Method}
\paragraph{Debye Model - ZT}

\paragraph{Lorentz Model - ZT}

\paragraph{Drude Model - ZT}

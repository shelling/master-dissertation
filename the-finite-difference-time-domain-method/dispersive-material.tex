\section{Dispersive Material}
\label{sec:dispersive}
This section conerns of how to retrieve the update equations could apply in dispersive material. What would be discussed
below are only about material having dispersive electic permittivity. The material having dispersive magnetic
permeability could be investigated via duality.

In fundamental Electromagnetics, an indispersive material having a electrical polarization $P$ when there is a foreign
electric field and the phasor of electric flux would become $D(\omega) = \epsilon_0 E(\omega) + P$. $P$ was also defined
as $\epsilon_0 \chi_e E(\omega)$, so that we could rewrite $D(\omega)$ as $\epsilon_0 (1+\chi_e)E(\omega)$ and give the middle term a notation
$\epsilon_r$ named relative permittivity.

By definition, dispersive material is the material having different electic permeability when encountering EM wave
having different frequency that means the relative permittivity should be a function of frequency ordinarily coming with
image part. So, $\epsilon_r^*(\omega)$ was given as the notation of dispersive electic permittivity and the phasor of
electic flux in dispersive material was defined as $D(\omega) = \epsilon_0 \epsilon_r^*(\omega)E(\omega)$.
\subsection{Common Isotropic Dispersive Material}
The dispersive permittivity is defined as $\epsilon_r^*(\omega) = \epsilon_r + \sigma_e / j\omega\epsilon_0 + \chi_p(\omega)$, where
$\epsilon_r$ is ... and $\chi_p(\omega)$ is ...
\subsubsection{Simple Lossy Media}
Simple Lossy Media is a material with constant electric conductivity $\sigma_e$ and without $\chi_p(\omega)$
\begin{equation}
  \epsilon_r^*(\omega) = \epsilon_0 + \frac{\sigma_e}{j\omega\epsilon_0}
\end{equation}

\subsubsection{Debye Media}
\begin{equation}
  \chi_p(\omega) = \frac{\Delta\epsilon_p}{1+j\omega\tau_p}  
\end{equation}
\begin{equation}
  \chi_p(t) = \frac{\Delta \epsilon_p}{\tau_p} e^{-\frac{t}{\tau_p}}U(t)  
\end{equation}
\begin{equation}
  \epsilon_r^*(\omega) = \epsilon_r + \frac{\sigma_e}{j\omega\epsilon_0} + \sum_{p=1}^P \frac{\Delta\epsilon_p}{1+j\omega\tau_p}  
\end{equation}



\subsubsection{Lorentz Media}
\begin{equation}
  \chi_p(\omega) = \frac{\Delta\epsilon_p\omega_p^2}{\omega_p^2 + 2j\omega\delta_p - \omega^2}  
\end{equation}
\begin{equation}
  \chi_p(t) = \frac{\Delta \epsilon_p \omega_p^2}{\sqrt{\omega_p^2 - \delta_p^2}}e^{-\delta_p t}\sin\left(\sqrt{\omega_p^2-\delta_p^2}\ t\right)U(t)
\end{equation}
\begin{equation}
  \epsilon_r^*(\omega) = \epsilon_r + \sum_{p=1}^P \frac{\Delta\epsilon_p\omega_p^2}{\omega_p^2 + 2j\omega\delta_p - \omega^2}  
\end{equation}


\subsubsection{Drude Media}
\begin{equation}
  \chi_p(\omega) = -\frac{\omega_p^2}{\omega^2 - j\omega\gamma_p}  
\end{equation}
\begin{equation}
  \chi_p(t) = -\frac{\omega_p^2}{\gamma_p}\left(1-e^{-\gamma_p t}\right) U(t)
\end{equation}
where U(t) is Heaviside
The single pole permittivity is 
\begin{equation}
  \epsilon_r^*(\omega) = \epsilon_r - \sum_{p=1}^P \frac{\omega_p^2}{\omega^2-j\omega\gamma_p}
\end{equation}
where $\omega_p$ is the Drude pole frequency and $\gamma_p$ is the inverse of the pole relaxation time, also known as the
electron oscillision frequency.



\subsection{Dispersion-Compatible Update Equations}
For the analysis of dispersive materials shown above, numbers of algorithms have already been proposed in literature.
Most of these frequency-dependent algorithms can be categorized into three types: 
\begin{inparaenum}[(1)]
\item the Recursive Convolution (RC) method
\item the Auxiliary Differential Equation (ADE) method
\item the Z-transform (ZT) method
\end{inparaenum}.

The RC method is the most basic method inheriting the convolution theorem in Laplace transform and Fourier
transform. The pros and cons are easy to understand with background of engineering mathematics and difficult to handle
complex material with multiple poles.

The ADE method offer high flexibility in fitting arbitrary permittivity functions, modeling nonlinear effects and
arbitrary numbers of poles.

Any of these methods focuses on tranforming the dispersive relations between $D(\omega))$ and $E(\omega)$ in frequency domain
back to the time doamin for discretization. By applying different discretizing scheme at each step, many varieties of
them were given. Following words are trying to give an overview and examples as many as possible to be a reference
during implementing.

\subsubsection{The Recursive Convolution Method}
There are many varieties published using convolution including 
\begin{inparaenum}[(1)]
\item Recursive Convolution (RC) Method
\item Piecewise-Linear Recursive Convolution (PLRC) Method
\item Trapezoidal Recursive Convolution(TRC) Method
\end{inparaenum}

\paragraph{\msjh Simple Conductive Media - RC}
Apply RC to simple conductive material having definition as
\begin{displaymath}
  \epsilon_r^{*}(\omega) = \epsilon_r + \frac{\sigma}{j \omega \epsilon_0}
\end{displaymath}
\begin{displaymath}
  \begin{split}
    D(\omega) & = \epsilon_0 \epsilon_r^{*}(\omega) E(\omega)\\
    & = \epsilon_0 \epsilon_r E(\omega) + \frac{\sigma}{j\omega}E(\omega)
  \end{split}
\end{displaymath}
\begin{displaymath}
  D(t) = \epsilon_0 \epsilon_rE(t) + \sigma \int_0^t E(t')dt'
\end{displaymath}
\begin{displaymath}
    D^n = \epsilon_0 \epsilon_r E^n + \sigma \Delta t \sum_{i=0}^{n}E^i
\end{displaymath}
The second term: dissipated displacement -> $I^n$, in recursive form
\begin{equation}
  I^n = \sigma\Delta t\cdot E^n + \sigma\Delta t\sum_{i=0}^{n-1}E^i = \sigma\Delta t\cdot E^n + I^{n-1}
\end{equation}
\begin{equation}
    D^n = \epsilon_0 \epsilon_r E^n + I^n = (\epsilon_0 \epsilon_r + \sigma \Delta t) E^n + I^{n-1}
\end{equation}
Finally, the update equations becomes
\begin{gather*}
  E^n = \frac{D^n - I^{n-1}}{\epsilon_0 \epsilon_r + \sigma \Delta t}\\
  I^n = I^{n-1} + \sigma \Delta t\cdot E^n
\end{gather*}
implementation
\begin{code}
  dx[k] = dx[k]
  ex[k] = ex[k]
  i[k]  = i[k]
\end{code}


\paragraph{{\msjh Debye Model - RC}}
\begin{displaymath}
  \epsilon_r^*(\omega) = \epsilon_r + \frac{\sigma}{j\omega \epsilon_0} + \frac{\Delta \epsilon_p}{1+j\omega \tau_p}
\end{displaymath}
\begin{displaymath}
  \begin{split}
    D(\omega) & = \epsilon_0 \epsilon_r^*(\omega) E(\omega)\\
    & = \epsilon_0 \epsilon_r E(\omega) + \frac{\sigma}{j\omega} E(\omega) + \frac{\Delta \epsilon_p}{1+j\omega \tau_p}E(\omega)
  \end{split}
\end{displaymath}
\begin{displaymath}
  D(t) = \epsilon_0 \epsilon_r E(t) + \sigma \int_0^t E(t')dt' + \epsilon_0 \int_0^t \frac{\Delta \epsilon_p}{\tau_p}e^{-\frac{t'-t}{\tau_p}}E(t')dt'
\end{displaymath}
\begin{equation}
  D^n = \epsilon_0 \epsilon_rE^n + \sigma \Delta t \sum_{i=0}^{n}E^i + \epsilon_0 \frac{\Delta \epsilon_p}{\tau_p}\Delta t \sum_{i=0}^{n} e^{-\frac{n-i}{\tau_p}\Delta t} E^i
\end{equation}
The second term, dissipated displacement, $I^n$ , the same as simple conductive media.
The third term, phasor polarization displacement, $J_p^n$. $J_p^n$ is slightly complex in recursive form.
\begin{equation}
  \begin{split}
    J_p^n & = \epsilon_0\frac{\Delta\epsilon_p}{\tau_p}\Delta t \sum_{i=0}^ne^{-\frac{n-i}{\tau_p}\Delta t}E^i\\
    & = \epsilon_0\frac{\Delta\epsilon_p}{\tau_p}\Delta t \left( E^n + \sum_{i=0}^{n-1}e^{-\frac{n-i}{\tau_p}\Delta t}E^i\right)
  \end{split}
\end{equation}
and 
\begin{equation}
  \begin{split}
    J_p^{n-1} & = \epsilon_0\frac{\Delta\epsilon_p}{\tau_p}\Delta t \sum_{i=0}^{n-1}e^{-\frac{n-1-i}{\tau_p}\Delta t}E^i\\
    & = \epsilon_0\frac{\Delta\epsilon_p}{\tau_p}\Delta t \left( e^{\frac{\Delta t}{\tau_p}} \right) \sum_{i=0}^{n-1}e^{-\frac{n-i}{\tau_p}\Delta t}E^i
  \end{split}
\end{equation}
substituted into $J_p^n$
\begin{equation}
  J_p^n = \epsilon_0\frac{\Delta\epsilon_p}{\tau_p}\Delta t\cdot E^n + e^{-\frac{\Delta t}{\tau_p}} J_p^{n-1}
\end{equation}
\begin{equation}
  \begin{split}
    D^n & = \epsilon_0 \epsilon_r E^n + \left[\sigma \Delta t\cdot E^n + I^{n-1}\right] + \left[\epsilon_0 \frac{\Delta \epsilon_p}{\tau_p}\Delta t\ E^n + e^{-\frac{\Delta t}{\tau_p}} J_p^{n-1}\right]\\
    & = \left(\epsilon_0 \epsilon_r + \sigma \Delta t + \epsilon_0 \frac{\Delta \epsilon_p}{\tau_p} \Delta t\right)E^n + I^{n-1} + e^{-\frac{\Delta t}{\tau_p}} J_p^{n-1}
  \end{split}
\end{equation}
Finally the discrete constitute relations in time domain becomes
\begin{gather}
  \begin{array}{@{}l@{}}
    E^n =  \frac{\displaystyle D^n - I^{n-1} - e^{-\frac{\Delta t}{\tau_p}}J_p^{n-1} }{\displaystyle\epsilon_0 \epsilon_r + \sigma \Delta t + \epsilon_0 \frac{\Delta \epsilon_p}{\tau_p} \Delta t}\\    
    I^n = \sigma\Delta t\cdot E^n + I^{n-1}\\
    J_p^n = \epsilon_0\frac{\Delta\epsilon_p}{\tau_p}\Delta t\cdot E^n + e^{-\frac{\Delta t}{\tau_p}} J_p^{n-1}
  \end{array}
\end{gather}
implement
\begin{code}
  dx[k] = dx[k] + 0.5 * ( hy[k-1] - hy[k])
  ex[k] = ( dx[k] - i[k] - exp(-dt/tau_p) * j[k] ) 
  * ( epsilon_0 * epsilon_r[k] + sigma_e[k] * dt +  )
  i[k] = i[k] + sigma_e[k] 
  j[k] = j[k] + del
\end{code}






\paragraph{{\msjh Lorentz Model - RC}}


\paragraph{{\msjh Drude Model - RC}}
\begin{equation}
  D(\omega) = \epsilon_0\epsilon_r^*(\omega)E(\omega) = \epsilon_0\epsilon_rE(\omega) - \epsilon_0\frac{\omega_p^2}{\omega^2-j\omega\gamma_p}E(\omega)
\end{equation}
\begin{equation}
  D(t) = \epsilon_0\epsilon_rE(t) + \epsilon_0\int_0^t -\frac{\omega_p^2}{\gamma_p}(1 - e^{-\gamma_p(t'-t)})E(t')dt'
\end{equation}
\begin{equation}
  D^n = \epsilon_0\epsilon_rE^n + \epsilon_0\frac{\omega_p^2}{\gamma_p}\Delta t\sum_{i=0}^{n}E^i + \epsilon_0\frac{\omega_p^2}{\gamma_p}\Delta t \sum_{i=0}^{n}e^{-\gamma_p(n-i)\Delta t} E^i
\end{equation}



\subsubsection{The Auxiliary Differential Equation Method}
The central idea of the Auxiliary Differential Equation (ADE) Method is to detach the phasor polarization displacement
$J_p(\omega)$, that is, $\chi_p(\omega)E(\omega)$, from original dispersive relation.
\begin{equation}
  \begin{split}
    \widetilde{D}(\omega) &= \epsilon_r^*(\omega)\widetilde{E}(\omega)\\
    & = \epsilon_r\widetilde{E}(\omega) + \sum_{p=1}^P\chi_p(\omega)\widetilde{E}(\omega)\\
    & = \epsilon_r\widetilde{E}(\omega) + \sum_{p=1}^{P}J_p(\omega)
  \end{split}
\end{equation}
For muiltipole material composed of different dispersions, $J_p$ were solved for each pole individually to attend the
updating loop, and then $E$ can be solved via rearrangment of previous relation.
\begin{equation}
  \widetilde{E}^n = \frac{1}{\epsilon_r}\left(\widetilde{D}^n - \sum_{p=1}^PJ_p^n\right)
\end{equation}
Primitive ADE method shown in literature require deriving formulations for each dispersion type, however, a generalized
way \textit{Alsunaidi et al.} proposed finds its strength in unifying the formulation of different dispersion models
into one form.

Starting with the most general form of $\chi_p(\omega)$, the phasor polarization displacement can be written
as
\begin{equation}
  J_p(\omega) = \frac{a}{b + j\omega c - d\omega^2}\widetilde{E}(\omega)
\end{equation}
Rearranging and performming inverse Fourier transform 
\begin{equation}
  bJ_p(\omega) + c \frac{\partial}{\partial t}J_p(\omega) + d \frac{\partial ^2}{\partial t^2}J_p(\omega) = a\widetilde{E}(t)
\end{equation}
apply leapfrog scheme 
\begin{equation}
  bJ_p^{n-1} + c \frac{J_p^n - J_p^{n-1}}{2\Delta t} + d \frac{J_p^n + 2J_p^{n-1} + J_p^{n-2}}{(\Delta t)^2} = a\widetilde{E}^{n-1}
\end{equation}
Solving 
\begin{equation}
  J_p^n = \frac{4d-2b(\Delta t)^2}{2d+c\Delta t}J_p^{n-1} + \frac{-2d+c\Delta t}{2d+c\Delta t}J_p^{n-2} + \frac{2a(\Delta t)^2}{2d+c\Delta t}\widetilde{E}^{n-1}
\end{equation}
which can be written in the form 
\begin{equation}
  J_p^n = C_1 J_p^{n-1} + C_2 J_p^{n-2} + C_3 \widetilde{E}^{n-1}
\end{equation}
where $C_1$, $C_2$ and $C_3$ can be found for any form of dispersion relation.



\paragraph{{\msjh Debye Model - ADE}}
Debye model is a fun case can be induced the same form of $J_p(t)$ as RC if the semi-implicit scheme was chose. However,
leapfrog scheme provide better accruacy. Just for verifying, result of apply semi-implicit scheme was also derived
here. In the real wrold implementation \textit{\uwave{yaFDTD}}, leapfrog was picked out.

Starting with constitute relation as before
\begin{equation}
  \begin{split}
    \widetilde{D}(\omega) & = \epsilon_r^*(\omega)\widetilde{E}(\omega)\\
    & = \epsilon_r\widetilde{E}(\omega) + \frac{\sigma}{j\omega\epsilon_0}\widetilde{E}(\omega) + \frac{\Delta \epsilon_p}{1+j\omega \tau_p}\widetilde{E}(\omega)\label{eq:debye_ade_start}
  \end{split}
\end{equation}
detach $J_p$
\begin{equation}
  J_p(\omega) = \frac{\Delta \epsilon_p}{1+j\omega \tau_p}\widetilde{E}(\omega)
\end{equation}
This is the ADE of Debye Model
\begin{equation}
  J_p(\omega) + j\omega\tau_{p}J_p(\omega) = \Delta\epsilon_p\widetilde{E}(\omega)
\end{equation}
performming IFT 
\begin{equation}
  J_p(t) + \tau_p\frac{\partial}{\partial t}J_p(t) = \Delta\epsilon_p\widetilde{E}(t)
\end{equation}
apply semi-implicit scheme
\begin{equation}
  \left( \frac{J_p^n - J_p^{n-1}}{2} \right) + \tau_p \left( \frac{J_p^n - J_p^{n-1}}{\Delta t}\right) = \Delta\epsilon_p\widetilde{E}^n
\end{equation}
Solving $J_p$
\begin{equation}
  J_p^n = \frac{\left(1-\frac{\Delta t}{2\tau_p}\right)}{\left(1+\frac{\Delta t}{2\tau_p}\right)}J_p^{n-1} 
  + \frac{\left(\frac{\Delta\epsilon_p}{\tau_p}\right)\Delta t}{\left(1+\frac{\Delta t}{2\tau_p}\right)}\widetilde{E}^n
\end{equation}
It should be noted
\begin{equation}
  \begin{array}{@{}lp{0.5cm}r@{}}
    \frac{\displaystyle1-\delta}{\displaystyle1+\delta} \cong e^{-2\delta} && if\ \delta \ll 1
  \end{array}
\end{equation}
\begin{equation}
  \frac{\left(1-\frac{\displaystyle\Delta t}{\displaystyle2\tau_p}\right)}{\left(1+\frac{\displaystyle\Delta t}{\displaystyle2\tau_p}\right)} \cong e^{-\frac{\Delta t}{\tau_p}}\quad  because\ 1 \gg \frac{\Delta t}{2\tau_p}
\end{equation}
That is 
\begin{equation}
  J_p^n \cong e^{-\frac{\Delta t}{\tau_p}}J^{n-1} + \frac{\Delta\epsilon_p}{\tau_p}\Delta t\cdot\widetilde{E}^n
\end{equation}
And by performming inverse Fourier transform on Eq.\ref{eq:debye_ade_start}
\begin{equation}
  \widetilde{D}(t) = \epsilon_r\widetilde{E}(t) + \frac{\sigma}{\epsilon_0} \int_0^t\widetilde{E}(t')dt' + J_p(t)
\end{equation}
\begin{equation}
  \begin{split}
    \widetilde{D}^n & = \epsilon_r\widetilde{E}^n + \frac{\sigma}{\epsilon_0}\Delta t\sum_{i=0}^n\widetilde{E}^i + J_p^n\\
    & = \epsilon_r\widetilde{E}^n + \frac{\sigma}{\epsilon_0}\Delta t\cdot\widetilde{E}^n + I^{n-1} + \frac{\Delta\epsilon_p}{\tau_p}\Delta t\cdot\widetilde{E}^n + e^{-\frac{\Delta t}{\tau_p}}J^{n-1}
  \end{split}
\end{equation}
The same result as Recursive Convolution Method.






\paragraph{\msjh Lorentz Model - ADE} Lorentz pole totally matches the general form used in previous introduction of ADE method.
The coefficients are as following.
\begin{equation*}
  \begin{array}{@{}llll@{}}
    a = \Delta\epsilon_p\omega_p^2 &
    b = \omega_p^2 &
    c = 2\delta_p &
    d = 1
  \end{array}
\end{equation*}
substitute into ...
\begin{gather*}
  \begin{array}{@{}lll@{}}
    C_1 = \frac{\displaystyle 2-\omega_p^2\Delta t^2}{\displaystyle 1+\delta_p\Delta t} &
    C_2 = \frac{\displaystyle -1 + \delta_p\Delta t}{\displaystyle 1+\delta_p\Delta t} &
    C_3 = \frac{\displaystyle \Delta\epsilon_p\omega_p^2\Delta t^2}{\displaystyle 1+\delta_p\Delta t}
  \end{array}
\end{gather*}
Formal derivation is also written down here.
\begin{equation}
  \begin{split}
    \widetilde{D}(\omega) & = \epsilon_r^*(\omega)\widetilde{E}(\omega)\\
    & = \epsilon_r\widetilde{E}(\omega) +  \frac{\Delta \epsilon_p \omega_p^2}{\omega_p^2+2j\omega\delta_p-\omega^2}\widetilde{E}(\omega)
  \end{split}
\end{equation}
\begin{equation}
  J_p(\omega) =  \frac{\Delta \epsilon_p \omega_p^2}{\omega_p^2+2j\omega\delta_p-\omega^2}\widetilde{E}(\omega)
\end{equation}
rearrange and IFT
\begin{equation}
  \omega_p^2J_p(t) + 2\delta_p\frac{\partial}{\partial t}J_p(t) + \frac{\partial^2}{\partial t^2}J_p(t) = \Delta\epsilon_p\omega_p^2\widetilde{E}(t)
\end{equation}
apply leapfrog scheme 
\begin{equation}
  \omega_p^2J_p^{n-1} + \delta_p\frac{J_p^n - J_p^{n-2}}{\Delta t} + \frac{J_p^n - 2 J_p^{n-1} + J_p^{n-2}}{\Delta t^2} = \Delta\epsilon_p\omega_p^2\widetilde{E}^{n-1}
\end{equation}
rearrange 
\begin{equation}
  J_p^n = 
  \frac{ 2-\omega_p^2\Delta t^2}{ 1+\delta_p\Delta t} J_p^{n-1} +
  \frac{ -1 + \delta_p\Delta t}{ 1+\delta_p\Delta t} J_p^{n-2} + 
  \frac{ \Delta\epsilon_p\omega_p^2\Delta t^2}{ 1+\delta_p\Delta t}\widetilde{E}^{n-1}
\end{equation}
as the result showed by general form of $C_1$, $C_2$, $C_3$.



\paragraph{\msjh Drude Model - ADE} Drude pole lacks the b coefficient when comparing to general form.
\begin{equation}
  \begin{split}
    \widetilde{D}(\omega) & = \epsilon_r^*(\omega)\widetilde{E}(\omega)\\
    & =  \epsilon_r\widetilde{E}(\omega) - \frac{\omega_p^2}{\omega(\omega-j\gamma_p)}\widetilde{E}(\omega)
  \end{split}
\end{equation}
Defining the last term as $J_p$, the phasor polarization current in physics,
\begin{equation}
  J_p(\omega) = -\frac{\omega_p^2}{\omega(\omega-j\gamma_p)}\epsilon{E}(\omega)
\end{equation}
Rearrange
\begin{equation}
  j\omega\gamma_pJ_p(\omega) - \omega^2J_p(\omega) = \omega_p^2\widetilde{E}(\omega)
\end{equation}
Performming inverse Fourier transformation
\begin{equation}
  \frac{\partial^2 J_p(t)}{\partial t^2} + \gamma_p \frac{\partial J_p(t)}{\partial t} = \omega_p^2\frac{\partial\widetilde{E}(t)}{\partial t}
\end{equation}
This is the ADE for $J_p$ for Drude Model. Then apply leapfrog scheme as general solution.
\begin{equation}
  \gamma_p\frac{J_p^n-J_p^{n-2}}{2\Delta t} + \frac{J_p^n - 2J_p^{n-1} + J_p^{n-2}}{(\Delta t)^2} = \omega_p^2\widetilde{E}^{n-1}
\end{equation}
rearrange 
\begin{equation}
  J_p^n = \frac{4}{2+ \gamma_p\Delta t} J_p^{n-1} + \frac{-2+\gamma_p\Delta t}{2+\gamma_p\Delta t}J_p^{n-2} + \frac{2\omega_p^2(\Delta t)^2}{2+\gamma_p\Delta t}\widetilde{E}^{n-1}
\end{equation}



All coefficients of different dispersion types are summarized here
\begin{center}
  \begin{tabular}[c]{|r|l|c|c|c|}
    \hline
    Dispersion Type & $\chi_p(\omega)$ & $C_1$ & $C_2$ & $C_3$ \\
    \hline
    Lorentz & $\frac{\Delta \epsilon_p \omega_p^2}{\sqrt{\omega_p^2 - \delta_p^2}}e^{-\delta_p t}\sin\left(\sqrt{\omega_p^2-\delta_p^2}\ t\right)$ & $\frac{2-\omega_p^2\Delta t^2}{1+\delta_p\Delta t}$ & $\frac{-1 + \delta_p\Delta t}{1+\delta_p\Delta t}$  & $\frac{\Delta\epsilon_p\omega_p^2\Delta t^2}{1+\delta_p\Delta t}$ \\
    \hline
    Drude & $\frac{\epsilon_0\omega_p^2}{j\omega\gamma_p-\omega^2}$ & $\frac{ 4}{ 2+\gamma_p\Delta t}$ & $\frac{ -2+\gamma_p\Delta t}{ 2+\gamma_p\Delta t}$ & $\frac{ 2\omega_p^2\Delta t^2}{ 2+\gamma_p \Delta t}$\\
    \hline
  \end{tabular}
\end{center}




\subsubsection{The Z Transform Method}
\paragraph{\msjh Debye Model - ZT}

\paragraph{\msjh Lorentz Model - ZT}

\paragraph{\msjh Drude Model - ZT}

\section{Dispersive Materials}
\label{sec:dispersive}
This section concerns how to retrieve the update equations for applied to dispersive materials. What would be discussed
below is only about material having dispersive electic permittivity. The material having dispersive magnetic
permeability could be investigated via duality.

In fundamental electromagnetics, a nondispersive material having an electrical polarization $P$ when there is a foreign
electric field and the phasor of electric flux would become $D(\omega) = \epsilon_0 E(\omega) + P$. $P$ was also defined
as $\epsilon_0 \chi_e E(\omega)$, so that we could rewrite $D(\omega)$ as $\epsilon_0 (1+\chi_e)E(\omega)$ and give the middle term a notation
$\epsilon_r$ named relative permittivity.

The dispersive material has its electic permittivity depending on the frequency of the interacting EM wave and usually
contains an imaginary part. So, $\epsilon_r^*(\omega)$ is given as the notation of the dispersive electic permittivity,
and the phasor of the electic flux in dispersive materials was defined as $D(\omega) = \epsilon_0
\epsilon_r^*(\omega)E(\omega)$ or $\widetilde{D} = \epsilon_r^*(\omega)\widetilde{E}$ in Gaussian unit.
\subsection{Common Isotropic Dispersive Materials}
The dispersive permittivity is defined as $\epsilon_r^*(\omega) = 1 + \chi_e + \sum \chi_p(\omega)$ and usually, $1 +
\chi_e$ is written as $\epsilon_r$. Many theorems have been proposed to describe the response characteristics of
dispersive material exposed to external fields. Here we briefly introduce Simple Lossy Media, Debye Media, Drude Media
and Lorentz Media.

\subsubsection{Simple Lossy Media}
Simple Lossy Media are materials with constant electric conductivity $\sigma_e$, which respond a current J to an
external electric field with the relation $J = \sigma_e E$. By substituting $J=\sigma_e E$ into
Maxwell's equtions, the following relationships are obtained.
\begin{gather}
  \chi_p(\omega) = \frac{\sigma_e}{j\omega\epsilon_0}\\
  \epsilon_r^*(\omega) = \epsilon_r + \frac{\sigma_e}{j\omega\epsilon_0}
\end{gather}

\subsubsection{Debye Media}
The Debye model is used to describe the dieletric relaxation response of an noninteracting population of dipoles to an
external field. The material response can be described by 
\begin{equation}
  \label{eq:debye_chi}
  \chi_p(\omega) = \frac{\Delta\epsilon_p}{1+j\omega\tau_p}  
\end{equation}
The multipole Debye Model has the following form
\begin{equation}
  \epsilon_r^*(\omega) = \epsilon_r + \sum_{p=1}^P \frac{\Delta\epsilon_p}{1+j\omega\tau_p}  
\end{equation}


\subsubsection{Drude Media}
The Drude model is frequently used to explain optical properties of metal through the theory of free electron gas moving
against a fixed positive ion cores. The electrons oscillate in response to the applied field, and in the meantime damped
due to collision with characteristic frequency $\gamma_p$, also known as the inverse of the pole relaxation time. The
equation of motion of free eletron gas can be written as
\begin{equation}
  m_e\frac{\partial^2 x(t)}{\partial t^2} + m_e\gamma_p\frac{\partial x(t)}{\partial t} = -eE(t)
\end{equation}
By assuming $x(t)$ and $E(t)$ are time-harmonic, $x(t) = \mathrm{Re}\{x_0(t)e^{j\omega t}\}$ and $E(t) =
\mathrm{Re}\{E_0(t)e^{j\omega t}\}$, the displacement of eletron can be solved to be
\begin{equation}
  x(t) = \frac{e}{m_e(\omega^2 - j\omega\gamma_p)}E(t)
\end{equation}
Because the macroscopic polarization $P = -Ne\cdot x(t)$ should also satisfy $P = \epsilon_0\chi_pE(t)$, $\chi_p$ can be
solved to be
\begin{equation}
  \chi_p = \frac{-Ne^2/m_e\epsilon_0}{\omega^2 - j\omega\gamma_p}
\end{equation}
where $(Ne^2/m_e\epsilon_0)^{1/2}$ is defined as the plasma frequency $\omega_p$ to be the common form of the Drude
model is expressed as
\begin{equation}
  \label{eq:drude_chi}
  \chi_p(\omega) = -\frac{\omega_p^2}{\omega^2 - j\omega\gamma_p}  
\end{equation}
Multipole Drude Media can be denote as 
\begin{equation}
  \epsilon_r^*(\omega) = \epsilon_r - \sum_{p=1}^P \frac{\omega_p^2}{\omega^2-j\omega\gamma_p}
\end{equation}



\subsubsection{Lorentz Media}
The Drude model does not fit experimental result of noble metal over some frequency region in which interband
transitions take place [Maier, 200?]. The Lorentz oscillator mechanism provides the description to the motion, and
interband transitions through appending a damping term into the equation of motion of a bound electron.
\begin{equation}\label{eq:lorentz_def}
    m_e\frac{\partial^2 x(t)}{\partial t^2} + 2m_e\delta_p\frac{\partial x(t)}{t} + m_e\omega_p^2x(t) = -eE(t)
\end{equation}
Assuming $x(t)$ and $E(t)$ are time-harmonic again, (\ref{eq:lorentz_def}) can be solved to obtain
\begin{equation}
  \label{eq:lorentz_chi}
  \chi_p(\omega) = \frac{\Delta\epsilon_p\omega_p^2}{\omega_p^2 + 2j\omega\delta_p - \omega^2}  
\end{equation}
\begin{equation}
  \epsilon_r^*(\omega) = \epsilon_r + \sum_{p=1}^P \frac{\Delta\epsilon_p\omega_p^2}{\omega_p^2 + 2j\omega\delta_p - \omega^2}  
\end{equation}
where $\Delta\epsilon_p$ is a weighting factor associated with the specified Lorentz pole. The linear combination of one
Drude pole and multiple Lorentz poles isoften employed to earn a better fitting for the nobel metal. [Taflove and
  Hagness, 2005].



\subsection{Dispersion-Compatible Update Equations}
For the analysis of dispersive materials shown above, a numbers of algorithms have already been proposed in literature.
Most of these frequency-dependent algorithms can be categorized into three types: \begin{inparaenum}[(1)]
\item the Recursive Convolution (RC) method
\item the Auxiliary Differential Equation (ADE) method, and
\item the Z-transform (ZT) method\end{inparaenum}.
Any of these methods focuses on tranforming the dispersive relations between $D(\omega)$ and $E(\omega)$ in frequency
domain back to the time doamin for discretization.


The RC method is the most basic method inheriting the convolution theorem in Laplace transform and Fourier transform. It
is are easy to understand with background of engineering mathematics but difficult to handle complex material with
multiple poles. Many varieties, such as the Piecewise-Linear Recursive Convolution (PLRC) Method and the Trapezoidal
Recursive Convolution (TRC) Method have been proposed to improve the performance of the RC method.


The ADE method and the Z-transform method offer high flexibility in fitting arbitrary permittivity functions, modeling
nonlinear effects and arbitrary numbers of poles. The primitive ADE and Z-transform methods shown in the literature
require deriving formulations for each dispersion type, On the contrary, a generalized way [\textit{Alsunaidi et al.},
  2009] proposed finds its strength in unifying the formulations of different dispersion models into one form, which not
only performs concisely in mathmetics but also reduces source code in developing. The generalized ADE method
is chosen in this work.

The central idea of the Auxiliary Differential Equation (ADE) Method is to detach the phasor polarization displacement
$J_p(\omega)$, that is, $\chi_p(\omega)E(\omega)$, from the original dispersive relation, that is
\begin{equation}
  \begin{split}
    \widetilde{D}(\omega) &= \epsilon_r^*(\omega)\widetilde{E}(\omega)\\
    & = \epsilon_r\widetilde{E}(\omega) + \sum_{p=1}^P\chi_p(\omega)\widetilde{E}(\omega)\\
    & = \epsilon_r\widetilde{E}(\omega) + \sum_{p=1}^{P}J_p(\omega)
  \end{split}
\end{equation}
For the multipole material composed of different dispersions, $J_p$ is solved for each pole individually to attend the
updating loop, and then $E$ can be solved via rearrangment of the previous relation, that is,
\begin{equation}\label{eq:dispersive}
  \widetilde{E}^n = \frac{1}{\epsilon_r}\left(\widetilde{D}^n - \sum_{p=1}^PJ_p^n\right)
\end{equation}

Starting with the most general form of $\chi_p(\omega)$, the phasor polarization displacement can be written
as
\begin{equation}
  J_p(\omega) = \frac{a}{b + j\omega c - d\omega^2}\widetilde{E}(\omega)
\end{equation}
By rearranging and performming inverse Fourier transforms, the following equations is obtained.
\begin{equation}
  bJ_p(t) + c \frac{\partial}{\partial t}J_p(t) + d \frac{\partial ^2}{\partial t^2}J_p(t) = a\widetilde{E}(t)
\end{equation}
Applying the leapfrog scheme to replace $\partial/\partial t$ gives
\begin{equation}
  bJ_p^{n-1} + c \frac{J_p^n - J_p^{n-1}}{2\Delta t} + d \frac{J_p^n + 2J_p^{n-1} + J_p^{n-2}}{(\Delta t)^2} = a\widetilde{E}^{n-1}
\end{equation}
where $J_p$ can be solved as 
\begin{equation}\label{eq:polarized_displacement_math}
  J_p^n = \frac{4d-2b(\Delta t)^2}{2d+c\Delta t}J_p^{n-1} + \frac{-2d+c\Delta t}{2d+c\Delta t}J_p^{n-2} + \frac{2a(\Delta t)^2}{2d+c\Delta t}\widetilde{E}^{n-1}
\end{equation}
which can be written in the form 
\begin{equation}\label{eq:polarized_displacement}
  J_p^n = C_1 J_p^{n-1} + C_2 J_p^{n-2} + C_3 \widetilde{E}^{n-1}
\end{equation}
where $C_1$, $C_2$, and $C_3$ can be found for any form of the dispersion relation.


\paragraph{\msjh Coefficients of The Lorentz Pole}
The Lorentz pole completely matches the general form used in previous introduction of the ADE method. The coefficients
are as 
\begin{equation*}
  \begin{array}{@{}llll@{}}
    a = \Delta\epsilon_p\omega_p^2 &
    b = \omega_p^2 &
    c = 2\delta_p &
    d = 1
  \end{array}
\end{equation*}
By substituting into (\ref{eq:polarized_displacement_math})
\begin{gather*}
  \begin{array}{@{}lll@{}}
    C_1 = \frac{\displaystyle 2-\omega_p^2\Delta t^2}{\displaystyle 1+\delta_p\Delta t} &
    C_2 = \frac{\displaystyle -1 + \delta_p\Delta t}{\displaystyle 1+\delta_p\Delta t} &
    C_3 = \frac{\displaystyle \Delta\epsilon_p\omega_p^2\Delta t^2}{\displaystyle 1+\delta_p\Delta t}
  \end{array}
\end{gather*}


\paragraph{\msjh Coefficients of The Drude Poles}
The Drude pole lacks the b coefficient when comparing to the general form:
\begin{equation*}
  \begin{array}{@{}llll@{}}
    a = \omega_p^2 &
    b = 0 &
    c = \gamma_p &
    d = 1
  \end{array}
\end{equation*}
That is,
\begin{gather*}
  \begin{array}{@{}lll@{}}
    C_1 = \displaystyle\frac{4}{2+ \gamma_p\Delta t} &
    C_2 = \displaystyle\frac{-2+\gamma_p\Delta t}{2+\gamma_p\Delta t} &
    C_3 = \displaystyle\frac{2\omega_p^2(\Delta t)^2}{2+\gamma_p\Delta t}
  \end{array}
\end{gather*}





All coefficients of different dispersion types are summarized in Table 2.1
\begin{center}
  Table 2.1\\[0.3cm]
  \begin{tabular}[c]{rlccc}
    \hline
    Dispersion Type & Relation & $C_1$ & $C_2$ & $C_3$ \\[0.1cm]
    \hline\noalign{\smallskip}
    Lorentz 
    & $J_p= \displaystyle\frac{a}{b + jc\omega - d\omega^2}E$ 
    & $\displaystyle\frac{2-\omega_p^2\Delta t^2}{1+\delta_p\Delta t}$ 
    & $\displaystyle\frac{-1 + \delta_p\Delta t}{1+\delta_p\Delta t}$  
    & $\displaystyle\frac{\Delta\epsilon_p\omega_p^2\Delta t^2}{1+\delta_p\Delta t}$ \\[0.5cm]

    Drude 
    & $J_p = \displaystyle\frac{a}{jc\omega - d\omega^2}E$ 
    & $\displaystyle\frac{ 4}{ 2+\gamma_p\Delta t}$ 
    & $\displaystyle\frac{ -2+\gamma_p\Delta t}{ 2+\gamma_p\Delta t}$ 
    & $\displaystyle\frac{ 2\omega_p^2\Delta t^2}{ 2+\gamma_p \Delta t}$\\[0.5cm]

    Plasma 
    & $J_p = \displaystyle\frac{a}{\omega^2} E$ 
    & $2$ 
    & $-1$ 
    & $2a\Delta t^2$\\[0.5cm]

    Debye
    & $J_p = \displaystyle\frac{a}{b+jc\omega} E$ 
    & $\displaystyle\frac{-2b\Delta t}{c}$ 
    & $1$ 
    & $\displaystyle\frac{2a\Delta t}{c}$\\[0.5cm]

    Conductivity
    & $J_p = \displaystyle\frac{a}{jc\omega}E$ 
    & 0 
    & 1 
    & $\displaystyle\frac{2a\Delta t}{c}$\\[0.3cm]
    \hline
  \end{tabular}
\end{center}



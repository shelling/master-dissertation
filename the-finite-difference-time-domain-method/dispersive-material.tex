\section{Dispersive Material}
\label{sec:dispersive}
This section conerns of how to retrieve the update equations could apply in dispersive material. What would be discussed
below are only about material having dispersive electic permittivity. The material having dispersive magnetic
permeability could be investigated via duality.

In fundamental Electromagnetics, an indispersive material having a electrical polarization $P$ when there is a foreign
electric field and the phasor of electric flux would become $D(\omega) = \epsilon_0 E(\omega) + P$. $P$ was also defined
as $\epsilon_0 \chi_e E(\omega)$, so that we could rewrite $D(\omega)$ as $\epsilon_0 (1+\chi_e)E(\omega)$ and give the middle term a notation
$\epsilon_r$ named relative permittivity.

By definition, dispersive material is the material having different electic permeability when encountering EM wave
having different frequency that means the relative permittivity should be a function of frequency ordinarily coming with
image part. So, $\epsilon_r^*(\omega)$ was given as the notation of dispersive electic permittivity and the phasor of
electic flux in dispersive material was defined as $D(\omega) = \epsilon_0 \epsilon_r^*(\omega)E(\omega)$ or
$\widetilde{D} = \epsilon_r^*(\omega)\widetilde{E}$ in Gaussian Unit.
\subsection{Common Isotropic Dispersive Material}
The dispersive permittivity is defined as $\epsilon_r^*(\omega) = 1 + \chi_e + \sum \chi_p(\omega)$ and usually, $1 +
\chi_e$ are written as $\epsilon_r$ to be identical with fundamental electromagnetics. Many theorems are proposed to
describe the response characteristics of dispersive material exposed to external fields. Here are brief introduction to
Simple Lossy Media, Debye Media, Drude Media and Lorentz Media.

\subsubsection{Simple Lossy Media}
Simple Lossy Media is a novel material with constant electric conductivity $\sigma_e$, which respond a constant current
J to an external electric field. By substituting $J=\sigma E$ into Maxwell eqution, the relationship can be obtained as
following.
\begin{gather}
  \chi_p(\omega) = \frac{\sigma_e}{j\omega\epsilon_0}\\
  \epsilon_r^*(\omega) = \epsilon_r + \frac{\sigma_e}{j\omega\epsilon_0}
\end{gather}

\subsubsection{Debye Media}
Debye Model are used to describe the dieletric relaxation response of an noninteracting population of dipoles to an
external field. Debye pole and Multipole Debye Media can be describe as following
\begin{equation}
  \label{eq:debye_chi}
  \chi_p(\omega) = \frac{\Delta\epsilon_p}{1+j\omega\tau_p}  
\end{equation}
Multipole Debye Model has following form
\begin{equation}
  \epsilon_r^*(\omega) = \epsilon_r + \sum_{p=1}^P \frac{\Delta\epsilon_p}{1+j\omega\tau_p}  
\end{equation}


\subsubsection{Drude Media}
Drude Model are frequently used to explain optical properties of metal through the theory of free electron gas moving
against a fixed positive ion cores. The electrons oscillate in response to the applied field, and meantime damped due to
collision with characteristic frequency $\gamma_p$, also known as the inverse of the pole relaxation time. The motion
equation of free eletron gas can be written as
\begin{equation}
  m_e\frac{\partial^2 x(t)}{\partial t^2} + m_e\gamma_p\frac{\partial x(t)}{\partial t} = -eE(t)
\end{equation}
By assuming $x(t)$ and $E(t)$ are time-harmonic, $x(t) = \mathrm{Re}\{x_0e^{j\omega t}\}$ and $E(t) =
\mathrm{Re}\{E_0e^{j\omega t}\}$, the displacement of eletron can be solved.
\begin{equation}
  x(t) = \frac{e}{m_e(\omega^2 - j\omega\gamma_p)}E(t)
\end{equation}
Because the macroscopic polarization $P = -Ne\cdot x(t)$ should also satisfy $P = \epsilon_0\chi_pE(t)$, $\chi_p$ can be
solved
\begin{equation}
  \chi_p = \frac{-Ne^2/m_e\epsilon_0}{\omega^2 - j\omega\gamma_p}
\end{equation}
where $Ne^2/m_e\epsilon_0$ is usually denoted as plasma frequency $\omega_p^2$ to be the common form of Drude Model
\begin{equation}
  \label{eq:drude_chi}
  \chi_p(\omega) = -\frac{\omega_p^2}{\omega^2 - j\omega\gamma_p}  
\end{equation}
Multipole Drude Media can be denote as 
\begin{equation}
  \epsilon_r^*(\omega) = \epsilon_r - \sum_{p=1}^P \frac{\omega_p^2}{\omega^2-j\omega\gamma_p}
\end{equation}



\subsubsection{Lorentz Media}
Drude Model are not perfectly fitting experimental result at some frequency in which interband transitions involving
[Stefan]. The Lorentz Oscillator mechanism provides description to motion between nucleus and smaller mass, and
interband transitions through appending a damping term into the motion equation of free electron gas
\begin{equation}
    m_e\frac{\partial^2 x(t)}{\partial t^2} + 2m_e\delta_p\frac{\partial x(t)}{t} + m_e\omega_p^2x(t) = -eE(t)
\end{equation}
Assuming $x(t)$ and $E(t)$ are time-harmonic again, Lorentz Oscillator can be solve as following. 
\begin{equation}
  \label{eq:lorentz_chi}
  \chi_p(\omega) = \frac{\Delta\epsilon_p\omega_p^2}{\omega_p^2 + 2j\omega\delta_p - \omega^2}  
\end{equation}
\begin{equation}
  \epsilon_r^*(\omega) = \epsilon_r + \sum_{p=1}^P \frac{\Delta\epsilon_p\omega_p^2}{\omega_p^2 + 2j\omega\delta_p - \omega^2}  
\end{equation}
Where $\Delta\epsilon_p$ is a weighting factor associated with the specified Lorentz pole. It's often describing
dispersion in terms of linear combination of one Drude pole and multiple Lorentz poles to earn a better fitting.
[\textit{Taflove}].



\subsection{Dispersion-Compatible Update Equations}
For the analysis of dispersive materials shown above, numbers of algorithms have already been proposed in literature.
Most of these frequency-dependent algorithms can be categorized into three types:\begin{inparaenum}[(1)]
\item the Recursive Convolution (RC) method
\item the Auxiliary Differential Equation (ADE) method
\item the Z-transform (ZT) method\end{inparaenum}.
Any of these methods focuses on tranforming the dispersive relations between $D(\omega)$ and $E(\omega)$ in frequency
domain back to the time doamin for discretization.


The RC method is the most basic method inheriting the convolution theorem in Laplace transform and Fourier
transform. The pros and cons are easy to understand with background of engineering mathematics and difficult to handle
complex material with multiple poles. Many varieties, such as
\begin{inparaenum}[(1)]
\item Piecewise-Linear Recursive Convolution (PLRC) Method
\item Trapezoidal Recursive Convolution(TRC) Method
\end{inparaenum} are proposed to improve the performance of RC method.


The ADE method and Z-transform method offer high flexibility in fitting arbitrary permittivity functions, modeling
nonlinear effects and arbitrary numbers of poles. Primitive ADE method and Z-transform method shown in literature
require deriving formulations for each dispersion type, however, a generalized way \textit{Alsunaidi et al.} proposed
finds its strength in unifying the formulation of different dispersion models into one form. Not only performing
concisely in mathmetics but also reducing source code in developing, since, the generalized ADE method are picked up
here.

The central idea of the Auxiliary Differential Equation (ADE) Method is to detach the phasor polarization displacement
$J_p(\omega)$, that is, $\chi_p(\omega)E(\omega)$, from original dispersive relation.
\begin{equation}
  \begin{split}
    \widetilde{D}(\omega) &= \epsilon_r^*(\omega)\widetilde{E}(\omega)\\
    & = \epsilon_r\widetilde{E}(\omega) + \sum_{p=1}^P\chi_p(\omega)\widetilde{E}(\omega)\\
    & = \epsilon_r\widetilde{E}(\omega) + \sum_{p=1}^{P}J_p(\omega)
  \end{split}
\end{equation}
For multipole material composed of different dispersions, $J_p$ were solved for each pole individually to attend the
updating loop, and then $E$ can be solved via rearrangment of previous relation.
\begin{equation}\label{eq:dispersive}
  \widetilde{E}^n = \frac{1}{\epsilon_r}\left(\widetilde{D}^n - \sum_{p=1}^PJ_p^n\right)
\end{equation}

Starting with the most general form of $\chi_p(\omega)$, the phasor polarization displacement can be written
as
\begin{equation}
  J_p(\omega) = \frac{a}{b + j\omega c - d\omega^2}\widetilde{E}(\omega)
\end{equation}
Rearranging and performming inverse Fourier transform 
\begin{equation}
  bJ_p(t) + c \frac{\partial}{\partial t}J_p(t) + d \frac{\partial ^2}{\partial t^2}J_p(t) = a\widetilde{E}(t)
\end{equation}
Applying leapfrog scheme to replace $\partial/\partial t$
\begin{equation}
  bJ_p^{n-1} + c \frac{J_p^n - J_p^{n-1}}{2\Delta t} + d \frac{J_p^n + 2J_p^{n-1} + J_p^{n-2}}{(\Delta t)^2} = a\widetilde{E}^{n-1}
\end{equation}
$J_p$ can be Solved as following
\begin{equation}
  J_p^n = \frac{4d-2b(\Delta t)^2}{2d+c\Delta t}J_p^{n-1} + \frac{-2d+c\Delta t}{2d+c\Delta t}J_p^{n-2} + \frac{2a(\Delta t)^2}{2d+c\Delta t}\widetilde{E}^{n-1}
\end{equation}
which can be written in the form 
\begin{equation}\label{eq:polarized_displacement}
  J_p^n = C_1 J_p^{n-1} + C_2 J_p^{n-2} + C_3 \widetilde{E}^{n-1}
\end{equation}
where $C_1$, $C_2$ and $C_3$ can be found for any form of dispersion relation.


\paragraph{\msjh Coefficients of Lorentz Pole}
Lorentz pole completely matches the general form used in previous introduction of ADE method. The coefficients are as
following.
\begin{equation*}
  \begin{array}{@{}llll@{}}
    a = \Delta\epsilon_p\omega_p^2 &
    b = \omega_p^2 &
    c = 2\delta_p &
    d = 1
  \end{array}
\end{equation*}
substitute into ...
\begin{gather*}
  \begin{array}{@{}lll@{}}
    C_1 = \frac{\displaystyle 2-\omega_p^2\Delta t^2}{\displaystyle 1+\delta_p\Delta t} &
    C_2 = \frac{\displaystyle -1 + \delta_p\Delta t}{\displaystyle 1+\delta_p\Delta t} &
    C_3 = \frac{\displaystyle \Delta\epsilon_p\omega_p^2\Delta t^2}{\displaystyle 1+\delta_p\Delta t}
  \end{array}
\end{gather*}


\paragraph{\msjh Coefficients of Drude Pole}
Drude pole lacks the b coefficient when comparing to general form.
\begin{equation*}
  \begin{array}{@{}llll@{}}
    a = \omega_p^2 &
    b = 0 &
    c = \gamma_p &
    d = 1
  \end{array}
\end{equation*}
That is 
\begin{gather*}
  \begin{array}{@{}lll@{}}
    C_1 = \displaystyle\frac{4}{2+ \gamma_p\Delta t} &
    C_2 = \displaystyle\frac{-2+\gamma_p\Delta t}{2+\gamma_p\Delta t} &
    C_3 = \displaystyle\frac{2\omega_p^2(\Delta t)^2}{2+\gamma_p\Delta t}
  \end{array}
\end{gather*}





All coefficients of different dispersion types are summarized here
\begin{center}
  \begin{tabular}[c]{rlccc}
    \hline
    Dispersion Type & Relation & $C_1$ & $C_2$ & $C_3$ \\[0.1cm]
    \hline\noalign{\smallskip}
    Lorentz 
    & $J_p= \displaystyle\frac{a}{b + jc\omega - d\omega^2}E$ 
    & $\displaystyle\frac{2-\omega_p^2\Delta t^2}{1+\delta_p\Delta t}$ 
    & $\displaystyle\frac{-1 + \delta_p\Delta t}{1+\delta_p\Delta t}$  
    & $\displaystyle\frac{\Delta\epsilon_p\omega_p^2\Delta t^2}{1+\delta_p\Delta t}$ \\[0.5cm]

    Drude 
    & $J_p = \displaystyle\frac{a}{jc\omega - d\omega^2}E$ 
    & $\displaystyle\frac{ 4}{ 2+\gamma_p\Delta t}$ 
    & $\displaystyle\frac{ -2+\gamma_p\Delta t}{ 2+\gamma_p\Delta t}$ 
    & $\displaystyle\frac{ 2\omega_p^2\Delta t^2}{ 2+\gamma_p \Delta t}$\\[0.5cm]

    Plasma 
    & $J_p = \displaystyle\frac{a}{\omega^2} E$ 
    & $2$ 
    & $-1$ 
    & $2a\Delta t^2$\\[0.5cm]

    Debye
    & $J_p = \displaystyle\frac{a}{b+jc\omega} E$ 
    & $\displaystyle\frac{-2b\Delta t}{c}$ 
    & $1$ 
    & $\displaystyle\frac{2a\Delta t}{c}$\\[0.5cm]

    Conductivity
    & $J_p = \displaystyle\frac{a}{jc\omega}E$ 
    & 0 
    & 1 
    & $\displaystyle\frac{2a\Delta t}{c}$\\[0.3cm]
    \hline
  \end{tabular}
\end{center}



\section{The Algorithm}

In 1966, The algorithm of the FDTD method was first introduced by Yee.

\subsection{The update equations}

The update equations is the core of the FDTD method. In every interation on the timeline, the update equations refresh
the value of field of every point in the simulation region.

In this section, we are going to derive update equation from Maxwell's Equations.Here is the most well-known form of
Maxwell's Equations\index{Maxwell's Equations}:
\begin{gather}
  \label{eq:maxwell}
  \begin{array}{@{}rclr@{}}
    \nabla \cdot D & = & \rho_{\nu} & \mathrm{(Gaussian's\ Law)}\\
    \nabla \times E & = & {\displaystyle -\frac{\partial B}{\partial t}} & \mathrm{(Faraday's\ Law)}\\
    \nabla \cdot B & = & 0 & \\
    \nabla \times H & = & {\displaystyle J_s + \frac{\partial D}{\partial t}} & \mathrm{(Amp\`ere's\ Law)}
  \end{array}
\end{gather}
the equations shown above is for general lossy dielectric media.

for getting sysmetrical form, novel magnetic current $M$ was added into Faraday's Law
\begin{gather}
  \frac{\partial D}{\partial t} = \nabla \times H - J\\
  \frac{\partial B}{\partial t} =  - \nabla \times E - M
\end{gather}
in terms of E and H
\begin{gather}
  \epsilon\frac{\partial E}{\partial t} = \nabla \times H - \sigma_eE\\
  \mu\frac{\partial H}{\partial t} = - \nabla \times E - \sigma_hH
\end{gather}
gather coefficient 
\begin{gather}
  \frac{\partial E}{\partial t} = \frac{1}{\epsilon_r\epsilon_0}\nabla\times H - \frac{\sigma_e}{\epsilon_r\epsilon_0}E\\
  \frac{\partial H}{\partial t} = - \frac{1}{\mu_r\mu_0}\nabla\times E - \frac{\sigma_h}{\mu_r\mu_0}H
\end{gather}
turn to Gaussian Unit 
\begin{equation}
  \label{eq:gaussian_unit}
  \begin{array}{@{}l@{}}
    {\displaystyle\widetilde{E} = \sqrt{\frac{\epsilon_0}{\mu_0}}E}\\
    {\displaystyle\widetilde{D} = \frac{1}{\sqrt{\epsilon_0\mu_0}}D}\\
    {\displaystyle\widetilde{B} = \frac{1}{\mu_0}B}
  \end{array}
\end{equation}
\begin{gather}
  \frac{\partial \widetilde{E}}{\partial t} = \frac{1}{\epsilon_r\sqrt{\mu_0\epsilon_0}}\nabla\times H - \frac{\sigma_e}{\epsilon_r\epsilon_0}\widetilde{E}\\
  \frac{\partial H}{\partial t} = - \frac{1}{\mu_r\sqrt{\mu_0\epsilon_0}}\nabla\times\widetilde{E} - \frac{\sigma_h}{\mu_r\mu_0}H
\end{gather}
\begin{gather}
  \left(\epsilon_r\frac{\partial}{\partial t} + \frac{\sigma_e}{\epsilon_0}\right)\widetilde{E} = \frac{1}{\sqrt{\mu_0\epsilon_0}}\nabla\times H\\
  \left(\mu_r\frac{\partial}{\partial t} + \frac{\sigma_h}{\mu_0}\right)H = - \frac{1}{\sqrt{\mu_0\epsilon_0}}\nabla\times\widetilde{E}
\end{gather}
$\frac{\partial}{\partial t} \rightarrow j\omega$
\begin{gather}
  j\omega\left(\epsilon_r + \frac{\sigma_e}{j\omega\epsilon_0}\right)\widetilde{E} = c\ \nabla\times H\\
  j\omega\left(\mu_r + \frac{\sigma_h}{j\omega\mu_0}\right)H = - c\ \nabla\times\widetilde{E}
\end{gather}
update equations and constitute relations in general frequency-independent lossy dielectric media under Gaussian Units.
\begin{gather}
  \frac{\partial}{\partial t}\widetilde{D} = \frac{1}{\sqrt{\mu_0\epsilon_0}}\nabla\times H\label{eq:up_d}\\
  \widetilde{D}(\omega) = \left(\epsilon_r + \frac{\sigma_e}{j\omega\epsilon_0}\right)\widetilde{E} = \epsilon_r^*(\omega)\widetilde{E}(\omega)\label{eq:cr_d}\\
  \frac{\partial}{\partial t}\widetilde{B} = -\frac{1}{\sqrt{\mu_0\epsilon_0}}\nabla\times\widetilde{E}\label{eq:up_b}\\
  \widetilde{B}(\omega) = \left(\mu_r + \frac{\sigma_h}{j\omega\mu_0}\right)H = \mu_r^*(\omega)H(\omega)\label{eq:cr_b}
\end{gather}
It's obvious separating constitute relations from updating of electric flux $D$ and magnetic flux $B$. The
material-related coefficients were collected into constitute relations to handle different objects, so that no matter
what object was changed in region of simulation Eq.\ref{eq:up_d} and Eq.\ref{eq:up_b} keep in this form.

For example, The formulas can be simpilified to describe loseless dielectric media by setting $\sigma_e$, $\sigma_h$ as
zero in Eq.\ref{eq:cr_d} and Eq.\ref{eq:cr_b} to be
\begin{gather*}
  \widetilde{D}(\omega) = \epsilon_r\cdot\widetilde{E}(\omega)\\
  \widetilde{B}(\omega) = \mu_r\cdot H(\omega)
\end{gather*}
Performing inverse Fourier Transformation
\begin{gather*}
  \widetilde{D}(t) = \epsilon_r\cdot\widetilde{E}(t)\\
  \widetilde{B}(t) = \mu_r\cdot H(t)
\end{gather*}
but Eq.\ref{eq:up_d} and Eq.\ref{eq:up_b} need not any modification.

Or to describe freespace by setting $\sigma_e$, $\sigma_h$ as zero and $\epsilon_r$, $\mu_r$ as one 
\begin{gather*}
  \widetilde{D}(\omega) = \widetilde{E}(\omega)\\
  \widetilde{B}(\omega) = H(\omega)
\end{gather*}
Performing inverse Fourier Transformation again
\begin{gather*}
  \widetilde{D}(t) = \widetilde{E}(t)\\
  \widetilde{B}(t) = H(t)
\end{gather*}
In general, every material has its own $\epsilon_r^*(\omega)$ varying through whole frequency spectrum duo to its own
characters. By applying some mathematical trick Eq.\ref{eq:cr_d} and Eq.\ref{eq:cr_b} can be specialized for different
material to retrieve $E$ from $D$ in every time step but Eq.\ref{eq:up_d} and Eq.\ref{eq:up_b} can be applied directly
on every kinds of material. That's the best advanteage separating constitute relations out of the two update equations
would be introduce in \ref{sec:dispersive}.

This way also shows some advantagewhen handling perfecly matched layer which would be discussed in \ref{subsec:pml}.

Extend to Cartesian coordinate system.
\begin{gather}
  \frac{\partial}{\partial t}\widetilde{D}_x = \frac{1}{\sqrt{\mu_0\epsilon_0}}\left(\frac{\partial H_z}{\partial y} - \frac{\partial H_y}{\partial z}\right)\label{eq:up_d_x}\\
  \frac{\partial}{\partial t}\widetilde{D}_y = \frac{1}{\sqrt{\mu_0\epsilon_0}}\left(\frac{\partial H_x}{\partial z} - \frac{\partial H_z}{\partial x}\right)\label{eq:up_d_y}\\
  \frac{\partial}{\partial t}\widetilde{D}_z = \frac{1}{\sqrt{\mu_0\epsilon_0}}\left(\frac{\partial H_y}{\partial x} - \frac{\partial H_x}{\partial y}\right)\label{eq:up_d_z}\\
  \frac{\partial}{\partial t}\widetilde{B}_x =-\frac{1}{\sqrt{\mu_0\epsilon_0}}\left(\frac{\partial \widetilde{E}_z}{\partial y} - \frac{\partial \widetilde{E}_y}{\partial z}\right)\label{eq:up_b_x}\\
  \frac{\partial}{\partial t}\widetilde{B}_y =-\frac{1}{\sqrt{\mu_0\epsilon_0}}\left(\frac{\partial \widetilde{E}_x}{\partial z} - \frac{\partial \widetilde{E}_z}{\partial x}\right)\label{eq:up_b_y}\\
  \frac{\partial}{\partial t}\widetilde{B}_z =-\frac{1}{\sqrt{\mu_0\epsilon_0}}\left(\frac{\partial \widetilde{E}_y}{\partial x} - \frac{\partial \widetilde{E}_x}{\partial y}\right)\label{eq:up_b_z}
\end{gather}

\subsection{Reduction in Dimensions}
There are three selections to choose a one dimension EM string: $\mathrm{TEM_x}$ ($\mathrm{E_{y}}$, $\mathrm{H_{z}}$,
$\mathrm{k_x}$), $\mathrm{TEM_y}$ ($\mathrm{E_z}$, $\mathrm{H_x}$, $\mathrm{k_y}$), and $\mathrm{TEM_z}$
($\mathrm{E_x}$, $\mathrm{H_y}$, $\mathrm{k_z}$). Similarly, $\mathrm{TEM_z}$ is the default choice when saying
TEM. Following the definition of TEM, Eq.\ref{eq:up_d_x} and Eq.\ref{eq:up_b_y} were picked out for reduction of 1-D
case.
\begin{gather*}
  \frac{\partial}{\partial t}\widetilde{D}_x = \frac{1}{\sqrt{\mu_0\epsilon_0}}\left(\frac{\partial H_z}{\partial y} - \frac{\partial H_y}{\partial z}\right)\\
  \frac{\partial}{\partial t}\widetilde{B}_y =-\frac{1}{\sqrt{\mu_0\epsilon_0}}\left(\frac{\partial \widetilde{E}_x}{\partial z} - \frac{\partial \widetilde{E}_z}{\partial x}\right)
\end{gather*}
The choice implies
\begin{displaymath}
  \frac{\partial}{\partial x} \rightarrow 0\quad
  \frac{\partial}{\partial y} \rightarrow 0
\end{displaymath}
apply
\begin{gather}
  \frac{\partial}{\partial t}\widetilde{D}_x = \frac{1}{\sqrt{\mu_0\epsilon_0}}\left( - \frac{\partial H_y}{\partial z}\right)\\
  \frac{\partial}{\partial t}\widetilde{B}_y =-\frac{1}{\sqrt{\mu_0\epsilon_0}}\left(\frac{\partial \widetilde{E}_x}{\partial z} \right)
\end{gather}
Discrete
\begin{gather}
  \frac{\widetilde{D}_x|_k^{n+1/2} - \widetilde{D}_x|_k^{n-1/2}}{\Delta t} = -c_0\cdot\frac{H_y|_{k+1/2}^n - H_y|_{k-1/2}^n}{\Delta z}\\
  \frac{\widetilde{B}_y|_{k+1/2}^{n+1} - \widetilde{B}_y|_{k+1/2}^n}{\Delta t} = -c_0\cdot\frac{E_x|_{k+1}^{n+1/2} - E_x|_{k}^{n+1/2}}{\Delta z}
\end{gather}
Here is the code
\begin{code}
  code here
  lines
\end{code}





There are 6 selections for us to choose a two dimensions EM plane: $\mathrm{TM_{x}} $, $\mathrm{TE_{x}}$,
$\mathrm{TM_{y}}$, $\mathrm{TE_{y}}$, $\mathrm{TM_{z}}$, $\mathrm{TE_{z}}$. By default, the choice in this thesis follow
the book of Taflove using $\mathrm{TM_{z}}$ ($\mathrm{H_x}$, $\mathrm{H_y}$, and $\mathrm{E_z}$) and $\mathrm{TE_{z}}$
($\mathrm{E_x}$, $\mathrm{E_y}$, and $\mathrm{H_z}$) as convention when saying TM and TE.
\begin{displaymath}
  \frac{\partial}{\partial z} \rightarrow 0
\end{displaymath}
\begin{code}
  other code
  here
\end{code}




\subsection{Stability}
Courant Conditions, Courant Number
\begin{equation}
  \Delta t \le \frac{\Delta x}{\sqrt{n}\cdot c_0}
\end{equation}
where n is the dimension of the simulation. For the convenience of designing mentioned latter, throughout this thesis we determine
$\Delta t$ by
\begin{equation}
  \Delta t = \frac{\Delta x}{2 \cdot c_0}
\end{equation}

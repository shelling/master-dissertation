\section{The Algorithm}

In 1966, The algorithm of the FDTD method was first introduced by Yee.

\subsection{The update equations}

The update equations is the core of the FDTD method. In every interation on the timeline, the update equations refresh
the value of field of every point in the simulation region.

In this section, we are going to derive update equation from Maxwell's Equations.Here is the most well-known form of
Maxwell's Equations\index{Maxwell's Equations}:
\begin{gather}
  \label{eq:maxwell}
  \begin{array}{@{}rclr@{}}
    \nabla \cdot D & = & \rho_{\nu} & (Gaussian's\ Law)\\
    \nabla \times E & = & {\displaystyle -\frac{\partial B}{\partial t}} & (Faraday's\ Law)\\
    \nabla \cdot B & = & 0 & \\
    \nabla \times H & = & {\displaystyle J_s + \frac{\partial D}{\partial t}} & (Ampere's\ Law)
  \end{array}
\end{gather}
It is a long established fact that a reader will be distracted by the readable content of a page when looking at its
layout. The point of using Lorem Ipsum is that it has a more-or-less normal distribution of letters, as opposed to using
'Content here, content here', making it look like readable English. Many desktop publishing packages and web page
editors now use Lorem Ipsum as their default model text, and a search for 'lorem ipsum' will uncover many web sites
still in their infancy. Various versions have evolved over the years, sometimes by accident, sometimes on purpose
(injected humour and the like).

There are many variations of passages of Lorem Ipsum available, but the majority have suffered alteration in some form,
by injected humour, or randomised words which don't look even slightly believable. If you are going to use a passage of
Lorem Ipsum, you need to be sure there isn't anything embarrassing hidden in the middle of text. All the Lorem Ipsum
generators on the Internet tend to repeat predefined chunks as necessary, making this the first true generator on the
Internet. It uses a dictionary of over 200 Latin words, combined with a handful of model sentence structures, to
generate Lorem Ipsum which looks reasonable. The generated Lorem Ipsum is therefore always free from repetition,
injected humour, or non-characteristic words etc.
\begin{gather}
  \frac{\partial D}{\partial t} = \nabla \times H - J\\
  \frac{\partial B}{\partial t} =  - \nabla \times E - M
\end{gather}
in terms of E and H
\begin{gather}
  \epsilon\frac{\partial E}{\partial t} = \nabla \times H - \sigma_eE\\
  \mu\frac{\partial H}{\partial t} = - \nabla \times E - \sigma_hH
\end{gather}
gather coefficient 
\begin{gather}
  \frac{\partial E}{\partial t} = \frac{1}{\epsilon_r\epsilon_0}\nabla\times H - \frac{\sigma_e}{\epsilon_r\epsilon_0}E\\
  \frac{\partial H}{\partial t} = - \frac{1}{\mu_r\mu_0}\nabla\times E - \frac{\sigma_h}{\mu_r\mu_0}H
\end{gather}
turn to Gaussian Unit 
\begin{equation}
  \label{eq:gaussian_unit}
  \begin{array}{@{}l@{}}
    {\displaystyle\widetilde{E} = \sqrt{\frac{\epsilon_0}{\mu_0}}E}\\
    {\displaystyle\widetilde{D} = \frac{1}{\sqrt{\epsilon_0\mu_0}}D}\\
    {\displaystyle\widetilde{B} = \frac{1}{\mu_0}B}
  \end{array}
\end{equation}
\begin{gather}
  \frac{\partial \widetilde{E}}{\partial t} = \frac{1}{\epsilon_r\sqrt{\mu_0\epsilon_0}}\nabla\times H - \frac{\sigma_e}{\epsilon_r\epsilon_0}\widetilde{E}\\
  \frac{\partial H}{\partial t} = - \frac{1}{\mu_r\sqrt{\mu_0\epsilon_0}}\nabla\times\widetilde{E} - \frac{\sigma_h}{\mu_r\mu_0}H
\end{gather}
\begin{gather}
  \left(\epsilon_r\frac{\partial}{\partial t} + \frac{\sigma_e}{\epsilon_0}\right)\widetilde{E} = \frac{1}{\sqrt{\mu_0\epsilon_0}}\nabla\times H\\
  \left(\mu_r\frac{\partial}{\partial t} + \frac{\sigma_h}{\mu_0}\right)H = - \frac{1}{\sqrt{\mu_0\epsilon_0}}\nabla\times\widetilde{E}
\end{gather}
$\frac{\partial}{\partial t} \rightarrow j\omega$
\begin{gather}
  j\omega\left(\epsilon_r + \frac{\sigma_e}{j\omega\epsilon_0}\right)\widetilde{E} = c\ \nabla\times H\\
  j\omega\left(\mu_r + \frac{\sigma_h}{j\omega\mu_0}\right)H = - c\ \nabla\times\widetilde{E}
\end{gather}
update equations and constitute relations in general frequency-independent lossy dielectric media under Gaussian Units.
\begin{gather}
  \frac{\partial}{\partial t}\widetilde{D} = \frac{1}{\sqrt{\mu_0\epsilon_0}}\nabla\times H\label{eq:up_d}\\
  \widetilde{D}(\omega) = \left(\epsilon_r + \frac{\sigma_e}{j\omega\epsilon_0}\right)\widetilde{E} = \epsilon_r^*(\omega)\widetilde{E}(\omega)\label{eq:cr_d}\\
  \frac{\partial}{\partial t}\widetilde{B} = -\frac{1}{\sqrt{\mu_0\epsilon_0}}\nabla\times\widetilde{E}\label{eq:up_b}\\
  \widetilde{B}(\omega) = \left(\mu_r + \frac{\sigma_h}{j\omega\mu_0}\right)H = \mu_r^*(\omega)H(\omega)\label{eq:cr_b}
\end{gather}
It's obviously separating constitute relations from updating of electric flux $D$ and magnetic flux $B$. The formulas
can be simpilified to describe loseless dielectric media by setting $\sigma_e$, $\sigma_h$ as zero in Eq.\ref{eq:cr_d}
and Eq.\ref{eq:cr_b} to be
\begin{gather*}
  \widetilde{D}(\omega) = \epsilon_r\cdot\widetilde{E}(\omega)\\
  \widetilde{B}(\omega) = \mu_r\cdot H(\omega)
\end{gather*}
but Eq.\ref{eq:up_d} and Eq.\ref{eq:up_b} need not any modification.

Or to describe freespace by setting $\sigma_e$, $\sigma_h$ as zero and $\epsilon_r$, $\mu_r$ as one 
\begin{gather*}
  \widetilde{D}(\omega) = \widetilde{E}(\omega)\\
  \widetilde{B}(\omega) = H(\omega)
\end{gather*}
More advantages would be shown when handling perfecly matched layer and dispersive material which would be discussed in \ref{subsec:pml} and
\ref{sec:dispersive} respectively.







\subsection{Reduction of dimensions}
There are three selections to choose a one dimension EM string: $\mathrm{TEM_x}$ ($\mathrm{E_{y}}$, $\mathrm{H_{z}}$,
$\mathrm{k_x}$), $\mathrm{TEM_y}$ ($\mathrm{E_z}$, $\mathrm{H_x}$, $\mathrm{k_y}$), and $\mathrm{TEM_z}$
($\mathrm{E_x}$, $\mathrm{H_y}$, $\mathrm{k_z}$). Similarly, $\mathrm{TEM_z}$ is the default choice when saying TEM.
\begin{displaymath}
  \frac{\partial}{\partial x}\ \frac{\partial}{\partial y} \rightarrow 0
\end{displaymath}

There are 6 selections for us to choose a two dimensions EM plane: $\mathrm{TM_{x}} $, $\mathrm{TE_{x}}$,
$\mathrm{TM_{y}}$, $\mathrm{TE_{y}}$, $\mathrm{TM_{z}}$, $\mathrm{TE_{z}}$. By default, the choice in this thesis follow
the book of Taflove using $\mathrm{TM_{z}}$ ($\mathrm{E_x}$, $\mathrm{E_y}$, and $\mathrm{H_z}$) and $\mathrm{TE_{z}}$
($\mathrm{H_x}$, $\mathrm{H_y}$, and $\mathrm{E_z}$) as convention when saying TM and TE.
\begin{displaymath}
  \frac{\partial}{\partial z} \rightarrow 0
\end{displaymath}



\subsection{Stability}
Courant Conditions, Courant Number
\begin{equation}
  \Delta t \le \frac{\Delta x}{\sqrt{n}\cdot c_0}
\end{equation}
where n is the dimension of the simulation. For the convenience of designing mentioned latter, throughout this thesis we determine
$\Delta t$ by
\begin{equation}
  \Delta t = \frac{\Delta x}{2 \cdot c_0}
\end{equation}

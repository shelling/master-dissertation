\section{The Algorithm}

In 1966, The algorithm of the FDTD method was first introduced by Yee.

In the rest of this chapter, we would spare our effort on deriving a set of update equations as general as possible for
handling freespace, dielectics, dispersive material, and even meta-material at the same time rather than requiring
different equations in simulation for different situation. By using not only E and H field but also D and B field,
material-related terms were collected into constitutive relations, and the goal was achieved. Of course, it does run
more updating loops in one time step, however we acquire the advantages to prevent rewriting the program and to
concentrate on modeling of structure on researching.

\subsection{Finite Difference}
The first thing being concerned is how to discrete space and time in FDTD, in other word, how to turn differential
equations to algerba equations.
\subsubsection{Explicit Leapfrog Scheme}
Laylor's series expansion 
\begin{equation}
  u(x_i+\Delta x) = u|_{x_i} + 
  \Delta x\cdot\left.\frac{\partial u}{\partial x}\right|_{x_i} + 
  \frac{(\Delta x)^2}{2}\cdot\left.\frac{\partial ^2 u}{\partial x^2}\right|_{x_i} + 
  \frac{(\Delta x)^3}{6}\cdot\left.\frac{\partial ^3 u}{\partial x^3}\right|_{x_i} + ...
\end{equation}
\begin{equation}
  u(x_i-\Delta x) = u|_{x_i} -
  \Delta x\cdot\left.\frac{\partial u}{\partial x}\right|_{x_i} + 
  \frac{(\Delta x)^2}{2}\cdot\left.\frac{\partial ^2 u}{\partial x^2}\right|_{x_i} -
  \frac{(\Delta x)^3}{6}\cdot\left.\frac{\partial ^3 u}{\partial x^3}\right|_{x_i} + ...
\end{equation}
difference 
\begin{equation}
  u(x_i+\Delta x) - u(x_i-\Delta x) = 2\Delta x\cdot\left.\frac{\partial u}{\partial x}\right|_{x_i}+...
\end{equation}
\begin{equation}
  \left.\frac{\partial u}{\partial x}\right|_{x_i} = \frac{u(x_i+\Delta x) - u(x_i-\Delta x)}{2\Delta x} = \frac{u^{i+1} - u^{i-1}}{2\Delta x} + O[(\Delta x)^3]
\end{equation}
sum 
\begin{equation}
  u(x_i+\Delta x) + u(x_i-\Delta x) = \left.2u\right|_{x_i} + (\Delta x)^2\cdot\left.\frac{\partial ^2 u}{\partial x^2}\right|_{x_i} + ...
\end{equation}
\begin{equation}
  \left.\frac{\partial^2 u}{\partial x^2}\right|_{x_i} = \frac{u(x_i+\Delta x) - 2u(x_i) + u(x_i-\Delta x)}{(\Delta x)^2} = \frac{u^{i+1} - 2u^i + u^{i-1}}{(\Delta x)^2} % + O[(\Delta x)^2]
\end{equation}
\subsubsection{Semi-Implicit Scheme}
\begin{equation}
  \left.\frac{\partial u}{\partial x}\right|_{x_i} = \frac{u(x_i+\Delta x) - u(x_i)}{\Delta x}
\end{equation}

\subsection{The Update Equations}

The update equations is the core of the FDTD method. In every interation on the timeline, the update equations refresh
the value of field of every point in the simulation region.

In this section, we are going to derive update equation from Maxwell's Equations.Here is the most well-known form of
Maxwell's Equations\index{Maxwell's Equations}:
\begin{gather}
  \label{eq:maxwell}
  \begin{array}{@{}rclr@{}}
    \nabla \cdot D & = & \rho_{\nu} & \mathrm{(Gaussian's\ Law)}\\
    \nabla \times E & = & {\displaystyle -\frac{\partial B}{\partial t}} & \mathrm{(Faraday's\ Law)}\\
    \nabla \cdot B & = & 0 & \\
    \nabla \times H & = & {\displaystyle J_s + \frac{\partial D}{\partial t}} & \mathrm{(Amp\`ere's\ Law)}
  \end{array}
\end{gather}
the equations shown above is for simple conductive media, the simplest lossy media having constant conductivity.

for getting sysmetrical form, novel magnetic current $M$ was added into Faraday's Law
\begin{gather}
  \frac{\partial D}{\partial t} = \nabla \times H - J\\
  \frac{\partial B}{\partial t} =  - \nabla \times E - M
\end{gather}
in terms of E and H
\begin{gather}
  \epsilon\frac{\partial E}{\partial t} = \nabla \times H - \sigma_eE\\
  \mu\frac{\partial H}{\partial t} = - \nabla \times E - \sigma_hH
\end{gather}
gather coefficient 
\begin{gather}
  \frac{\partial E}{\partial t} = \frac{1}{\epsilon_r\epsilon_0}\nabla\times H - \frac{\sigma_e}{\epsilon_r\epsilon_0}E\\
  \frac{\partial H}{\partial t} = - \frac{1}{\mu_r\mu_0}\nabla\times E - \frac{\sigma_h}{\mu_r\mu_0}H
\end{gather}
turn to Gaussian Unit 
\begin{equation}
  \label{eq:gaussian_unit}
  \begin{array}{@{}l@{}}
    {\displaystyle\widetilde{E} = \sqrt{\frac{\epsilon_0}{\mu_0}}E}\\
    {\displaystyle\widetilde{D} = \frac{1}{\sqrt{\epsilon_0\mu_0}}D}\\
    {\displaystyle\widetilde{B} = \frac{1}{\mu_0}B}
  \end{array}
\end{equation}
\begin{gather}
  \frac{\partial \widetilde{E}}{\partial t} = \frac{1}{\epsilon_r\sqrt{\mu_0\epsilon_0}}\nabla\times H - \frac{\sigma_e}{\epsilon_r\epsilon_0}\widetilde{E}\\
  \frac{\partial H}{\partial t} = - \frac{1}{\mu_r\sqrt{\mu_0\epsilon_0}}\nabla\times\widetilde{E} - \frac{\sigma_h}{\mu_r\mu_0}H
\end{gather}
\begin{gather}
  \left(\epsilon_r\frac{\partial}{\partial t} + \frac{\sigma_e}{\epsilon_0}\right)\widetilde{E} = \frac{1}{\sqrt{\mu_0\epsilon_0}}\nabla\times H\\
  \left(\mu_r\frac{\partial}{\partial t} + \frac{\sigma_h}{\mu_0}\right)H = - \frac{1}{\sqrt{\mu_0\epsilon_0}}\nabla\times\widetilde{E}
\end{gather}
$\frac{\partial}{\partial t} \rightarrow j\omega$
\begin{gather}
  j\omega\left(\epsilon_r + \frac{\sigma_e}{j\omega\epsilon_0}\right)\widetilde{E} = c\ \nabla\times H\\
  j\omega\left(\mu_r + \frac{\sigma_h}{j\omega\mu_0}\right)H = - c\ \nabla\times\widetilde{E}
\end{gather}
update equations and constitute relations in simple conductive media under Gaussian Units.
\begin{gather}
  \frac{\partial}{\partial t}\widetilde{D} = \frac{1}{\sqrt{\mu_0\epsilon_0}}\nabla\times H\label{eq:up_d}\\
  \widetilde{D}(\omega) = \left(\epsilon_r + \frac{\sigma_e}{j\omega\epsilon_0}\right)\widetilde{E} = \epsilon_r^*(\omega)\widetilde{E}(\omega)\label{eq:cr_d}\\
  \frac{\partial}{\partial t}\widetilde{B} = -\frac{1}{\sqrt{\mu_0\epsilon_0}}\nabla\times\widetilde{E}\label{eq:up_b}\\
  \widetilde{B}(\omega) = \left(\mu_r + \frac{\sigma_h}{j\omega\mu_0}\right)H = \mu_r^*(\omega)H(\omega)\label{eq:cr_b}
\end{gather}
It's obvious separating constitute relations from updating of electric flux $D$ and magnetic flux $B$. The
material-related coefficients were collected into constitute relations to handle different objects, so that no matter
what object was changed in region of simulation Eq.\ref{eq:up_d} and Eq.\ref{eq:up_b} keep in this form.

For example, The formulas can be simpilified to describe loseless dielectric media by setting $\sigma_e$, $\sigma_h$ as
zero in Eq.\ref{eq:cr_d} and Eq.\ref{eq:cr_b} to be
\begin{gather*}
  \widetilde{D}(\omega) = \epsilon_r\cdot\widetilde{E}(\omega)\\
  \widetilde{B}(\omega) = \mu_r\cdot H(\omega)
\end{gather*}
Performing inverse Fourier Transformation
\begin{gather*}
  \widetilde{D}(t) = \epsilon_r\cdot\widetilde{E}(t)\\
  \widetilde{B}(t) = \mu_r\cdot H(t)
\end{gather*}
but Eq.\ref{eq:up_d} and Eq.\ref{eq:up_b} need not any modification.

Or to describe freespace by setting $\sigma_e$, $\sigma_h$ as zero and $\epsilon_r$, $\mu_r$ as one 
\begin{gather*}
  \widetilde{D}(\omega) = \widetilde{E}(\omega)\\
  \widetilde{B}(\omega) = H(\omega)
\end{gather*}
Performing inverse Fourier Transformation again
\begin{gather*}
  \widetilde{D}(t) = \widetilde{E}(t)\\
  \widetilde{B}(t) = H(t)
\end{gather*}
In general, every material has its own $\epsilon_r^*(\omega)$ varying through whole frequency spectrum duo to its own
characters. By applying some mathematical trick Eq.\ref{eq:cr_d} and Eq.\ref{eq:cr_b} can be specialized for different
material to retrieve $E$ from $D$ in every time step but Eq.\ref{eq:up_d} and Eq.\ref{eq:up_b} can be applied directly
on every kinds of material. That's the best advanteage separating constitute relations out of the two update equations
would be introduce in \ref{sec:dispersive}.

This way also shows some advantagewhen handling perfecly matched layer which would be discussed in \ref{subsec:pml}.

Extend to Cartesian coordinate system.
\begin{gather}
  \frac{\partial}{\partial t}\widetilde{D}_x = \frac{1}{\sqrt{\mu_0\epsilon_0}}\left(\frac{\partial H_z}{\partial y} - \frac{\partial H_y}{\partial z}\right)\label{eq:up_d_x}\\
  \frac{\partial}{\partial t}\widetilde{D}_y = \frac{1}{\sqrt{\mu_0\epsilon_0}}\left(\frac{\partial H_x}{\partial z} - \frac{\partial H_z}{\partial x}\right)\label{eq:up_d_y}\\
  \frac{\partial}{\partial t}\widetilde{D}_z = \frac{1}{\sqrt{\mu_0\epsilon_0}}\left(\frac{\partial H_y}{\partial x} - \frac{\partial H_x}{\partial y}\right)\label{eq:up_d_z}\\
  \frac{\partial}{\partial t}\widetilde{B}_x =-\frac{1}{\sqrt{\mu_0\epsilon_0}}\left(\frac{\partial \widetilde{E}_z}{\partial y} - \frac{\partial \widetilde{E}_y}{\partial z}\right)\label{eq:up_b_x}\\
  \frac{\partial}{\partial t}\widetilde{B}_y =-\frac{1}{\sqrt{\mu_0\epsilon_0}}\left(\frac{\partial \widetilde{E}_x}{\partial z} - \frac{\partial \widetilde{E}_z}{\partial x}\right)\label{eq:up_b_y}\\
  \frac{\partial}{\partial t}\widetilde{B}_z =-\frac{1}{\sqrt{\mu_0\epsilon_0}}\left(\frac{\partial \widetilde{E}_y}{\partial x} - \frac{\partial \widetilde{E}_x}{\partial y}\right)\label{eq:up_b_z}
\end{gather}
\clearpage
Discrete use semi-implicit scheme
\begin{gather}
  \begin{array}{@{}l@{}}
    \displaystyle \frac{\widetilde{D}_x|_{i+\frac{1}{2},j,k}^{n+\frac{1}{2}} - \widetilde{D}_x|_{i+\frac{1}{2},j,k}^{n-\frac{1}{2}}}{\Delta t} = \\
    \displaystyle c_0\left(\frac{H_z|_{i+\frac{1}{2},j+\frac{1}{2},k}^{n} - H_z|_{i+\frac{1}{2},j-\frac{1}{2},k}^{n}}{\Delta y} - \frac{H_y|_{i+\frac{1}{2},j,k+\frac{1}{2}}^{n} - H_y|_{i+\frac{1}{2},j,k-\frac{1}{2}}^{n}}{\Delta z}\right)\\[3em]
    \displaystyle \frac{\widetilde{D}_y|_{i,j+\frac{1}{2},k}^{n+\frac{1}{2}} - \widetilde{D}_x|_{i,j+\frac{1}{2},k}^{n-\frac{1}{2}}}{\Delta t} = \\
    \displaystyle c_0\left(\frac{H_x|_{i,j+\frac{1}{2},k+\frac{1}{2}}^{n} - H_x|_{i,j+\frac{1}{2},k-\frac{1}{2}}^{n}}{\Delta z} - \frac{H_z|_{i+\frac{1}{2},j+\frac{1}{2},k}^{n} - H_z|_{i-\frac{1}{2},j+\frac{1}{2},k}^{n}}{\Delta x}\right)\\[3em]
    \displaystyle \frac{\widetilde{D}_z|_{i,j,k+\frac{1}{2}}^{n+\frac{1}{2}} - \widetilde{D}_z|_{i,j,k+\frac{1}{2}}^{n-\frac{1}{2}}}{\Delta t} = \\
    \displaystyle c_0\left(\frac{H_y|_{i+\frac{1}{2},j,k+\frac{1}{2}}^{n} - H_y|_{i-\frac{1}{2},j,k+\frac{1}{2}}^{n}}{\Delta x} - \frac{H_x|_{i,j+\frac{1}{2},k+\frac{1}{2}}^{n} - H_x|_{i,j-\frac{1}{2},k+\frac{1}{2}}^{n}}{\Delta y}\right)\\[3em]
    \displaystyle \frac{\widetilde{B}_x|_{i,j+\frac{1}{2},k+\frac{1}{2}}^{n+1} - \widetilde{B}_x|_{i,j+\frac{1}{2},k+\frac{1}{2}}^{n}}{\Delta t} = \\
    \displaystyle - c_0\left(\frac{\widetilde{E}_z|_{i,j+1,k+\frac{1}{2}}^{n+\frac{1}{2}} - \widetilde{E}_z|_{i,j,k+\frac{1}{2}}^{n+\frac{1}{2}}}{\Delta y} - \frac{\widetilde{E}_y|_{i,j+\frac{1}{2},k+1}^{n+\frac{1}{2}} - \widetilde{E}_y|_{i,j+\frac{1}{2},k}^{n+\frac{1}{2}}}{\Delta z}\right)\\[3em]
    \displaystyle \frac{\widetilde{B}_y|_{i+\frac{1}{2},j,k+\frac{1}{2}}^{n+1} - \widetilde{B}_y|_{i+\frac{1}{2},j,k+\frac{1}{2}}^{n}}{\Delta t} = \\
    \displaystyle - c_0\left(\frac{\widetilde{E}_x|_{i+\frac{1}{2},j,k+1}^{n+\frac{1}{2}} - \widetilde{E}_x|_{i+\frac{1}{2},j,k}^{n+\frac{1}{2}}}{\Delta z} - \frac{\widetilde{E}_z|_{i+1,j,k+\frac{1}{2}}^{n+\frac{1}{2}} - \widetilde{E}_z|_{i,j,k+\frac{1}{2}}^{n+\frac{1}{2}}}{\Delta x}\right)\\[3em]
    \displaystyle \frac{\widetilde{B}_z|_{i+\frac{1}{2},j+\frac{1}{2},k}^{n+1} - \widetilde{B}_z|_{i+\frac{1}{2},j+\frac{1}{2},k}^{n}}{\Delta t} = \\
    \displaystyle - c_0\left(\frac{\widetilde{E}_y|_{i+1,j+\frac{1}{2},k}^{n+\frac{1}{2}} - \widetilde{E}_y|_{i,j+\frac{1}{2},k}^{n+\frac{1}{2}}}{\Delta x} - \frac{\widetilde{E}_x|_{i+\frac{1}{2},j+1,k}^{n+\frac{1}{2}} - \widetilde{E}_x|_{i+\frac{1}{2},j,k}^{n+\frac{1}{2}}}{\Delta y}\right)
  \end{array}
\end{gather}
Throughout this thesis, we use regular grid, that is, $\Delta x = \Delta y = \Delta z$. That means we could rewrite update equations as
\begin{equation}
  \begin{split}
    \widetilde{D}_x|_{i+\frac{1}{2},j,k}^{n+\frac{1}{2}} & = \widetilde{D}_x|_{i+\frac{1}{2},j,k}^{n-\frac{1}{2}}\\
    & + \frac{c_0\Delta t}{\Delta x}\left(H_z|_{i+\frac{1}{2},j+\frac{1}{2},k}^{n} - H_z|_{i+\frac{1}{2},j-\frac{1}{2},k}^{n} - H_y|_{i+\frac{1}{2},j,k+\frac{1}{2}}^{n} + H_y|_{i+\frac{1}{2},j,k-\frac{1}{2}}^{n}\right)
  \end{split}
\end{equation}
\begin{equation}
  \begin{split}
    \widetilde{D}_y|_{i,j+\frac{1}{2},k}^{n+\frac{1}{2}} & = \widetilde{D}_x|_{i,j+\frac{1}{2},k}^{n-\frac{1}{2}}\\
    & + \frac{c_0\Delta t}{\Delta x}\left(H_x|_{i,j+\frac{1}{2},k+\frac{1}{2}}^{n} - H_x|_{i,j+\frac{1}{2},k-\frac{1}{2}}^{n} - H_z|_{i+\frac{1}{2},j+\frac{1}{2},k}^{n} + H_z|_{i-\frac{1}{2},j+\frac{1}{2},k}^{n}\right)
  \end{split}
\end{equation}
\begin{equation}
  \begin{split}
    \widetilde{D}_z|_{i,j,k+\frac{1}{2}}^{n+\frac{1}{2}} & = \widetilde{D}_z|_{i,j,k+\frac{1}{2}}^{n-\frac{1}{2}}\\
    & + \frac{c_0\Delta t}{\Delta x}\left(H_y|_{i+\frac{1}{2},j,k+\frac{1}{2}}^{n} - H_y|_{i-\frac{1}{2},j,k+\frac{1}{2}}^{n} - H_x|_{i,j+\frac{1}{2},k+\frac{1}{2}}^{n} + H_x|_{i,j-\frac{1}{2},k+\frac{1}{2}}^{n}\right)
  \end{split}
\end{equation}
\begin{equation}
  \begin{split}
    \widetilde{B}_x|_{i,j+\frac{1}{2},k+\frac{1}{2}}^{n+1} & = \widetilde{B}_x|_{i,j+\frac{1}{2},k+\frac{1}{2}}^{n}\\
    & - \frac{c_0\Delta t}{\Delta x}\left(\widetilde{E}_z|_{i,j+1,k+\frac{1}{2}}^{n+\frac{1}{2}} - \widetilde{E}_z|_{i,j,k+\frac{1}{2}}^{n+\frac{1}{2}} - \widetilde{E}_y|_{i,j+\frac{1}{2},k+1}^{n+\frac{1}{2}} - \widetilde{E}_y|_{i,j+\frac{1}{2},k}^{n+\frac{1}{2}}\right)
  \end{split}
\end{equation}
\begin{equation}
  \begin{split}
    \widetilde{B}_y|_{i+\frac{1}{2},j,k+\frac{1}{2}}^{n+1} & = \widetilde{B}_y|_{i+\frac{1}{2},j,k+\frac{1}{2}}^{n}\\
    & - \frac{c_0\Delta t}{\Delta x}\left(\widetilde{E}_x|_{i+\frac{1}{2},j,k+1}^{n+\frac{1}{2}} - \widetilde{E}_x|_{i+\frac{1}{2},j,k}^{n+\frac{1}{2}} - \widetilde{E}_z|_{i+1,j,k+\frac{1}{2}}^{n+\frac{1}{2}} - \widetilde{E}_z|_{i,j,k+\frac{1}{2}}^{n+\frac{1}{2}}\right)
  \end{split}
\end{equation}
\begin{equation}
  \begin{split}
    \widetilde{B}_z|_{i+\frac{1}{2},j+\frac{1}{2},k}^{n+1} & = \widetilde{B}_z|_{i+\frac{1}{2},j+\frac{1}{2},k}^{n}\\
    & - \frac{c_0\Delta t}{\Delta x}\left(\widetilde{E}_y|_{i+1,j+\frac{1}{2},k}^{n+\frac{1}{2}} - \widetilde{E}_y|_{i,j+\frac{1}{2},k}^{n+\frac{1}{2}} - \widetilde{E}_x|_{i+\frac{1}{2},j+1,k}^{n+\frac{1}{2}} - \widetilde{E}_x|_{i+\frac{1}{2},j,k}^{n+\frac{1}{2}}\right)
  \end{split}
\end{equation}
This is complete update equations derived for 3D cases.
\begin{equation}
  k+\frac{1}{2}\rightarrow k\quad \mathrm{and} \quad
  k-\frac{1}{2}\rightarrow k-1
\end{equation}
Pseudo code
\begin{code}
Points.each \textcolor{blue}{do} 
  Dx[i,j,k] += 0.5 * ( Hz[i,j,k] - Hz[i,\textcolor{red}{j-1},k] 
                     - Hy[i,j,k] + Hy[i,j,\textcolor{red}{k-1}] )
  Dy[i,j,k] += 0.5 * ( Hx[i,j,k] - Hx[i,j,k-1] 
                     - Hz[i,j,k] + Hz[i-1,j,k] )
  Dz[i,j,k] += 0.5 * ( Hy[i,j,k] - Hy[i-1,j,k] 
                     - Hx[i,j,k] + Hx[i,j-1,k] )
  Bx[i,j,k] -= 0.5 * ( Ez[i,j+1,k] - Ez[i,j,k] 
                     - Ey[i,j,k+1] + Ey[i,j,k] )
  By[i,j,k] -= 0.5 * ( Ex[i,j,k+1] - Ex[i,j,k] 
                     - Ez[i+1,j,k] + Ez[i,j,k] )
  Bz[i,j,k] -= 0.5 * ( Ey[i+1,j,k] - Ey[i,j,k] 
                     - Ex[i,j+1,k] + Ex[i,j,k] )
end
\end{code}


\subsection{Stability}
Courant Conditions, Courant Number
\begin{equation}
  \Delta t \le \frac{\Delta x}{\sqrt{n}\cdot c_0}
\end{equation}
where n is the dimension of the simulation. For the convenience of designing mentioned latter, throughout this thesis we determine
$\Delta t$ by
\begin{equation}
  \Delta t = \frac{\Delta x}{2 \cdot c_0}
\end{equation}


\subsection{Reduction to One Dimensions}
There are three selections to choose a one dimension EM string: $\mathrm{TEM_x}$ ($\mathrm{E_{y}}$, $\mathrm{H_{z}}$,
$\mathrm{k_x}$), $\mathrm{TEM_y}$ ($\mathrm{E_z}$, $\mathrm{H_x}$, $\mathrm{k_y}$), and $\mathrm{TEM_z}$
($\mathrm{E_x}$, $\mathrm{H_y}$, $\mathrm{k_z}$). Similarly, $\mathrm{TEM_z}$ is the default choice when saying
TEM. Following the definition of TEM, Eq.\ref{eq:up_d_x} and Eq.\ref{eq:up_b_y} were picked out for reduction of 1-D
case.
\begin{gather*}
  \frac{\partial}{\partial t}\widetilde{D}_x = \frac{1}{\sqrt{\mu_0\epsilon_0}}\left(\frac{\partial H_z}{\partial y} - \frac{\partial H_y}{\partial z}\right)\\
  \frac{\partial}{\partial t}\widetilde{B}_y =-\frac{1}{\sqrt{\mu_0\epsilon_0}}\left(\frac{\partial \widetilde{E}_x}{\partial z} - \frac{\partial \widetilde{E}_z}{\partial x}\right)
\end{gather*}
The choice implies
\begin{displaymath}
  \frac{\partial}{\partial x} \rightarrow 0\quad \mathrm{and} \quad
  \frac{\partial}{\partial y} \rightarrow 0
\end{displaymath}
apply
\begin{gather}
  \frac{\partial}{\partial t}\widetilde{D}_x = \frac{1}{\sqrt{\mu_0\epsilon_0}}\left( - \frac{\partial H_y}{\partial z}\right)\\
  \frac{\partial}{\partial t}\widetilde{B}_y =-\frac{1}{\sqrt{\mu_0\epsilon_0}}\left(\frac{\partial \widetilde{E}_x}{\partial z} \right)
\end{gather}
Discrete
\begin{gather}
  \frac{\widetilde{D}_x|_k^{n+\frac{1}{2}} - \widetilde{D}_x|_k^{n-\frac{1}{2}}}{\Delta t} = -c_0\cdot\frac{H_y|_{k+\frac{1}{2}}^n - H_y|_{k-\frac{1}{2}}^n}{\Delta z}\\
  \frac{\widetilde{B}_y|_{k+\frac{1}{2}}^{n+1} - \widetilde{B}_y|_{k+\frac{1}{2}}^n}{\Delta t} = -c_0\cdot\frac{\widetilde{E}_x|_{k+1}^{n+\frac{1}{2}} - \widetilde{E}_x|_{k}^{n+\frac{1}{2}}}{\Delta z}
\end{gather}
That is
\begin{gather}
  \widetilde{D}_x|_k^{n+\frac{1}{2}} = \widetilde{D}_x|_k^{n-\frac{1}{2}} - \frac{c_0\Delta t}{\Delta z}\left( H_y|_{k+\frac{1}{2}}^n - H_y|_{k-\frac{1}{2}}^n \right)\\
  \widetilde{B}_y|_{k+\frac{1}{2}}^{n+1} = \widetilde{B}_y|_{k+\frac{1}{2}}^{n} = - \frac{c_0\Delta t}{\Delta z}\left( \widetilde{E}_x|_{k+1}^{n+\frac{1}{2}} - \widetilde{E}_x|_{k}^{n+\frac{1}{2}} \right)
\end{gather}
Here is the code
\begin{code}
Points.each do
  Dx[k] += 0.5 * ( Hy[k-1] - Hy[k] )
  Hy[k] += 0.5 * ( Ex[k] - Ex[k+1] )
end
\end{code}




\subsection{Reduction to Two Dimensions}
There are 6 selections for us to choose a two dimensions EM plane: $\mathrm{TM_{x}} $, $\mathrm{TE_{x}}$,
$\mathrm{TM_{y}}$, $\mathrm{TE_{y}}$, $\mathrm{TM_{z}}$, $\mathrm{TE_{z}}$. By default, the choice in this thesis follow
the book of Taflove using $\mathrm{TM_{z}}$ ($\mathrm{H_x}$, $\mathrm{H_y}$, and $\mathrm{E_z}$) and $\mathrm{TE_{z}}$
($\mathrm{E_x}$, $\mathrm{E_y}$, and $\mathrm{H_z}$) as convention when saying TM and TE.
\begin{displaymath}
  \frac{\partial}{\partial z} \rightarrow 0
\end{displaymath}

$\mathrm{TM_z}$
\begin{displaymath}
  \frac{\partial}{\partial t}\widetilde{D}_z = \frac{1}{\sqrt{\mu_0\epsilon_0}}\left(\frac{\partial H_y}{\partial x} - \frac{\partial H_x}{\partial y}\right)
\end{displaymath}
\begin{displaymath}
  \frac{\partial}{\partial t}\widetilde{B}_x =-\frac{1}{\sqrt{\mu_0\epsilon_0}}\left(\frac{\partial \widetilde{E}_z}{\partial y} - \frac{\partial \widetilde{E}_y}{\partial z}\right)
\end{displaymath}
\begin{displaymath}
  \frac{\partial}{\partial t}\widetilde{B}_y =-\frac{1}{\sqrt{\mu_0\epsilon_0}}\left(\frac{\partial \widetilde{E}_x}{\partial z} - \frac{\partial \widetilde{E}_z}{\partial x}\right)
\end{displaymath}
Discretize
\begin{displaymath}
  \frac{\widetilde{D}_z|_{i,j}^{n+\frac{1}{2}}-\widetilde{D}_z|_{i,j}^{n+\frac{1}{2}}}{\Delta t} =
  c_0 \left(\frac{H_y|_{i+\frac{1}{2},j}^{n} - H_y|_{i-\frac{1}{2},j}^n}{\Delta x} - \frac{H_x|_{i,j+\frac{1}{2}}^{n} - H_x|_{i,j-\frac{1}{2}}^{n}}{\Delta y}\right)
\end{displaymath}
\begin{displaymath}
  \frac{\widetilde{B}_x|_{i,j+\frac{1}{2}}^{n+1} - \widetilde{B}_x|_{i,j+\frac{1}{2}}^{n}}{\Delta t} = 
  - c_0\left(\frac{\widetilde{E}_z|_{i,j+1}^{n+\frac{1}{2}} - \widetilde{E}_z|_{i,j}^{n+\frac{1}{2}}}{\Delta y}\right)
\end{displaymath}
\begin{displaymath}
  \frac{\widetilde{B}_y|_{i+\frac{1}{2},j}^{n+1} - \widetilde{B}_y|_{i+\frac{1}{2},j}^{n}}{\Delta t} =
  - c_0\left( - \frac{\widetilde{E}_z|_{i+1,j}^{n+\frac{1}{2}} - \widetilde{E}_z|_{i,j}^{n+\frac{1}{2}}}{\Delta x}\right)
\end{displaymath}
Pseudo code for $\mathrm{TM_z}$ polarization
\begin{code}
Points.each do 
  
end
\end{code}

$\mathrm{TE_z}$
\begin{displaymath}
    \frac{\partial}{\partial t}\widetilde{B}_z =-\frac{1}{\sqrt{\mu_0\epsilon_0}}\left(\frac{\partial \widetilde{E}_y}{\partial x} - \frac{\partial \widetilde{E}_x}{\partial y}\right)\label{eq:up_b_z}
\end{displaymath}
\begin{displaymath}
  \frac{\partial}{\partial t}\widetilde{D}_x = \frac{1}{\sqrt{\mu_0\epsilon_0}}\left(\frac{\partial H_z}{\partial y} - \frac{\partial H_y}{\partial z}\right)
\end{displaymath}
\begin{displaymath}
  \frac{\partial}{\partial t}\widetilde{D}_y = \frac{1}{\sqrt{\mu_0\epsilon_0}}\left(\frac{\partial H_x}{\partial z} - \frac{\partial H_z}{\partial x}\right)
\end{displaymath}
Discretize
\begin{displaymath}
  \frac{\widetilde{B}_z|_{i+\frac{1}{2},j+\frac{1}{2}}^{n} - \widetilde{B}_z|_{i+\frac{1}{2},j+\frac{1}{2}}^{n}}{\Delta t} =
  c_0\left(\frac{\widetilde{E}_y|_{i+1,j+\frac{1}{2}}^{} - \widetilde{E}_y|_{i,j+\frac{1}{2}}^{}}{\Delta x} - \frac{\widetilde{E}_x|_{i+\frac{1}{2},j+1}^{} - \widetilde{E}_x|_{i+\frac{1}{2},j}^{}}{\Delta y}\right)
\end{displaymath}
\begin{displaymath}
  \frac{\widetilde{D}_x|_{i+\frac{1}{2},j}^{n+\frac{1}{2}} - \widetilde{D}_x|_{i+\frac{1}{2},j}^{n-\frac{1}{2}}}{\Delta t} =
  c_0\left(\frac{H_z|_{i+\frac{1}{2},j+\frac{1}{2}}^{n} - H_z|_{i+\frac{1}{2},j-\frac{1}{2}}^{n}}{\Delta y} - \right)
\end{displaymath}
\begin{displaymath}
  \frac{\widetilde{D}_y|_{i,j+\frac{1}{2}}^{n+\frac{1}{2}} - \widetilde{D}_x|_{i,j+\frac{1}{2}}^{n-\frac{1}{2}}}{\Delta t} =
  c_0\left( - \frac{H_z|_{i+\frac{1}{2},j+\frac{1}{2}}^{n} - H_z|_{i-\frac{1}{2},j+\frac{1}{2}}^{n}}{\Delta x}\right)
\end{displaymath}

Pseudocode for $\mathrm{TE_z}$ polarization
\begin{code}
Points.each do
end
\end{code}







